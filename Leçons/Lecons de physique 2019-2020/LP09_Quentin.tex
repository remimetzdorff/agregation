\section{LP09 Conversion de puissance électromécanique (Quentin)}

\paragraph{Bibliographie :}
\begin{itemize}
\item 
\end{itemize}

\paragraph{Niveau : CPGE} 

\paragraph{Pré-requis :}
\begin{itemize}
\item Magnétostatique ;
\item EM dans les milieux.
\end{itemize}

\subsection{Introduction}

La grève : il vaut mieux prendre le vélo : dynamo pour l'éclairage et assistance électrique avec le vélib.
Plusisuers apliccations, machines outils, transport etc.
Plusieurs éléments nécessaires : le fer

\paragraph{Diapo : } Canalisation des lignes de champ dans un ferromagnétique.
$\mu_r$ doit être très grand

Dans les machines il y a plusieurs éléments : rotor et stator

\subsection{Machine synchrone}

Repose sur un champ magnétique produit par le stator

\subsubsection{Champ statorique}

Soit $\overrightarrow{M}$ dans un champ magnétique $\overrightarrow{B}$.
On a un couple :
\begin{equation}
\overrightarrow{\Gamma} = \overrightarrow{M} \times \overrightarrow{B}.
\end{equation}
On  obtient un alignement du moment avec le champ B.
Il faut faire tourner le champ magnétique.

\paragraph{Diapo : } Trois bobine à 120 \degree.
Les courants doivent former un système triphasé équilibré.
\begin{equation}
I_1 = I_s\cos (\omega t)
\end{equation}
\begin{equation}
I_2 = I_s\cos (\omega t-\frac{2\pi}{3})
\end{equation}
\begin{equation}
I_3 = I_s\cos (\omega t-\frac{4\pi}{3})
\end{equation}
Ceci crée un champ tournant. (calcul du B créé par les spires) 
\paragraph{Diapo : } animation pour illustrer le champ tournant

\subsubsection{Energie magnétique et couple}

De la partie précédente, le champ statorique s'exprime
\begin{equation}
B_s(\theta, t) = B_s\cos(\omega_s t-\theta)
\end{equation}
Le champ du rotor est
\begin{equation}
B_r(\theta, t) = B_r\cos(\omega_r t-\theta+\alpha)
\end{equation}
Si $\mu_r$ est très grand (hypothèse initiale), l'énergie magnétique est contenue dans l'entrefer (entre le rotor et le stator).

Calcul de l'énergie électrique......

Le couple s'exprime en fonction de l'énergie :
\begin{equation}
\Gamma_z = \frac{\partial U_e}{\partial\alpha} = U_c\sin((\omega_r-\omega_s)t+\alpha)
\end{equation}
On voit que la moyenne du couple est non nulle ssi $\omega_r-\omega_s=0$ ce qui impose la synchronicité du fonctionnement du moteur.
Dans ce cas, on obtient un couple moteur :
\begin{equation}
\Gamma_z = U_c\sin(\alpha)
\end{equation}
 
\subsubsection{Couple résistif et stabilité}

Discussion sur le point de fonctionnement en fonction du couple résistif : cf cours Jérémy Neveu.
Dans le cas d'un couple résistif négatif, on obtient un fonctionnement en générateur.

Le fonctionnement de ce moteur impose un courant alternatif triphasé.
On aimerait pouvoir utiliser un courant DC pour des raisons de simplicité.

\subsection{Machine à courant continu}

\subsubsection{Couple}

\paragraph{Diapo : } Schéma du moteur DC.

\paragraph{Animation}

Principe de fonctionnement avec une spire dans un champ en montrant l'expression du couple total depuis les forces de Laplace :
\begin{equation}
\Gamma = 2ahI_rB_s\overrightarrow{e_z}\times\cos(\omega t)
\end{equation}
où a et h sont les dimensions de la spire.
En moyenne nul, il faut changer le sens de I et c'est le rôle du collecteur.
Dans ce cas la moyenne du couple de vient $\Phi_BI_r2/\pi$ où $\Phi$ est le flux de $B$ à travers la spire.

\subsubsection{CPEM et fem}

Calcul de la fem dans la spire tournante :
\begin{equation}
e(t) = -2ah\omega B_s\cos(\omega t)
\end{equation}

Calcul des puissances électrique et mécanique et somme des deux nulle.

\subsubsection{Généralités}

\paragraph{Exp :}

\subsection{Conclusion}

\subsection{Question}

\begin{itemize}
\item Quels sont les avantages des différents types de moteur ?
\item Comment fonctionnent les trains ?
\item Comment tourner plus vite que 50Hz avec un moteur synchrone ? Conversion de fréquence
\item Deux ingrédients importants : le fer et l'induction : sont-ils vraiment nécessaires tous les deux ?
\item Pourquoi un ferro canalise les lignes de champ ? Peux tu le redémontrer ?
\item D'où vient l'expression du couple exercé par un champ B sur un moment magnétique ?
\item Le rotor des moteurs : aimant permanent ou bobine ?
\item Pourquoi le triphasé ?
\item Pourquoi le fer doux ?
\item Quelles sont les origines des pertes dues aux ferromagnétique ?
\item Est ce pénible de feuilleter une carcasse de moteur ? Oui mais il faut
\item Problème au démarrage des moteurs synchrones ? cage d'écureuil


Travail des forces de Lorentz est nul mais pas celui des forces de Laplace. ??????? 



défaut et réalignement des domaines de Weiss
\end{itemize}