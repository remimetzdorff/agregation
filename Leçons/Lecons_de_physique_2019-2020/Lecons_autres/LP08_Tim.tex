\section{LP08 Phénomènes de transport (Timothé)}

\niveau CPGE

\prerequis
\begin{itemize}
\item thermodynamique à l'équilibre ;
\item mécanique des fluides ;
\item théorie cinétique des gaz.
\end{itemize}

\objectif Template

\footnotesize{\bibliography{biblio}}
\bibentry{Template}

\subsection{Introduction}

Ce sont les inhomogénéités du milieu qui provoquent des transport de quantité de matière, d'énergie.
En général le système n'est alors pas à l'équilibre ce qui rend son étude impossible en thermodynamique.

\subsection{Étude d'un système hors équilibre (2')}

L'étude des phénomènes physique dépend de l'échelle à laquelle on se place pour étudier les variations.

\subsubsection{Équilibre thermodynamique local ETL (3')}

On dira que l'ETL est réalisé si le système macroscopique à l'étude peut se décomposer en un ensemble de sous systèmes mésoscopiques.
Ces systèmes mésoscopiques doivent pouvoir être considérés à l'équilibre thermodynamique (où l'on peut définir des densités volumiques d'énergie, d'entropie).
Le volume du système $\delta V$ doit être grand pour que le nombre de particules qu'il contient $\delta N$ soit très grand mais aussi petit pour que le temps d'évolution du système soit faible devant le temps d'évolution du système.

\begin{slide}
\textbf{Trois échelles de travail.}
\end{slide}

Pour un gaz hors équilibre : quel est le temps pour que les efluves d'une bouteille de parfum cassée remplisse une pièce.
On souhaite connaitre le temps caractéristique pour que l'ETL soit atteint.
Ce qui favorise l'homogénéisation du système sont les collisions entre les molécules de gaz qui ammènent le système à l'équilibre.
En général une dizaine de collisions par particules sont suffisantes pour rétablir l'équilibre.
Le libre parcours moyen étant de l'ordre de \unit{100}{\nano\meter} d'où une echelle de longueur caractéristique de l'ordre de $\delta V = \unit{1}{\micro\meter}$.
Le temps de retour à l'équilibre est donc de l'ordre de
\begin{itemize}
\item $\overline{l}/u = 10^{-9} s$ pour un ETL à l'échelle mésoscopique ;
\item 10 h  pour la pièce
\end{itemize} 

\subsubsection{Différents modes de transport (14')}

\begin{slide}
\textbf{Différents modes de transport.}
\end{slide}

\subsection{Transport de particules}

\subsubsection{Bilan de matière et loi de conservation (16')}

Bilan de matière sur un petit volume de gaz avec les flux sortants.
On retrouve l'équation locale de conservation de la quantité de matière.

\subsubsection{Approche phénoménologique (20'30)}

1855 A Fick.
On relie $jn$ au gradient de la densité volumique de particule.
C'est une loi phénoménologique, valable dans l'approximation linéaire, pour $T$ et $P$ uniformes, isotrope.

\subsubsection{Equation de diffusion (23')}

On retrouve une équation de diffusion.

\begin{slide}
\textbf{Ordre de grandeur du coefficient de diffusion.}
\end{slide}

\subsection{Caractéristiques du phénomène de diffusion}

\begin{slide}
\textbf{Caractéristiques de l'équation de diffusion.}
A la différence de l'équation de d'Alembert, l'équation de diffusion n'est pas invariante par renversement du temps ce qui traduit un processus irréversible.
\end{slide}

Par analyse dimensionnelle on voit que l'échelle spatiale caractéristique est de l'ordre de $\sqrt{D\tau}$, ce qui rend les phénomènes de diffusion lents.

\begin{slide}
\textbf{Similitudes entre différents phénomènes de diffusion.}
\end{slide}

\begin{slide}
\textbf{Conductivité thermique du cuivre}
\end{slide}

\begin{experience}
\textbf{Conductivité thermique du cuivre}
\end{experience}

\subsection{Conclusion (40')}

\newpage