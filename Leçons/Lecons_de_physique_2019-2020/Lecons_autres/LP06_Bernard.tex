\section{LP06 Premier principe de la thermodynamique (Bernard)}

\niveau CPGE (1ère année)

\prerequis
\begin{itemize}
\item Gaz parfait ;
\item Capacité thermique ;
\item Travail de forces.
\end{itemize}

\objectif 

\footnotesize{\bibliography{biblio}}

\subsection{Introduction}

\begin{slide}
Rappels et contextualisation du niveau de la leçon.
\end{slide}

\subsection{Premier principe et applications}

\subsubsection{Définition du premier principe}

Explication des différentes origines de variation de l'énergie : 
\begin{itemize}
\item travail ;
\item transfert d'énergie thermique
\end{itemize}

\begin{slide}
Transfert d'énergie thermique :
\begin{itemize}
\item : conduction ;
\item convection ;
\item rayonnement.
\end{itemize}
\end{slide}

Rappel : algébrisation des transferts thermiques : on définit comme positif un transfert de l'extérieur vers le système.

\paragraph{Définition :}
Pour un système macroscopiquement au repos, fermé qui subit une transormation thermodynamique, il existe une fonction d'état $U$, appelée énergie interne telle que
\begin{equation}
\Delta U = U_f-U_i = W+Q
\end{equation}
et version infinitésimale.

\subsubsection{Un système mécanique isolé avec frottements 8'}

\begin{table}[!h]
\center
\begin{tabular}{l|l}
Etat initial & Etat final \\
\hline
$E_i = E_c^L + U_i^L + U_i^T$ & $E_f = U_f^L + U_f^T$
\end{tabular}
\end{table}

Avec les hypothèses adiabatique et sans travail extérieur, on applique le premier principe...

\subsubsection{Compression adiabatique brutale 12'}

\begin{slide}
Compression adiabatique brutale d'un gaz parfait diatomique
\end{slide}

Description des états initial et final en terme d'énergie, température, pression et volume.
Calcul pour aboutir à la température du gaz à l'état final.
Ordres de grandeur et vidéo de Veritassium sur la Fire syringe.

Les transferts thermiques et le travail restent des formes différentes de transfert d'énergie.

\subsubsection{Détente de Joule Gay-Lussac}

\begin{slide}
Détente de Joule Gay-Lussac.
\end{slide}

Description de l'état initial et de l'état final.
Transformation adiabatique, sans travail.
En appliquant le premier principe, on voit que la variation d'énergie interne est nulle ce qui implique que la transformation est isotherme.
Ceci peut être exploité pour vérifier qu'un gaz est parfait ou non.

\subsection{Cas d'une transformation isobare ou monobare avec équilibre mécanique aux instants initial et final (23')}

\subsubsection{Reformulation du premier principe 23'}

Définition de l'enthalpie

\subsubsection{Enthalpie 28'}

\subsubsection{Calcul de la capacité calorifique massique de l'eau}

\subsection{Questions}

\begin{itemize}
\item Que représente l'énergie l'interne ?  Comment définit-on l'énergie cinétique ?
\item L'énergie potentielle dépend du repère ? Pourquoi ce n'est pas le cas en thermo ? Le travail des forces intérieur à un système ne dépend pas du référentiel choisit.
\item
\end{itemize}

\begin{experience}
etst
\end{experience}

\begin{transition}
Pompélopie
\end{transition}
\begin{remarque}
Pompélopie
\end{remarque}
\note{Trop de la bombe}
