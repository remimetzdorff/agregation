\documentclass[12pt,a4paper]{article}
\usepackage[utf8]{inputenc}
\usepackage[french]{babel}
\usepackage[T1]{fontenc}
\usepackage{amsmath}
\usepackage{amsfonts}
\usepackage{amssymb}
\usepackage{graphicx}
\usepackage[left=2cm,right=2cm,top=2cm,bottom=2cm]{geometry}
\author{Rémi Metzdorff}

\newcommand{\ex}{\overrightarrow{e_x}}
\newcommand{\ey}{\overrightarrow{e_y}}
\newcommand{\ez}{\overrightarrow{e_z}}
\newcommand{\er}{\overrightarrow{e_r}}
\newcommand{\et}{\overrightarrow{e_\theta}}
\newcommand{\ep}{\overrightarrow{e_\phi}}
\renewcommand{\d}{\mathrm{d}}
\newcommand{\grad}{\overrightarrow{\mathrm{grad}}}
\newcommand{\rot}{\overrightarrow{\mathrm{rot}}}
\newcommand{\lap}{\Delta}
\renewcommand{\div}{\mathrm{div}}

\begin{document}

\begin{center}
\Large
\bf
Opérateurs vectoriels
\end{center}

\subsection*{Systèmes de coordonnées}

\begin{itemize}
\item cartésiennes : ($\ex$, $\ey$, $\ez$)
\item cylindriques : ($\er$, $\et$, $\ez$)
\item sphériques : ($\er$, $\et$, $\ep$)
\end{itemize}

\subsection*{Champ scalaire}

\begin{itemize}
\item $U(x,y,z)$
\item $U(r,\theta,z)$
\item $U(r, \theta,\phi)$
\end{itemize}

\subsection*{Champ vectoriel}

\begin{equation*}
\overrightarrow{A}(x,y,z) = A_x(x,y,z)\ex + A_y(x,y,z)\ey + A_z(x,y,z)\ez
\end{equation*}
\begin{equation*}
\overrightarrow{A}(r,\theta,z) = A_r(r,\theta,z)\er + A_\theta(r,\theta,z)\et + A_z(r,\theta,z)\ez
\end{equation*}
\begin{equation*}
\overrightarrow{A}(r,\theta,\phi) = A_r(r,\theta,\phi)\er + A_\theta(r,\theta,\phi)\et + A_z(r,\theta,\phi)\ep
\end{equation*}

\section*{Gradient $\grad$}

Il s'applique à un champ scalaire et est défini par
\begin{equation*}
\d U = \grad U.\overrightarrow{\d l}.
\end{equation*}
En coordonnées cartésiennes :
\begin{equation*}
\grad U = \overrightarrow{\nabla} U = \frac{\partial U}{\partial x}\ex + \frac{\partial U}{\partial y}\ey + \frac{\partial U}{\partial z}\ez
\end{equation*}
En coordonnées cylindriques :
\begin{equation*}
\grad U = \frac{\partial U}{\partial r}\er + \frac{1}{r}\frac{\partial U}{\partial \theta}\et + \frac{\partial U}{\partial z}\ez
\end{equation*}
En coordonnées sphériques :
\begin{equation*}
\grad U = \frac{\partial U}{\partial r}\er + \frac{1}{r}\frac{\partial U}{\partial \theta}\et + \frac{1}{r\sin\theta}\frac{\partial U}{\partial \phi}\ep
\end{equation*}

\paragraph*{Remarque :} Si $\overrightarrow{A} = -\grad U$, le vecteur $\overrightarrow{A}$ dérive du potentiel $U$ et est à circulation conservative.
\end{document}