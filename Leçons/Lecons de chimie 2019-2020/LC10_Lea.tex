\section{LP10 capteur électrochimiques}

\paragraph{Bibliographie :}
\begin{itemize}
\item ;
\end{itemize}

\paragraph{Niveau :} Lycée

\paragraph{Pré-requis :}
\begin{itemize}
\item Titrage, dosage par étallonage ;
\item Oxydoréduction ;
\item Loi d'Ohm
\end{itemize}

\paragraph{Objectifs de la leçon :}
\begin{itemize}
\item ;
\end{itemize}

\paragraph{Expériences :}
\begin{itemize}
\item ;
\end{itemize}

\subsection{Introduction}

Polution : polution des eaux usées des rejets industriels, pollution liée à des accidents (incedie à Notre Dame).
Pour contrôler l'état de la polution : capteur lectrochimiques

Un capteur électrochimique est un capteur qui permet de relier la concentration d'une espèce en solution à une grandeur électrique.
On s'intéresse à des solution ioniques : le courant est créé par un déplacement d'ions.
On peut faire de la conductimétrie, ou de la potetiométrie.

\subsection{Cellule conductimétrique}

Sous l'effet d'une ddp, les charges vont se déplacer

\subsubsection{Conductivité elctrique d'une solution}

La conductivité d'une solution est la capacité d'une solution chimique à conduire le courant notée $\sigma$.
Elle s'exprime en S.m$^{-1}$

\subsubsection{Cellule conductiùétrique}

La cellule conductiétrique est coposée de deux plaque métalliques et d'un ohmètre.
Elle permet de mesure $G=1/R$.
Le conductimètre relie la conductance mesurée à la conductivité de la solution :
\begin{equation}
\sigma = kG
\end{equation}
où $k$ est la constance de cellule

\paragraph{EXP :} mesure de la conductivité d'une solution de nitrate de plomb

\subsubsection{Loi de Kolraush}

\begin{equation}
\sigma = \lambda^0_\mathrm{PB^{2+}} [\mathrm{Pb^{2+}}] + ...
\end{equation}
Le coefficient $\lambda^0$ est a conductivité ionique molaire exprimée en S.m$^2$.mol$^{-1}$. La loi de Kolraush exprime les concentration en mol.m$^{-3}$.

\paragraph{EXP :} Mesure de la concentration en ion Pb d'une solution inconnue avec un dosage conductimétrique par étalonnage

\subsection{Capeur potetiométrique}

definition : voltmètre qui mesure la différente de potiel entre deux électrodes : electrode de travail et électrode de référence.
Une électrode est l'association d'un métal conducteur et d'une solution conenant les deux memebres d'un couple oxred.
Ex : électrode de zinc, Fer, électrode de référence à hydrogène

\subsubsection{Loi de Nernst}