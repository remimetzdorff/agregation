\section{LP06 Premier principe de la thermodynamique}

\begin{header}
\begin{tabular}{p{0.4\textwidth} l}
\niveau & \prerequis \\
Template& \textbullet{} Template \\
        & \textbullet{} Template \\
\end{tabular}

\noindent
\objectif
Template
\end{header}

{
\subsubsection*{Bibliographie}
\footnotesize{}
%\begin{itemize}
%\item 
%\end{itemize}
}

\begin{remarque}
Pour expliquer que $C_V$ soit définit à volume constant : on cherche à expliquer la variation d'énergie interne causée par un $\delta Q$ seulement.
Il faut donc annuler le travail des forces de pression en travaillant à volume constant.
\end{remarque}

\subsection*{Introduction}

\subsection{Template}

\subsubsection{Template}

\subsection*{Questions}

\begin{enumerate}
\item L'énergie est toujours conservée en physique : ça vient d'où ? Postulé en mécanique newtonienne et démontrable en mécanique lagrangienne par l'intermédiaire du théorème de Noether.
\item Quel est l'énergie de l'Univers ?
\item Qu'est ce qu'une fonction d'état ? C'est une fonction de variables d'état qui définissent l'état d'équilibre d'un système. Pour une transformation, elle ne dépend pas du chemin suivi.
\item Questions autour de variable/fonction d'état.
\item Qu'est ce qu'une variable extensive ?
\item Est ce que l'énergie interne est toujours extensive ? Non, il faut aussi que les interactions entre les particules du système soient à courte portée.
\item Qu'est ce qu'une transformation ? Passage d'un état d'équilibre à l'autre.
\item Qu'est ce qu'un état d'équilibre ? Équilibre thermique, mécanique et chimique.
\item Quelle est l'énergie cinétique et l'énergie potentielle du premier principe ? Énergie macroscopique du centre de masse
\end{enumerate}

\begin{experience}
\textbf{Template}
\end{experience}

\begin{slide}
\textbf{Template}
\end{slide}

\begin{transition}
\textbf{Template}
\end{transition}

\begin{remarque}
\textbf{Template}
\end{remarque}

\newpage