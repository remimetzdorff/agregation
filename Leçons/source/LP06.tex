\section{LP06 Premier principe de la thermodynamique}

\begin{header}
\begin{tabular}{p{0.4\textwidth} l}
\niveau & \prerequis \\
CPGE & \textbullet{} Modèle du gaz parfait \\
     & \textbullet{} Travail, transfert thermique \\
     & \textbullet{} Transformation thermodynamique \\
\end{tabular}

\noindent
\objectif
Template
\end{header}

{
\subsubsection*{Bibliographie}
\footnotesize{}
\begin{itemize}
\item \cite{Michel2017}
\item \cite{Olivier1998}
\item \cite{Salamito2016}
\item \cite{Diu2008}
\end{itemize}
}

\begin{remarque}
Relire le début de \cite{Diu2008} sur le postulat fondamental de la thermodynamique et p18 pour les variables d'état et équation d'état.
Un mot sur les fonctions d'état p66 et p143.
\end{remarque}

\subsection*{Introduction}

Partir du problème de la conservation de l'énergie mécanique avec l'exemple du livre lancé sur la table \cite{Salamito2016} p905 : ou va l'énergie ?
L'énergie mécanique n'est pas conservée mais l'énergie totale du système l'est.
L'augmentation de la température marque une augmentation de l'énergie microscopique que l'on va chercher à décrire dans cette leçon.

\subsection{Le premier principe}

\begin{remarque}
Équivalence travail-chaleur montrée par Joule en 1843.
Énoncé du premier principe par Mayer en 1845.
\end{remarque}

\subsubsection{Énoncé}

Exprimer la variation de l'énergie totale d'un système pour faire apparaitre l'énergie interne
\begin{equation}
\Delta E_c^\mathrm{macro} + \Delta E_p^\mathrm{macro} + \Delta U = W_\mathrm{nc}^\mathrm{ext} + Q.
\end{equation}
Relire la note de bas de page de \cite{Olivier1998} p132 pour expliquer qu'on suppose $W_\mathrm{nc}^\mathrm{micro} = 0$.
Revenir sur l'exemple introductif.

Se mettre dans le cadre du programme : pas de variation d'énergie potentielle et énergie interne extensive.
Restreindre au cas courant : $\Delta E_c^\mathrm{macro}=0$ et $W_\mathrm{nc}^\mathrm{ext}$ est le travail des forces de pression.
Marquer la différence entre les termes de gauche qui sont des fonctions d'état et les termes de droite qui dépendent du chemin suivi \cite{Michel2017} p533 puis passer à la forme infinitésimale.
\begin{remarque}
L'énergie interne n'est extensive que si les interactions à longue portée sont inexistante ou négligeable : c'est strictement vrai pour le gaz parfait mais pas dans les autres cas.
On peut penser à la tension superficielle notamment, qui est toujours négligée.
\end{remarque}

Insister sur la convention : travail et transfert thermique REÇU par le système comptés positivement.

\begin{transition}
L'énergie interne résulte essentiellement de l'agitation thermique : on souhaiterait relier $U$ et $T$.
\end{transition}

\subsubsection{Capacité calorifique}

Suivre \cite{Salamito2016} p839 et première loi de Joule p841.
Détailler le cas du GP et du solide (loi de Dulong et Petit 1819)

\begin{remarque}
\noindent
Pour expliquer que $C_V$ soit définit à volume constant : on cherche à expliquer la variation d'énergie interne causée par un $\delta Q$ seulement.
Il faut donc annuler le travail des forces de pression en travaillant à volume constant.
De même pour $C_P$, on souhaite connaitre la variation d'enthalpie causée par un transfert thermique.

\noindent
Le modèle d'Einstein permet d'expliquer la dépendance en température de la capacité calorifique mais il est remplacé par le modèle de Debye pour expliquer la variation aux basses températures.
\end{remarque}

\begin{transition}
Appliquons le premier principe dans le cas de quelques transformations classiques.
\end{transition}

\subsection{Quelques transformations}

\subsubsection{Détente de Joule Gay-Lussac}

\begin{slide}
\textbf{Détente de Joule Gay-Lussac.}
\end{slide}

Suivre \cite{Michel2017} p547.
Bien poser les hypothèses avant de partir dans le calcul.

\begin{remarque}
Si l'on fait l'expérience, on voit que juste après l'ouverture de la vanne, le gaz du récipient initialement plein se refroidit et celui du récipient vide se réchauffe.
En effet le gaz du récipient plein subit une détente adiabatique et celui du récipient vide subit une compression adiabatique.
Après retour à l'équilibre, la température est bien identique à la température initiale.
\end{remarque}

Dans le cas du gaz parfait, il n'y a pas d'élévation de la température ce ui peut servir à savoir si un gaz peut être considéré comme un GP.

\begin{remarque}
Relire l'exercice suivant sur la détente de Joule Thomson au cas où.
\end{remarque}

\begin{transition}
On ressent un échauffement lors du gonflage d'un pneu : peut-on l'expliquer avec le premier principe ?
\end{transition}

\subsubsection{Compression adiabatique}

Suivre \cite{Salamito2016} p903 sur le principe, mais sans force de frottement et en posant une masse $m$ de \unit{5}{\kilo\gram} sur un piston de surface $s=\unit{1}{\centi\meter\squared}$.
On trouve
\begin{equation}
T_f = T_i \frac{5P_0+2P_f}{7P_0} = \unit{729}{\kelvin},
\end{equation}
où $P_f = P_0+mg/s$ et $T_0=\unit{300}{\kelvin}$.

\begin{slide}
\textbf{\href{https://youtu.be/4qe1Ueifekg?t=146}{Fire syringe.}}
\end{slide}

À partir de cet état final, on laisse évoluer (isobare) le système qui va se refroidir en raison d'un transfert thermique $Q$.
Faire apparaitre :
\begin{equation}
\Delta(U+PV) = Q.
\end{equation}

\begin{transition}
Dans une évolution monobare ou isobare, la quantité $U+PV$ est conservée...
\end{transition}

\subsection{Enthalpie}

\subsubsection{Définition}

Suivre \cite{Salamito2016} p906.
Donner la définition et l'unité.
Il peut y avoir d'autres travaux que celui des forces de pression qu'il faut prendre en compte.

Introduire la capacité calorifique à pression constante et la donner pour un GP diatomique, relation de Mayer.
Donner le cas des phases condensées et justifier que $C_P = C_V$ par leur incompressibilité.

\begin{transition}
L'enthalpie se prête particulièrement à la calorimétrie comme nous allons le voir.
\end{transition}

\subsubsection{Calorimétrie}

Présenter le principe \cite{Salamito2016} p915 :
\begin{itemize}
\item vase Dewar : limite les transferts thermiques de conduction et radiatifs ;
\item couvercle percé : peu de convection mais expérience monobare ;
\item agitateur pour l'équilibre ;
\item thermomètre car c'est le seul instrument nécessaire.
\end{itemize}
Principe de la mesure de capacité thermique \cite{Salamito2016} p916.

\begin{remarque}
Ne pas oublier que le calorimètre est un composant du système : il faut prendre en compte sa capacité thermique ou la négliger sous les bonnes approximations.
\end{remarque}

\begin{transition}
Une application importante de la calorimétrie est la détermination des enthalpies de changement d'état.
\end{transition}

\subsubsection{Enthalpie de changement d'état}

Suivre \cite{Salamito2016} p917 et adapter au cas de présent.

\begin{experience}
\textbf{Enthalpie de vaporisation de l'azote liquide.}
\end{experience}

\subsection*{Conclusion}

L'expérience commune montre que les transferts thermiques se font du corps le plus chaud vers le corps le plus froids : deuxième principe.
Il est nécessaire pour comprendre le fonctionnement des machines thermiques.

\subsection*{Questions (FD)}

\begin{enumerate}
\item L'énergie est toujours conservée en physique : ça vient d'où ? Postulé en mécanique newtonienne et démontrable en mécanique lagrangienne par l'intermédiaire du théorème de Noether.
\item Quelle est l'énergie de l'Univers ?
\item Qu'est ce qu'une fonction d'état ? C'est une fonction de variables d'état qui définissent l'état d'équilibre d'un système. Pour une transformation, elle ne dépend pas du chemin suivi.
\item Questions autour de variable/fonction d'état.
\item Qu'est ce qu'une variable extensive ?
\item Est ce que l'énergie interne est toujours extensive ? Non, il faut aussi que les interactions entre les particules du système soient à courte portée.
\item Qu'est ce qu'une transformation ? Passage d'un état d'équilibre à l'autre.
\item Qu'est ce qu'un état d'équilibre ? Équilibre thermique, mécanique et chimique.
\item Quelle est l'énergie cinétique et l'énergie potentielle du premier principe ? Énergie macroscopique du centre de masse
\end{enumerate}

\newpage