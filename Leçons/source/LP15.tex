\section{LP15 Propagation quidée des ondes}

\begin{header}
\begin{tabular}{p{0.4\textwidth} l}
\niveau & \prerequis \\
CPGE & \textbullet{} Ondes électromagnétiques dans le vide \\
     & \textbullet{} Ondes acoustiques \\
     & \textbullet{} Interférences lumineuses
\end{tabular}

\noindent
\objectif
Template
\end{header}

{
\footnotesize{}
\subsection*{Bibliographie}
\begin{itemize}
\item \cite{Thibierge2014}
\item \cite{Olivier2000}, p771.
\item \cite{Perez2009}
\item \cite{Cardini2017}
\item \cite{Moreau1992}
\end{itemize}
}

\subsection*{Introduction}

Dans le cas d'une onde sphérique valable aussi pour une antenne, l'amplitude décroit en $1/r$ et l'énergie en $1/r^2$.
Le guidage permet d'éviter cette dilution de l'onde dans le milieu et de transporter de l'information sur de plus grandes distances.
Le guidage peut intervenir dans plusieurs types d'ondes et de fréquences.
Le guidage introduit cependant de la dispersion qu'il faudra prendre en compte dans les applications.

\subsection{Modèle du guide électromagnétique}

\subsubsection{Position du problème}

Globalement, on suivra \cite{Thibierge2014} à partir de la p51.

Faire un schéma et donner les hypothèses.
Établir l'équation de d'Alembert.
Poser les conditions aux limites.
Mettre en évidence l'existence de deux groupes de solutions TE et TM formant une base des solutions, notion de mode hybride.

\begin{transition}
On va s'intéresser aux solutions TE.
\end{transition}

\subsubsection{Solutions TE}

Justifier la forme des solutions cherchées par l'idée d'une solution propagative qui satisfasse l'équation de d'Alembert et les conditions aux limites.
Commenter sur la solution : stationnaire dans une direction et propagative dans l'autre.
Introduire la constante de propagation, faire le lien avec $k$ en marquant bien qu'il n'ont pas la même signification.

Aboutir sur la forme du champ $E$ en mentionnant que tout le reste s'exprime en fonction de lui.
Quantification des modes, caractérisé par un entier donné.

\begin{slide}
\textbf{Allure des solutions TE.}
\end{slide}

On peut mentionner la décomposition en OPPH pour l'analyse géométrique du problème.

\begin{transition}
Si les conditions aux limites ne modifient pas l'équation de propagation, elles imposent des conditions fortes sur les modes pouvant se propager dans le guide.
\end{transition}

\subsubsection{Relation de dispersion}

Établir la relation de dispersion.

\begin{slide}
\textbf{Relation de dispersion pour le guide plan-plan.}
\end{slide}

Mettre en évidence les notions de fréquence de coupure (le guide est un passe haut), avec l'interprétation si l'on envoie une OPPH sur le guide (\cite{Thibierge2014}, p56), de guidage monomode ou non.

Faire apparaitre la vitesse de phase et la vitesse de groupe et discuter de leur signification.

\begin{transition}
Ce modèle du guide a permis de mettre en évidence les particularités liées à la propagation guidée d'une onde (dispersion, mode et fréquence de coupure) mais il ne représente pas un outils pratique de la vie courante.
\end{transition}

\subsection{Guide d'onde réels}

\subsubsection{Guide microonde}

\cite{Olivier2000} p771.
\cite{Thibierge2014} p57.

Il s'agit d'un guide rectangulaire.
Les conditions aux limites sont similaires aux précédentes mais présentes dans deux directions.
Donner la relation de dispersion et reparler de la condition monomode, et de comment rendre un guide monomode à une fréquence fixée : en réduisant sa dimension transverse. 
Faire l'AN pour les microonde d'un four.
Justifier qu'un guide "plat" fonctionne.

\begin{transition}
Le guidage ne s'applique pas seulement au domaine des ondes électromagnétiques.
Voyons un exemple avec les ondes acoustiques.
\end{transition}

\subsubsection{Tuyau sonore}

Présenter les conditions aux limites dans le cadre de l'écoulement parfait :  vitesse tangentielle sur les parois.
Insister sur la particularité des ondes longitudinales : le mode fondamental n'est pas affecté, les autres modes ont une vitesse plus faible que dans l'air.
Donner la relation de dispersion.
La symétrie cylindrique complique la résolution analytique et fait intervenir les fonctions de Bessel dans les bases de solution.

\begin{experience}
\textbf{Mesure des vitesses de groupe des ondes acoustiques dans un tuyau sonore.}
Montrer le fondamental en déplaçant latéralement l'émetteur.
Incliner l'émetteur.
Faire avec plusieurs tuyaux.
\end{experience}

Cette expérience est assez abstraite mais on utilisait avant des cornets acoustiques comme amplificateurs.

\begin{transition}
On voit qu'envoyant un pulse dans un milieu dispersif, on limite le débit... ce qui est un problème dans les télécommunications actuelles qui se font par fibre optiques. 
\end{transition}

\subsubsection{Fibre optique}

Faire le raisonnement géométrique à partir de considérations d'interférences constructives. qui impose la quantification des modes.
Approche très simpliste mais met en évidence la dispersion : une impulsion est élargie.
On utilise des fibres à gradient d'indice pour limiter la dispersion.

\begin{slide}
\textbf{Fibre optique.}
\end{slide}

On pourrait utiliser des fibres monomode.
Les différents modes permettent de faire voyager plus d'information

\subsection*{Conclusion}

Modes, dispersion, télécommunications.

\newpage