\section{LP13 Ondes progressives, ondes stationnaires}

\begin{header}
\begin{tabular}{p{0.4\textwidth} l}
\niveau & \prerequis \\
Template& \textbullet{} Template \\
        & \textbullet{} Template \\
\end{tabular}

\noindent
\objectif
Template
\end{header}

\begin{remarque}
Taux d'onde stationnaire : rapport entre l'amplitude max (au niveau des ventres) et l'amplitude min (au niveau des nœuds).
Pour une onde progressive, il vaut 1, pour une onde parfaitement stationnaire, il tend vers l'infini.
\end{remarque}


{
\subsubsection*{Bibliographie}
\footnotesize{}
%\begin{itemize}
%\item 
%\end{itemize}
}


\subsection*{Introduction}

\subsection{Template}

\subsubsection{Template}

\begin{experience}
\textbf{Template}
\end{experience}

\begin{slide}
\textbf{Template}
\end{slide}

\begin{transition}
\textbf{Template}
\end{transition}

\begin{remarque}
\textbf{Template}
\end{remarque}

\newpage