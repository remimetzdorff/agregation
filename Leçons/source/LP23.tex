\section{LP23 Mécanismes de la conduction électrique dans les solides}

\begin{header}
\begin{tabular}{p{0.4\textwidth} l}
\niveau & \prerequis \\
Template& \textbullet{} Template \\
        & \textbullet{} Template \\
\end{tabular}

\noindent
\objectif
Template
\end{header}

{
\subsubsection*{Bibliographie}
\footnotesize{}
\begin{itemize}
\item je sais pas
\end{itemize}
}


\subsection*{Questions}

\begin{enumerate}
\item Comment savoir si un conducteur est un métal ou non ?
Définition chimique ou matériau dont les atomes sont unis par des liaisons métalliques.
Notion de bande.
\item Hypothèse derrière la loi d'Ohm locale ? Pas de gradient de température et matériau isotrope, réponse linéaire.
\item Dans le modèle de Drude, l'approche macroscopique amène le terme de frottement et donne une vitesse limite des électrons.
C'est équivalent de modélisation des chocs vient de la vitesse 
\end{enumerate}

\newpage