\section{LP05 Phénomènes interfaciaux impliquant des liquides}

\begin{header}
\begin{tabular}{p{0.4\textwidth} l}
\niveau & \prerequis \\
Licence & \textbullet{} Statique des fluides \\
        & \textbullet{} Potentiels thermodynamiques \\
        & \textbullet{} Interactions moléculaires
\end{tabular}

\noindent
\objectif
Décrire la tension superficielle et voir ses conséquences sur des interfaces statiques et dynamiques.
\end{header}

{
\subsection*{Bibliographie}
\footnotesize{}
\begin{itemize}
\item \cite{Rabaud2019}
\item \cite{Diu2008}
\item \cite{Marchand2011}
\item \cite{Sanz2016}
\item \cite{Guyon2001}
\item \cite{Graner2011}
\item \cite{Charru2007}
\end{itemize}
}

\begin{remarque}
Pas évident de placer cette leçon au niveau CPGE compte tenu du programme restreint sur cette notion.

\noindent
Rattacher les phénomènes décrits à des observations courantes !
\end{remarque}

\subsection*{Introduction}

Dans la vie courante, et depuis tout petit, on sait qu'il vaut mieux utiliser de l'eau savonneuse pour faire des bulles.
On observe que liquides mouillent différemment les parois solide, qu'il est possible de traiter certaines surfaces pour modifier son comportement par rapport à un fluide.
C'est effet sont liés à la tension de surface que nous allons décrire.

\subsection{La tension de surface}

On modélise l'interface entre deux fluides comme une surface séparant deux fluides...
Dire qu'il s'agit d'un volume dont l'épaisseur est de quelques molécules qui concentre les variation de densité, etc.
Voir la figure 5 de \cite{Marchand2011}.

\subsubsection{Interprétation mécanique}

Suivre \cite{Diu2008} p82-83 pour introduire la tension de surface depuis une expérience.

\begin{experience}
\textbf{Fil tendu par un film de savon.}
\end{experience}

Donner son expression et ses caractéristiques : direction, attractive et unité du coefficient.
Dire que cela ne dépend que de la nature de l'interface.

\begin{experience}
\textbf{Tensiomètre à lame mouillée.}
Passer sur les subtilités dues au mouillage total sur lesquelles on revient plus tard.
\end{experience}

Introduire le travail de la tension superficielle pour la transition.

\begin{transition}
On voit que la tension superficielle est liée à une énergie surfacique.
\end{transition}

\subsubsection{Interprétation thermodynamique}

Bien définir le système et les quantités constantes pour justifier l'emploi de tel potentiel thermodynamique : \cite{Graner2011} p100.
Deux façons de faire :
\begin{itemize}
\item à partir de l'énergie libre dans \cite{Marchand2011} p1 à $T$, $V$ et $n$ fixé ;
\item à partir de l'enthalpie libre dans \cite{Diu2008} p210 dans le cas d'une bulle, à $S$ et $V$ fixés.
\end{itemize}
La deuxième méthode est probablement plus propre et mais compliquée à ce stade je trouve.

Insister sur l'interprétation : l'augmentation de la surface d'une interface est couteuse en énergie.
Le système tend à minimiser la surface de l'interface.
Dans le cas ou plusieurs interfaces sont susceptibles de se former, la situation d'équilibre est celle d'énergie minimale.

\begin{slide}
\textbf{Surface minimale.}
\end{slide}

\begin{transition}
Poursuivons l'analyse microscopique du phénomène.
\end{transition}

\subsubsection{Interprétation microscopique}

Suivre \cite{Marchand2011} p4-5.
Prendre le temps pour expliquer la figure 6 de l'article.
Donner les ordres de grandeur de la fin de \cite{Marchand2011} p1, avec l'interprétation en terme d'énergie d'interaction dans le liquide.
\begin{slide}
\textbf{Quelques valeurs du coefficient de tension superficielle.}
\end{slide}

Facteurs influençant $gamma$ :
\begin{itemize}
\item tensioactifs : \cite{Rabaud2019} p81 ;
\item température : \cite{Rabaud2019} p80.
\end{itemize}

\begin{remarque}
Attention à bien justifier que la résultante est parallèle à l'interface.

\noindent
A ce stade, il est normal que l'enthalpie de vaporisation soit liée à $\gamma$ puisque les mêmes interactions microscopiques interviennent.
\end{remarque}

\begin{transition}
Quel est l'effet de la tension superficielle sur l'équilibre des interfaces entre fluides ?
\end{transition}

\subsection{Interface statique}

\subsubsection{Pression}

Retrouver la loi de Laplace avec a force de tension superficielle \cite{Sanz2016} p328.
Citer des exemples pratiques :
\begin{itemize}
\item cohésion du sable mouillé ;
\item retard à l'ébullition \cite{Guyon2001} p57 ;
\item bulle de savon qui se vide dans une plus grande.
\end{itemize}

\begin{remarque}
La formule n'est valable que dans le cas d'une sphère.
Dans le cas général, il faut tenir compte des rayons de courbure dans les deux direction de l'espace normale à l'interface :
\begin{equation}
P_\mathrm{int} - P_\mathrm{ext} = \gamma \left( \frac{1}{R_x} + \frac{1}{R_y} \right).
\end{equation}
Attention aussi au cas de la bulle qui possède deux interfaces.
\end{remarque}

Insister sur l'interprétation qualitative :
\begin{itemize}
\item la surpression est d'autant plus grande que le rayon de courbure est faible ;
\item la pression est plus élevée à l'intérieur de la courbure.
\end{itemize}

\begin{transition}
La tension superficielle permet aussi d'expliquer la capillarité.
\end{transition}

\subsubsection{Capillarité}

Retrouver la loi de Jurin avec l'interprétation énergétique \cite{Sanz2016} p329.

\begin{slide}
\textbf{Loi de Jurin.}
\end{slide}

\begin{experience}
\textbf{Loi de Jurin.}
\end{experience}

\begin{transition}
Utiliser la fin de la slide : la forme de l'interface dépend des coefficients de tension superficielle mis en jeu.
\end{transition}

\subsubsection{Mouillage}

Retrouver la relation de Young-Dupré à partir des deux lois précédentes \cite{Sanz2016} et discuter des différents régimes.

\begin{remarque}
Pour les mesures de l'angle de mouillage, voir \cite{Guyon2001}.

\noindent
Relire \cite{Marchand2011} p7 pour l'interprétation des interactions solide-liquide.
\end{remarque}

\begin{slide}
\textbf{Mouillage.}
\end{slide}

\begin{transition}
Utiliser la fin de la slide avec le mercure pour introduire qualitativement la longueur capillaire.
Elle intervient dans de nombreux phénomènes dynamiques.
\end{transition}

\subsection{Interface dynamique}

\subsubsection{Instabilité de Rayleigh-Taylor}

\begin{experience}
\textbf{Bouteille remplie d'eau et tube fin bouché à une extrémité.}
But comprendre pourquoi l'une des situations est stable et pas l'autre.
\end{experience}
Suivre \cite{Guyon2001} p68-70 et \cite{Charru2007} p53-55.
Donner la longueur capillaire de l'eau.

\subsubsection{Ondes capillaires}

\begin{slide}
\textbf{Ondes à la surface d'un fluide.}
Faire l'analyse qualitative du problème avant de commencer pour mettre en évidence la compétition entre tension superficielle et gravité.
\end{slide}
Faire apparaitre la longueur capillaire.
Dire que l'on peut mesurer aussi la tension superficielle en mesurant cette relation de dispersion.


\subsection*{Conclusion}

On a mis en évidence les propriétés et les manifestations de la tension superficielle qui permettent d'expliquer de nombreux phénomènes de la vie courante, mais aussi des phénomènes plus complexes.

\begin{remarque}
Autres méthodes de mesure de la tension superficielle :
\begin{itemize}
\item méthode d'arrachement, ou méthode de DuNoüy : avec un anneau ;
\item méthode de la goutte pendante ;
\item loi de Jurin ;
\item relation de dispersion des ondes de surface.
\end{itemize}
\end{remarque}

\newpage