\section{LP04 Synthèse inorganique}

\paragraph{Bibliographie :}
\begin{itemize}
\item Site web : \href{https://www.education.gouv.fr/pid25535/bulletin_officiel.html?cid_bo=57629}{Bulletin officiel} ;
\item Site web : \href{http://sciences-physiques-et-chimiques-de-laboratoire.org/course/view.php?id=7&section=16}{Livre numérique de Terminal STL} ;
\item Site web : \href{https://www.lelementarium.fr/product/eau-de-javel/}{Production industrielle de l'eau de Javel} ;
\item Site web : \href{https://www.eurochlor.org/about-chlor-alkali/how-are-chlorine-and-caustic-soda-made/}{How are chlorine and caustic soda made?} ;
\item \cite{Buchere2017} ;
\item \cite{Cachau-Hereillat2011} ;
\item \cite{Fosset2016} ;
\item \cite{Fosset2014} ;
\end{itemize}

\paragraph{Niveau :} Lycée (STL - SPCL)

\paragraph{Pré-requis :}
\begin{itemize}
\item Dosages par titrage, étalonnage ;
\item Structure de Lewis ;
\item Constante d'équilibre ;
\item Électrolyse
\end{itemize}

\paragraph{Objectifs de la leçon :}
\begin{itemize}
\item Synthèses inorganiques industrielles : aspects cinétiques, thermodynamiques, environnementaux ;
\item Réaction de formation d'un complexe, constante de formation globale d'un complexe, synthèse et analyse d'un complexe ;
\item Complexes inorganiques, bio-inorganiques.
\end{itemize}

\paragraph{Expériences :}
\begin{itemize}
\item Synthèse de l'eau de Javel par électrolyse de NaCl \cite{Cachau-Hereillat2011} p.337 ;
\item Révélation de quelques cations métalliques de transition \cite{Buchere2017} p.263 ;
\item Synthèse du complexe $\mathrm{K_3[Fe(C_2O_4)_3],3H_2O}$ \cite{Buchere2017} p.291.
\end{itemize}


\subsection{Introduction}
Par synthèse, on sous-entend la création d'une nouvelle espèce chimique par transformation d'un ou plusieurs réactifs. 
Dans cette leçon on s'intéresse à la synthèse de composés inorganiques, i.e. pas centrés autour d'un squelette carboné (qui relève du domaine de la chimie organique).
Historiquement, c'est ce qu'on appelle la chimie minérale, même si ses frontières sont parfois ténues, notamment comme on le verra quand on s'intéresse à des complexes faisant intervenir des ligands organiques.

On s'intéressera tout d'abord à la synthèse de composés simples à travers l'exemple de la synthèse de l'ion hypochlorite de l'eau de Javel, puis on introduira de nouveau assemblages atomiques avec les complexes dont on verra un exemple de synthèse.

\subsection{Synthèse de composés simples}
\subsubsection{Synthèse de l'eau de Javel en laboratoire}

Un peu d'histoire :
\begin{itemize}
\item $\sim 1785$ : blanchiment au dichlore ;
\item $\mathrm{Cl_2}$ obtenu par oxydation de l'acide chlorhydrique le dioxyde de manganèse $$\mathrm{MnO_2 + 4HCl \rightarrow MnCl_2 + Cl_2 + H_2O}$$;
\item blanchiment de toiles ;
\item blanchiment de papier ;
\item production actuelle : $1,6$ millions de tonnes en 2017.
\end{itemize}

\paragraph{Diapo : } Schéma de la manip.

On peut synthétiser l'ion hypochlorite par électrolyse de la saumure.
Sur la cathode on observe la réduction de l'eau :
\begin{equation*}
\mathrm{H_2O + 2e^- \rightarrow H_2 + 2HO^-}
\end{equation*}
et sur l'anode l'oxydation des ions chlorure :
\begin{equation*}
\mathrm{2Cl^- \rightarrow Cl_2 + 2e^-}
\end{equation*}
Sous agitation, on peut ainsi dissoudre le dichlore dans une solution basique qui conduit par dismutation à :
\begin{equation*}
\mathrm{Cl_2 + 2HO^- \rightarrow Cl^- + ClO^- + H_2O} 
\end{equation*}

\paragraph{Expérience : }
\begin{itemize}
\item lancer l'électrolyse dès le début de la leçon ;
\item mettre en évidence la formation de $\mathrm{ClO^-}$ avec l'iodure de potassium + empois d'amidon ;
\begin{equation*}
\mathrm{ClO^- + H_2O + 2I^- \rightarrow I_2 + Cl^- + 2HO^-} 
\end{equation*}
\item comparer à un prélèvement avant l'électrolyse et un prélèvement de la préparation.
\end{itemize}

\paragraph{Transition : } Ce processus ne permet pas la production d'eau de Javel à grande échelle.
Qu'en est-il des méthodes de production industrielles ?

\subsubsection{Synthèse industrielle}

\paragraph{Diapo :} Synthèse industrielle de l'eau de Javel.
Comparaison des différentes méthodes et un mot sur le réacteur ouvert.

\paragraph{Transition : } On a vu que les méthodes de production s'efforcent d'être plus en accord avec les enjeux environnementaux de notre époque.
Un autre exemple qui illustre cette préoccupation envers les problématiques environnementales est celui de la synthèse de l'ammoniac.

\subsubsection{Vers des synthèses plus vertes}

Production actuelle : plus de 100 millions de tonnes par an, utilisé dans les engrais, les explosifs, les carburants, polymères, etc.
Sa synthèse repose sur le procédé Haber-Bosch développé au début du $\mathrm{XX^e}$ siècle, par réaction directe de diazote et dihydrogène en présence d'un catalyseur (Fer $\alpha$), à haute température (\unit{450}{\celsius}) et haute pression (\unit{250}{bar}) :
\begin{equation*}
\mathrm{N_{2(g)}} + 3\mathrm{H_{2(g)}} \rightarrow 2\mathrm{NH_{3(g)}}
\end{equation*}
L'idéal serait de parvenir à s'inspirer de la nature où l'on trouve de nombreuses plantes capables de réaliser cette transformation sans avoir besoin d'une telle énergie, par catalyse enzymatique.

La difficulté est de rompre la triple liaison du diazote. 
Pour cela, certains progrès récents proposent l'utilisation de complexes organométalliques.

\paragraph{Transition : } Que sont les complexes et comment les synthétiser.

\subsection{Complexes}

\subsubsection{Définitions}

\paragraph{Complexe : } édifice polyatomique formé d'un centre métallique (souvent un cation d'un métal de transition) autour duquel sont liés (coordonnées ou coordinés) des molécules ou anions appelés ligands.

\paragraph{Diapo :} Exemple de complexe.

L'ion central est un accepteur d'électrons :
\begin{itemize}
\item fer(II), fer(III) ;
\item cuivre(I), cuivre(II) ;
\item cobalt(II)...
\end{itemize}
alors que les ligands sont donneurs d'électrons, ce qui permet de former une ou plusieurs liaison(s) par partage de doublets non liants :
\begin{itemize}
\item eau $\mathrm{H_2O}$
\item ion cyanure $\mathrm{CN}^-$ ;
\item ion oxalate $\mathrm{C_2O_4}^{2-}$ ;
\item ion thiocyanate $\mathrm{SCN^-}$...
\end{itemize}
Les complexes sont très souvent colorés.

\paragraph{Expérience :} Révélations de quelques cations métalliques de transition (\cite{Buchere2017} p.263).

\paragraph{Diapo :} Révélations de quelques cations métalliques de transition

\paragraph{Indice de coordination : } Nombre de liaison(s) entre l'atome central et les ligands.

\paragraph{Diapo :} Exemple de ligands.

\paragraph{Monodentate :} Un ligand est est monodentate s'il ne se lie au centre métallique que par un seul de ses atomes.
\paragraph{Polydentate :} Au contraire s'il se lie par plusieurs sites de fixation, on dit que le ligand est polydentate.



\subsubsection{Synthèse d'un complexe}

On s'intéresse ici à la synthèse du complexe oxalatofer (III) :
\begin{equation*}
\mathrm{Fe}^{3+} + 3 \mathrm{C_2O_4}^{2-} \rightarrow \mathrm{[Fe(C_2O_4)_3]}^{3-}
\end{equation*}
La constante d'équilibre de cette réaction est appelée constante de formation globale du complexe $\beta$ telle que
\begin{equation*}
\beta = \frac{[\mathrm{[Fe(C_2O_4)_3]}^{3-}](c^0)^3}{[\mathrm{Fe}^{3+}][\mathrm{C_2O_4}^2-]^3},
\end{equation*}
où $c^0 = \unit{1}{\mole\per\liter}$

\paragraph{Expérience :} Synthèse du complexe $\mathrm{K_3[Fe(C_2O_4)_3],3H_2O}$

\subsection{Complexes bioinorganiques}

\subsubsection{Transport de l'oxygène}

\paragraph{Diapo : }

\subsubsection{Un complexe en chimiothérapie}

\subsection{Conclusion}