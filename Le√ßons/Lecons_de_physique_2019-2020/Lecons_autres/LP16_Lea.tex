\section{LP16 Microscopie optique (Léa)}

\paragraph{Bibliographie :}
\begin{itemize}
\item 
\end{itemize}

\paragraph{Niveau : L3} 

\paragraph{Pré-requis :}
\begin{itemize}
\item Optique géométrique
\item Optique ondulatoire (diffraction)
\end{itemize}

\subsection{Introduction}

Former une image agrandie d'un objet trop petit pour être vu (bactérie, objet transparent, etc.). 
Lumière visible seulement

\subsection{Principe du microscope moderne}

\subsubsection{Microscope simplifié}

Constitué de deux optique successives : objectif et oculaire et deux diaphragme

\paragraph{Exp : } montage du microscope simplifié pas à pas : d'abord image du F par une lentille de 33 cm pour symboliser l'oeil, puis objectif pour montrer qu'on fait une image agrandie, puis finalement ajout de l'oculaire par autocolimation.
Il faut alors mettre l'écran dans le plas focal image de la lentille.
Rôle des différent éléments :
\begin{itemize}
\item Objectif : agrandir l'image ;
\item oculaire : agit comme une image.
\end{itemize}
Ajout des diaphragmes de champ et d'ouverture :
\begin{itemize}
\item Do : permet de contrôler le flux lumineux ;
\item Dc : permet de contrôler le champ visible de l'objet sur l'écran. 
\end{itemize}

Le microscope est caractérisé par le grossissement commercial.

\subsubsection{Grossissement commercial du microscope}

\begin{equation}
G_\mathrm{c, micro} = \frac{\theta '}{\theta}
\end{equation}
où $\theta$ est l'angle sous lequel est vu l'objet au puctum proximum ($d_m=\unit{25}{cm}$).
On peut exprimer ce grossissement en fonction des caractéristique des lentilles :
\begin{equation}
G_\mathrm{c, micro} = |\gamma_\mathrm{obj}G_\mathrm{c, oc}|
\end{equation}

Est-ce qu'on peut agrandir une image autant qu'on le souhaite ?

\subsubsection{Limites du microscope}

\begin{enumerate}
\item Pouvoir de résolution : lié aux inhomogénéité à l'échelle de la longueur d'onde. On s'intéresse à l'ouverture numérique : $\omega_0 = n\sin(u)$
\paragraph{Diapo : } Pouvoir de résolution de l'objectif
Pour quantifier la limite de résolution :
\begin{equation}
(AB)_\mathrm{min} = 0,61\frac{\lambda_0}{n\sin(u)}
\end{equation}
Odg : $\lambda_0 = \unit{400}{nm}$, $n=1,5$, $\omega_0=1,25$.
\item Aberration : géométriques liées à la forme de la lentille, chromatiques liées au caractère dispersif du verre des lentilles.
\end{enumerate}

\subsection{Développement récents en microscopie}

\subsubsection{Microscopie à contraste de phase}

Certains objets sont transparent et le contraste est nul, mais il y a une variation de longueur optique : 
\begin{equation}
s(A') = s_0 \rightarrow s(A') = |s_0|^2
\end{equation}
\begin{equation}
s(A') = s_0 e^{i\phi} \rightarrow s(B') = |s_0|^2
\end{equation}
On n'observe donc pas de variation d'intensité.

En ajoutant une lame quart d'onde au centre du cercle oculaire, on affecte seulement les faisceaux non diffracté par l'inhomogénéité en $B'$
\begin{equation}
s(B') = s_0e^{i\pi/2} + s_0(e^{i\phi}-1)
\end{equation}
Après calcul et en supposant le déphasage faible, on obtient 
\begin{equation}
|s_0^2||1-2\phi|
\end{equation}

\subsubsection{Microscopie de fluorescence confocale}

Exploite la réponse de molécule fluorescente qui, à une longueur d'onde d'excitation, émettent une longueur d'onde d'émission.
Schéma du principe, etc.

\paragraph{Vidéo :} Microscopie confocale

\subsection{Conclusion}