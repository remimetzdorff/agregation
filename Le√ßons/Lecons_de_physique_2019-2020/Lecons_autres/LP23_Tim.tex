\section{LP23 Mécanismes de conduction dans les solides (Timothé)}

\niveau L3

\prerequis
\begin{itemize}
\item Template.
\end{itemize}

\objectif Template

\footnotesize{\bibliography{biblio}}
\bibentry{Template}

\subsection{Introduction}

Solide : état de la matière auquel on est tous les jours confronté.
Leurs propriétés sont très variées.
La physique des solides cherche à comprendre ces propriétés.
Avant le XXe, les propriétés étaient essentiellement macroscopiques.
Certaines découvertes ont beaucoup fait avancer ce domaine :
\begin{itemize}
\item 1897 : électron ;
\item 1911 : noyau ;
\item 1912 : diffraction des rayon X par les cristaux.
\end{itemize}

On s'intéresse particulièrement à la conductivité des métaux ici.
La conductivité des solides s'étale sur plusieurs ordres de grandeur.
\begin{slide}
\textbf{Conductivité électrique dans les solides.}
Tableau avec plusieurs valeurs pour des solides différents depuis des bons isolants jusqu'à des bons conducteurs.
\end{slide}

\subsection{Modèle de l'électron libre}

\subsubsection{Approche classique : le modèle de Drude (5'30)}

On veut trouver l'équivalent de la loi d'Ohm (macroscopique) au niveau microscopique.
On applique la théorie cinétique des gaz à la conduction dans les métaux.
Les électrons dans le métal sont considéré comme un gaz parfait.
\begin{slide}
\textbf{Hypothèse du modèle de Drude}
\begin{itemize}
\item les ions du réseau cristallin sont supposés fixes ;
\item les électrons de conduction forment un ensemble de particules libres et indépendantes ;
\item probabilité de collision en $1/\tau$;
\item équilibre thermique établi par les collision.
\end{itemize}
\end{slide}

Schéma d'une section cylindrique de métal, orientation des surfaces.
\begin{equation}
\d N = n\int<\overrightarrow{v}>.\overrightarrow{\d S} \d t
\end{equation}
\begin{equation}
\d Q = q\d N
\end{equation}
d'où
\begin{equation}
\overrightarrow{j} = \frac{\d Q}{\d t} \overrightarrow{u} = qn<\overrightarrow{v}>.
\end{equation}

Si la vitesse moyenne est nulle, il n'y a pas de courant.
Si on applique un champ électrique non  nul, on applique le PFD à un seul électron et on trouve
\begin{equation}
\overrightarrow{j} = \frac{n q^2}{m} \tau \overrightarrow{E}
\end{equation}
(Pas de force de frottement ???)
C'est la loi d'Ohm locale.
On peut poser la conductivité $\sigma$ comme 
\begin{equation}
\sigma = \frac{n q^2}{m} \tau.
\end{equation}

On veut vérifier que cette loi est cohérente avec les observations expérimentales.
On vérifie la loi de Mathiessen (1864) :
\begin{equation}
\frac{\sigma_0}{\sigma} = 1 + \alpha (T-T_0).
\end{equation}

\begin{experience}
\textbf{Mesure de la résistance d'un long fil de cuivre en fonction de la température.}
\end{experience}

Critique du modèle de Drude basée sur l'évaluation des valeurs de libre parcours moyen (de l'ordre de l'Anström) et des temps entre collision, incohérent avec les observations expérimentales.

\subsubsection{Approche quantique : le modèle de Sommerfeld (1926) (28'30)}

\begin{slide}
\textbf{Description quantique du solide.}
Hamiltonien du système.
\end{slide}
On veut résoudre l'équation de Schrödinger
\begin{equation}
H\Psi = E\Psi
\end{equation}
\onfait plusieurs approximations :
\begin{itemize}
\item Born Oppenheimer : électrons fixes ;
\item Hartree-Fock : champ moyen donc potentiel constant.
\end{itemize}
Résolution de l'équation.
Energie de Fermi...

\newpage