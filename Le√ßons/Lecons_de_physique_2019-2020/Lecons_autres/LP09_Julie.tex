\section{LP09 Modèle de l'écoulement parfait}

\niveau CPGE

\prerequis
\begin{itemize}
\item Viscosité, nombre de Reynolds ;
\item Navier Stokes ;
\item cinématique des fluides.
\end{itemize}

\objectif

\footnotesize{\bibliography{biblio}}
\bibentry{}

\subsection{Introduction}

Equation de NS compliqué car non linéaire : dans l'approximation de l'écoulement parfait, c'est plus simple.

\subsection{Description du modèle de l'écoulement parfait}
\subsubsection{Définition d'un écoulement parfait}

\paragraph{Définition :}
Si on peut négliger tous les phénomènes diffusifs (viscosité mais aussi transfert thermique), l'écoulement est parfait.
L'évolution d'une particule de fluide et alors isentropique (adiabatique et réversible).
Il faut faire la différence entre écoulement turbulent et écoulement parfait qui sont deux écoulement à grand nombre de Reynolds.

\subsubsection{Une équation de la dynamique régissant l'écoulement parfait, l'équation d'Euler (4')}

On applique le PFD à une particule de fluide de masse $\d m = \rho \d \tau$, dans un référentiel galiléen où les seules forces sont les forces de pression $-\grad P \d \tau$.
On obtient l'équation d'Euler
\begin{equation}
\rho \left(\frac{\partial \overrightarrow{v}}{\partial t} + (\overrightarrow{v}\grad) \overrightarrow{v} \right) = -\grad P + \rho \overrightarrow{g}
\end{equation}
En changeant $t$ en $-t$ on voit que l'équation est réversible ce qui est cohérent avec l'absence de terme de viscosité.

\subsection{Forces du modèle}
\subsubsection{Equation de Bernoulli (9')}

On considère un écoulement parfait, stationnaire et incompressible d'un fluide homogène.
On utilise $(\overrightarrow{v\grad)\overrightarrow{v}} = \grad\frac{v^2}{2}+(\rot\overrightarrow{v}\wedge\overrightarrow{v})$.

Dans le cas d'un écoulement irrotationnel, on obtient Bernoulli, sinon, il faut se placer sur une ligne de courant.
Bernoulli est une équation de conservation de l'énergie.

\subsubsection{Application : le tube de Pitot 16'}

Schéma.

Calcul de l'écart de pression pour le tube de Pitot.
On obtient
\begin{equation}
v_\infty = \sqrt{\frac{2\Delta P}{\rho}}
\end{equation}

AN avec $\Delta P = \unit{0,26}{bar}$ et $\rho = \unit{1,3}{kg.m^{-3}}$ et on trouve $v_\infty = \unit{200}{m.s^{-2}}$.

\subsection{Lien entre écoulement parfait et réel (25')}

Paradoxe de d'Alembert, ou l'absence de trainée dans un écoulement parfait autour d'une aile d'avion.
Le théorème de Kelvin dit qu'un écoulement irrotationnel ne peut devenir rotationnel.

Pour faire le lien entre les écoulements parfaits et réels, on introduit la notion de couche limite.

\subsection{Conclusion (39')}

\subsection{Question}

\begin{itemize}
\item A propos de l'équation de Bernoulli, est il possible de l'appliquer sur un autre contour ?
On a utilisé $\overrightarrow{v}\wedge\overrightarrow{\d l} = 0$, on peut ausi utiliser $\rot\overrightarrow{v} \wedge \overrightarrow{\d l}$.
\item Peut-on utiliser un tube de Pitot pour mesurer la vitesse d'un vélo ?
On fait l'application numérique et on trouve $\Delta h = \unit{1}{\milli\meter}$ pour $v=\unit{5}{m.s^{-1}}$ avec de l'eau, ce qui est très faible.
Le tube de Pitot est plutôt utilisé pour mesurer des vitesses élevées.
\item Pour le tube de Pitot, quelle est la ligne de courant sur laquelle on applique Bernoulli ?
\item Le tube de Pitot doit être parfaitement aligné sur l'écoulement, y a-t-il une façon de s'affranchir de ce problème ?
Un vrai tube de Pitot a plusieurs trous tout autur, connectés, qui permettent de moyenner ces effets.
\item Y-a-t-il un lien entre la circulation et la vorticité ?
La circulation sur un contour fermé est égal au flux de la vorticité.
\item 
\end{itemize}


\begin{experience}
\textbf{Mesure de la vitesse du son dans l'air}
\end{experience}
\begin{slide}
Stylé
\end{slide}
\begin{transition}
Pompélopie
\end{transition}
\begin{remarque}
Pompélopie
\end{remarque}
\note{Trop de la bombe}
