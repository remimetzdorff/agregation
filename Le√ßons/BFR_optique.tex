\documentclass[12pt,a4paper]{article}
\usepackage[utf8]{inputenc}
\usepackage[french]{babel}
\usepackage[T1]{fontenc}
\usepackage{amsmath}
\usepackage{amsfonts}
\usepackage{amssymb}
\usepackage{graphicx}
\usepackage[left=2cm,right=2cm,top=2cm,bottom=2cm]{geometry}
\author{Rémi Metzdorff}

\begin{document}
On donne le nom général d'onde à un phénomène physique décrit par une fonction scalaire ou vectorielle dépendant à la fois de l'espace et du temps. Dans les cas les plus simples =, cette fonction est solution d'une équation aux dérivées partielles, dite fonction d'onde.

Jolie démonstration des lois de Snell-Descartes d'après le principe de Fermat p100 et après.
\end{document}