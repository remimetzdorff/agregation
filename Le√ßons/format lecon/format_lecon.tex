%%%%%%%%%%%%%%%%%%%%%%%%%%%%%%%%%%%%%%%%%%%%%%%%%%%%%%%%%%%%%
%% HEADER
%%%%%%%%%%%%%%%%%%%%%%%%%%%%%%%%%%%%%%%%%%%%%%%%%%%%%%%%%%%%%
\documentclass[a4paper,10pt]{article}
% Alternative Options:
%	Paper Size: a4paper / a5paper / b5paper / letterpaper / legalpaper / executivepaper
% Duplex: oneside / twoside
% Base Font Size: 10pt / 11pt / 12pt

\voffset = 0pt
\topmargin = 0pt
\textheight = 650pt
\headheight = 0pt
\headsep = 0pt
\oddsidemargin = 5pt
\textwidth = 445pt

%% Language %%%%%%%%%%%%%%%%%%%%%%%%%%%%%%%%%%%%%%%%%%%%%%%%%
\usepackage[greek,french]{babel} %francais, polish, spanish, ...
\usepackage[T1]{fontenc}
\usepackage[UTF8]{inputenc}
\usepackage{upgreek} % Permet d'avoir des lettres grecs non italiques
\usepackage{lmodern} %Type1-font for non-english texts and characters
\usepackage{sectsty}% Pour changer la taille des chapitres
\usepackage{enumitem}
\usepackage{siunitx} % Pour avoir des unité propres
\usepackage{ stmaryrd } % fancy arrow


%\sectiontitlefont{\Huge}
\sectionfont{\fontsize{11}{15}\selectfont\color{black}\bfseries}
\bibliographystyle{abbrv}
%\bibliographystyle{alpha}

\usepackage{bibentry}


%% Packages for Graphics & Figures %%%%%%%%%%%%%%%%%%%%%%%%%%
\usepackage{graphicx} %%For loading graphic files
%\usepackage{pst-all} %%PSTricks - not useable with pdfLaTeX
\usepackage{subfig}
%Pour la page de garde
\usepackage{tabularx} % Permet d'utiliser l'environnement tabularx
\usepackage{calc} % Pour pouvoir donner des formules dans les définitions de longueur
\usepackage{epstopdf} % à combiner avec la commande pdflatex -shell-escape
\usepackage[export]{adjustbox}

%% Table Packages %%%%%%%%%%%%%%%%%%%%%%%%%%%%%%%%%%%%%%%%%%%%
\usepackage{array,multirow,makecell}



%% Math Packages %%%%%%%%%%%%%%%%%%%%%%%%%%%%%%%%%%%%%%%%%%%%
\usepackage{amsmath}
\usepackage{amsthm}
\usepackage{amsfonts}
\usepackage{amssymb}

%% Line Spacing %%%%%%%%%%%%%%%%%%%%%%%%%%%%%%%%%%%%%%%%%%%%%
%\usepackage{setspace}
%\singlespacing        %% 1-spacing (default)
%\onehalfspacing       %% 1,5-spacing
%\doublespacing        %% 2-spacing


%% Other Packages %%%%%%%%%%%%%%%%%%%%%%%%%%%%%%%%%%%%%%%%%%%
%\usepackage{a4wide} %%Smaller margins = more text per page.
%\usepackage{fancyhdr} %%Fancy headings
%\usepackage{longtable} %%For tables, that exceed one page
\usepackage{changepage}   % for the adjustwidth environment


%%%%%%%%%%%%%%%%%%%%%%%%%%%%%%%%%%%%%%%%%%%%%%%%%%%%%%%%%%%%%
%% Options / Modifications
%%%%%%%%%%%%%%%%%%%%%%%%%%%%%%%%%%%%%%%%%%%%%%%%%%%%%%%%%%%%%

%\input{options} %You need a file 'options.tex' for this
%% ==> TeXnicCenter supplies some possible option files
%% ==> with its templates (File | New from Template...).



%%%%%%%%%%%%%%%%%%%%%%%%%%%%%%%%%%%%%%%%%%%%%%%%%%%%%%%%%%%%
%% Commmandes
%%%%%%%%%%%%%%%%%%%%%%%%%%%%%%%%%%%%%%%%%%%%%%%%%%%%%%%%%%%%
%\usepackage[colorinlistoftodos,prependcaption,textsize=tiny,textwidth=3.0cm]{todonotes}
\usepackage[colorinlistoftodos,prependcaption,textsize=footnotesize,textwidth=3.0cm]{todonotes}

%\usepackage[colorinlistoftodos]{todonotes}
\newcommand{\note}[1]{\todo[color=red!15]{#1}}
%\newcommand{\change}[1]{\todo[color=blue!25]{#1}}
%\newcommand{\todos}[1]{\todo[inline,color=green!25]{#1}}
%\newcommand{\todoss}[1]{\todo[color=green!40]{#1}}
%\newcommand{\improvement}[1]{\todo[color=Plum!25]{#1}}


\newcommand{\Real}{\textrm{Re}}
\newcommand{\Imag}{\textrm{Im}}
\newcommand{\bra}[1]{\ensuremath{\left\langle#1\right|}}
\newcommand{\ket}[1]{\ensuremath{\left|#1\right\rangle}}
\newcommand{\braket}[2]{\left\langle #1\middle|#2\right\rangle}
\newcommand{\tg}[1]{\foreignlanguage{greek}{#1}}
\newcommand{\bibo}{BiB$_3$O$_6$ }
\newcommand{\knbo}{KnBO$_3$ }



\usepackage{xcolor}
\usepackage[framemethod=tikz]{mdframed}
\usepackage{chngcntr}



%%%%%%%%%%%%%%%%%%%%%%%%%%%%%%%%%%%%%%%%%%%%%%%%%%%%%%%%%%%%%%%%%%%%%%%%
% Définition des cadre comme dans le cours de Jéremy
%%%%%%%%%%%%%%%%%%%%%%%%%%%%%%%%%%%%%
% Entete niveau/messsage/prérequis
\newcommand{\niveau}{\colorbox[rgb]{0 0.6 0}{\textbf{\color{white} Niveau}} }
\newcommand{\messages}{\colorbox[rgb]{0 0.6 0}{\textbf{\color{white} Message}} }
\newcommand{\prerequis}{\colorbox[rgb]{0 0.6 0}{\textbf{\color{white} Pré-requis }} }




%%%%%%%%%%%%%%%%%%%%%%%%%%%%%%%%%%%%%
% Transition

\definecolor{brun}{rgb}{0.87,0.72,0.53}
\definecolor{brunfonce}{rgb}{0.8,0.4,0.2}
\definecolor{chamois}{rgb}{1.0,0.90,0.70}
\definecolor{bleu_f}{rgb}{0.1,0.1,0.53}
\definecolor{bleu_c}{rgb}{0.8,0.8,0.95}
\definecolor{red_f}{rgb}{0.53,0.1,0.1}
\definecolor{red_c}{rgb}{0.95,0.8,0.8}
\definecolor{green_f}{rgb}{0.1,0.53,0.1}
\definecolor{green_c}{rgb}{0.8,0.95,0.8}

\mdfdefinestyle{s_trans}{%
	linecolor=brunfonce!,
	outerlinewidth=3pt,%
	frametitlerule=false,
	topline=false,
	bottomline=false,
	rightline=false,
	backgroundcolor=chamois,
	innertopmargin=\topskip,
	roundcorner=0pt
}
\newmdenv[style=s_trans]{transition2_env}
\newenvironment{transition}
{%\stepcounter{exa}%
	\addcontentsline{ldf}{figure}{0}%
	\begin{transition2_env}[]{\noindent \textbf{Transition : \\}}}
	{\end{transition2_env}}

%%%%%%%%%%%%%%%%%%%%%%%%%%%%%%%%%%%%%
% Expérience


\mdfdefinestyle{s_experience}{%
	linecolor=bleu_f!,
	outerlinewidth=3pt,%
	frametitlerule=false,
	topline=false,
	bottomline=false,
	rightline=false,
	backgroundcolor=bleu_c,
	innertopmargin=\topskip,
	roundcorner=0pt
}
\newmdenv[style=s_experience]{experience_env}
\newenvironment{experience}
{%\stepcounter{exa}%
	\addcontentsline{ldf}{figure}{0}%
	\begin{experience_env}[]{\noindent \textbf{Expérience : \\}}}
%	\begin{experience_env}[]{\noindent\colorbox[rgb]{0.1 0.1 0.53}{\textbf{\color{white} Expérience : }}\\}}
	{\end{experience_env}}

%%%%%%%%%%%%%%%%%%%%%%%%%%%%%%%%%%%%%
% Slide



\mdfdefinestyle{s_slide}{%
	linecolor=green_f!,
	outerlinewidth=3pt,%
	frametitlerule=false,
	topline=false,
	bottomline=false,
	rightline=true,
	backgroundcolor=white,
	innertopmargin=\topskip,
	roundcorner=0pt
}
\newmdenv[style=s_slide]{slide_env}

\newenvironment{slide}
	{%\stepcounter{exa}%
%		\newenvironment{myenv}{\begin{adjustwidth}{2cm}{}}{\end{adjustwidth}}
		\addcontentsline{ldf}{figure}{0}%
		\begin{slide_env}[]{\noindent \textbf{Slide : \\}}}
		{\end{slide_env}
	}

%%%%%%%%%%%%%%%%%%%%%%%%%%%%%%%%%%%%%
% Remarques


\mdfdefinestyle{s_remarque}{%
	linecolor=red_f!,
	outerlinewidth=3pt,%
	frametitlerule=false,
	topline=false,
	bottomline=false,
	rightline=false,
	backgroundcolor=red_c,
	innertopmargin=\topskip,
	roundcorner=0pt
}
\newmdenv[style=s_remarque]{remarque_env}
\newenvironment{remarque}
{%\stepcounter{exa}%
	\addcontentsline{ldf}{figure}{0}%
	\begin{remarque_env}[]{\noindent \textbf{Remarque : \\}}}
	{\end{remarque_env}}

