\section{LP08 Phénomènes de transport}

\begin{header}
\begin{tabular}{p{0.4\textwidth} l}
\niveau & \prerequis \\
CPGE & \textbullet{} Hydrodynamique : notion de viscosité \\
     & \textbullet{} Thermodynamique à l'équilibre \\
     & \textbullet{} Électromagnétisme : loi d'ohm locale, de Joule \\
\end{tabular}

\noindent
\objectif
Template
\end{header}

{
\subsubsection*{Bibliographie}
\footnotesize{}
\begin{itemize}
\item \cite{Olivier1998}
\item \cite{Olivier2000}
\item \cite{Sanz2016}
\item \cite{Guyon2001}
\item \cite{Augier2014}
\item \cite{Taillet2018}
\item \cite{Diu2008}
\end{itemize}
}

\begin{remarque}
En l'état, la leçon est probablement un peu catalogue, ce qui va à l'encontre des retours de jury.
On parle ici majoritairement de diffusion, mais ce n'est pas une leçon sur la diffusion.
Le programme de CPGE ne permet pat d'aller beaucoup plus loin.
\end{remarque}

\begin{experience}
On peut réaliser des expériences qualitatives :
\begin{itemize}
\item diffusion d'une goutte d'encre sur un buvard humide ;
\item simulations de marche aléatoire pour la modélisation microscopique de la diffusion de particules ;
\end{itemize}
et quantitatives :
\begin{itemize}
\item mesure du coefficient de diffusion du glycérol dans l'eau (TP Fluides - Capilarité ;
\item conductivité thermique d'un barreau de cuivre (TP Métaux).
\end{itemize}
\end{experience}

\subsection*{Introduction}

Jusqu'à présent en thermodynamique, on s'est intéressé à des systèmes à l'équilibre auxquels on pouvait associer des grandeurs intensives telles que la température ou la pression.
Le détail des transformations est souvent difficile à étudier car elles peuvent passer par à des systèmes hors équilibres, dans lesquels il existe des transport de plusieurs quantités.
L'étude de ces phénomènes de transport ne ce restreint pas à la thermodynamique :
\begin{itemize}
\item mécanique des fluides : transport de quantité de mouvement ;
\item thermodynamique : transport d'énergie et de particules ;
\item électromagnétisme : transport de charge.
\end{itemize}

L'idée générale est décrite dans \cite{Diu2008} p461 ou l'on étudie le transport de quantités conservées.

\subsection{Généralités}

On va principalement étudier les phénomènes de transport dans le cadre de la thermodynamique tout en établissant les similitudes à d'autres domaines.
Il faut étudier les phénomènes de transport si le système n'est pas à l'équilibre.
Cela nécessite de poser quelques hypothèses (linéarité et ETL) que nous allons tout d'abord énoncer.

\begin{transition}
Quels sont les phénomènes de transports que nous pouvons rencontrer ?
\end{transition}

\subsubsection{Différents phénomènes}

Décrit dans \cite{Olivier2000} p342.
\begin{slide}
\textbf{Différents phénomènes de transport.}
Donner quelque exemples \cite{Sanz2016} p89 :
\begin{itemize}
\item diffusion : parfum, thermique, viscosité, conduction électrique ;
\item convection : brassage de l'air, convection forcée ou naturelle ;
\item rayonnement : prendre l'exemple du vase Dewar.
\end{itemize}
\end{slide}

En général il faut considérer tous les phénomènes de transport.
Cependant, si l'un de ces phénomènes est majoritaire, on peut négliger les autres.
Prendre le cas de l'énergie thermique et donner l'expression du vecteur densité de courant surfacique et son unité \cite{Sanz2016} p122.

\begin{remarque}
À ce stade, on ne précise pas s'il s'agit de conduction, de convection ou de rayonnement.
Voir \cite{Sanz2016} p148 exercice 3{.}9 pour les modèles sur les différents phénomènes.
\end{remarque} 

\begin{transition}
Comme les systèmes étudiés ne sont pas à l'équilibre thermodynamique, il faut faire quelques hypothèses pour étudier ces phénomènes avec les outils de la thermodynamique ou de la mécanique des fluides.
\end{transition}

\subsubsection{Équilibre thermodynamique local}

Reprendre les hypothèses sur l'étude des phénomènes de transport énoncées dans \cite{Diu2008} p465 en commençant par l'équilibre thermodynamique local \cite{Diu2008} p462, définition p464 et \cite{Olivier2000} p344 ;

Il faut dégager les différentes échelles de temps et d'espace.
\begin{slide}
\textbf{Différentes échelles spatiales.}
Faire le parallèle avec la mécanique des fluides et la notion de particules de fluide.
\end{slide}

Pour les échelles temporelles, voir \cite{Diu2008} p463.

Faire le bilan énergétique local à 1D en présence de sources de \cite{Sanz2016} et donner l'équivalent 3D.

\begin{transition}
Pour simplifier l'analyse, on se restreint à des situations proches de l'équilibre ce qui mène à la linéarisation des équations. 
\end{transition}

\subsubsection{Linéarité}

Approximation linéaire \cite{Diu2008} p476.

Donner la loi de Fourier ainsi que ses limites \cite{Sanz2016} p130-131 et \cite{Olivier2000} p348.

Aboutir à l'équation de diffusion avec source \cite{Sanz2016} p132.

\begin{transition}
Les deux équations obtenues permettent d'aboutir à une équation rencontrée dans de nombreux domaines de la physique.
\end{transition}

\subsection{Équation de diffusion}

\subsubsection{Analogies}

Donner l'expression de l'équation de diffusion thermique sans source et mettre en évidence
\begin{itemize}
\item la grandeur intensive : $T$ ;
\item la grandeur extensive conservée : $E$ ;
\item l'origine de la diffusion : $\grad$ qui donne lieu à un flux de la quantité conservée.
\end{itemize}
\cite{Diu2008} p479 sur les conditions initiales nécessaires à la résolution de l'équation.

\begin{slide}
\textbf{Équation de diffusion.}
Bien relire \cite{Olivier2000} p361-363 et marquer les limites de l'analogie.
Pour la viscosité, on se place dans le cas de l'écoulement de Couette plan.

\noindent
Donner un exemple pour chaque situation :
\begin{itemize}
\item barreau de cuivre ;
\item fibre à gradient d'indice, \cite{Olivier1998} p92 ;
\item mise en évidence du caractère diffusif de la viscosité \cite{Olivier2000} p444 ;
\item conduction électrique dans un conducteur.
\end{itemize}
\end{slide}

\begin{slide}
\textbf{Ordres de grandeur.}
Continuer sur les limites de l'analogie en montrant les différences qui peuvent être énormes entre les ordres de grandeur.
En revanche, on remarque la proximité des coefficients de diffusion dans les gaz : c'est normal car le même processus microscopique est à l'œuvre : \cite{Olivier2000} p428.
\end{slide}

Donner la loi d'échelle qui lie temps caractéristique, taille caractéristique et coefficient de diffusion comme \cite{Olivier2000} p354 et \cite{Diu2008} p480.

\begin{remarque}
Davantage de détails sur la loi de Fick dans \cite{Diu2008} p477, sur la loi d'Ohm p483 et sur le transport de quantité de mouvement p514.

\noindent
Être conscient que dans les deux cas, le gradient à l'origine du transport porte en toute rigueur sur le potentiel chimique.
De manière générale, l'inhomogénéité d'une grandeur se traduit par un flux de la grandeur thermodynamique conjuguée \cite{Diu2008} p524.
\end{remarque}

\begin{transition}
Cette équation générale décrit un phénomène irréversible : on le voit directement car elle fait intervenir une dérivée première par rapport au temps.
Un interprétation microscopique permet de comprendre pourquoi.
\end{transition}

\subsubsection{Caractère irréversible}

Remarque générale de \cite{Olivier2000} et lire \cite{Diu2008} p480.

Bien poser le cadre : étude d'un gaz monoatomique sans mouvement macroscopique ce qui permet de considérer uniquement la diffusion.

\begin{slide}
\textbf{Diffusion et marche aléatoire.}
\end{slide}

\begin{remarque}
Modèle microscopique de la diffusion thermique dans un gaz monoatomique dans \cite{Olivier2000} p351 et de la viscosité p426.

\noindent
On peut montrer le caractère irréversible de la diffusion thermique en faisant un bilan d'entropie \cite{Sanz2016} p134.
\end{remarque}

\begin{remarque}
Attention à l'équation de Schrödinger : sa forme est celle d'une équation de diffusion mais à coefficient complexe.
Elle est réversible par un changement de $t$ en $-t$ et de $\psi$ en $\psi^\dagger$.
Voir les entrées Schrödinger, rotation de Wick et réversible de \cite{Taillet2018}.
\end{remarque}

\begin{transition}
On peut exploiter les propriétés du phénomène de diffusion pour mesurer la conductivité d'un matériau.
\end{transition}

\subsubsection{Mesure de la conductivité thermique du cuivre}

\begin{experience}
\textbf{Mesure de la conductivité thermique d'un barreau de cuivre.}
Valeurs tabulées : 
\begin{itemize}
\item $\lambda_\mathrm{Cu}=\unit{390\times 10^{-6}}{\watt\cdot\meter^{-1}\cdot\kelvin^{-1}}$ ;
\item $\rho_\mathrm{Cu} = \unit{8{,}93\times10^3}{\kilo\gram\cdot\meter^{-3}}$ ;
\item $c_\mathrm{Cu} = \unit{0{,}382}{\kilo\joule\cdot\kilo\gram^{-1}\cdot\kelvin^{-1}}$ ;
\item $D_\mathrm{Cu} = \unit{117\times10^{-6}}{\meter^2\cdot\second^{-1}}$.  
\end{itemize}
On en déduit un temps caractéristique pour l'établissement du régime permanent de l'ordre de \unit{85}{\second} (pour un barreau de \unit{10}{\centi\meter}).
\end{experience}

Faire un schéma et résoudre l'équation de Laplace.
Les conditions aux limites $T(x=0)=T_0$ et ${\cal{P}}=S\lambda\frac{\partial T}{\partial x}$ donnent :
\begin{equation}
T(x) = \frac{\cal{P}}{S\lambda} x + T_0.
\end{equation}
Comparer la valeur mesurée avec les valeurs tabulées en justifiant l'écart avec le fait que l'intégrale de la puissance électrique fournie à la résistance n'est pas transmise au barreau.
On doit normalement obtenir une conductivité plus grande que prévue.

\begin{remarque}
Avoir en tête la loi de Wiedemann-Franz qui lie la conductivité thermique, électrique et la température (voir \href{http://ressources.agreg.phys.ens.fr/ressources/}{TP Métaux}).
\end{remarque}

\begin{transition}
On l'a vu la diffusion n'est pas le seul moyen de transport.
Comment peut-on les comparer ?
\end{transition}

\subsection{Comparaison entre les différents phénomènes}

\subsubsection{Résistance thermique}

Établir l'expression de la résistance thermique de conduction \cite{Sanz2016} p140, donner la loi de Newton et la résistance associée au rayonnement p148.

\begin{slide}
\textbf{Résistance thermique en régime permanent.}
La résistance associée à la convection est donnée pour un \og jour de grand vent \fg{}.
Donner l'interprétation de cette loi : continuité microscopique sur une couche limite thermique de faible épaisseur qui apparait comme une discontinuité macroscopique \cite{Diu2008} p498.
\end{slide}

\begin{remarque}
Remarques sur le transport convectif de chaleur \cite{Guyon2001} p564.
\end{remarque}

\begin{transition}
Pour expliquer le comportement d'un système, il est souvent suffisant de tenir compte uniquement du phénomène majoritaire.
Comme en mécanique des fluides, on peut distinguer différents régime grâce à des nombres addimensionnés.
\end{transition}

\subsubsection{Phénomène prépondérant}

Exprimer le rapport entre convection en $U \times \rho U$ et diffusion en $\eta \grad U$ et remarquer qu'il s'agit du nombre de Reynolds et revenir sur le rapport entre les termes de l'équation de Navier-Stokes.

Mentionner les nombres de Péclet et Prandtl \cite{Guyon2001} p102-103.
Application numérique pour une tasse et du sucre :
\begin{equation}
Pe = \frac{UL}{D} = \frac{10^{-2}\times 5.10^{-2}}{0,52.10^{-9}} \approx 10^4.
\end{equation}
La diffusion est remarquablement inefficace !

\subsection*{Conclusion}

\begin{slide}
\textbf{Obtention de très basses températures  : un cryostat à dilution.}
Récapituler les différents transferts thermiques qu'il est nécessaire contrôler pour obtenir des température aussi basses. 
\end{slide}

Ouvrir sur l'effet Seebeck, Peltier \cite{Diu2008} p518 et suivantes.

\newpage