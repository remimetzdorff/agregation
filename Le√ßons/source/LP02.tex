\section{LP02 Lois de conservation en dynamique}

\begin{header}
\begin{tabular}{p{0.4\textwidth} l}
\niveau & \prerequis \\
CPGE & \textbullet{} Dynamique du point \\
     & \textbullet{} PFD, TMC, TEC, TEM \\
\end{tabular}

\noindent
\objectif
Montrer la puissance des lois de conservation et faire apparaitre leur lien avec des symétries du problème. 
\end{header}

{
\subsection*{Bibliographie}
\footnotesize{}
\begin{itemize}
\item \cite{Faroux1996}
\item \cite{Bocquet2002}
\item \cite{Michel2017}
\item \cite{Salamito2016}
\item \cite{Seigne2014}
\item \cite{Landau1969} Chapitre II : en mécanique lagrangienne...
\end{itemize}
}

\begin{remarque}
Reprendre cette leçon avec de la bibliographie de MP...

\noindent
Avoir en tête le  \href{https://fr.wikipedia.org/wiki/Th\%C3\%A9or\%C3\%A8me_de_Noether_(physique)}{théorème de Noether} mais ne pas en parler.
Je ne pense pas que les histoires de symétries doivent être mentionnées dans la leçon en la plaçant au niveau CPGE.

\noindent
Dans le théorème de Noether, l'invariance porte sur le potentiel/le lagrangien.

\noindent
Pour le problème à deux corps, les rapports de jury stipulent que les cas les plus intéressants apparaissent quand les deux corps ont des masses proches : pas sûr de comprendre pourquoi.
Peut-être parce que là, le mouvement d'un des deux corps ne peut être négligé ?
\end{remarque}

\subsection*{Introduction}

\subsection{Conservation de l'impulsion}

\subsubsection{Loi de conservation}

Développer le cas d'un système dans un référentiel galiléen et écrire le PFD pour faire apparaitre les conditions de conservation :
\begin{itemize}
\item système isolé.
\item résultante nulle (mentionner le cas de l'impesanteur : chute libre).
\item projection dans une direction nulle.
\end{itemize}
L'impulsion est constante donc $\overrightarrow{p}$ est une intégrale première du mouvement.

Cela vient de l'homogénéité de l'espace : une invariance par translation dans l'espace est équivalente à une conservation de l'impulsion.
Si le potentiel $V$ est indépendant de $x$ par exemple, pas de force selon $x$ et $p_x$ est conservé.

\begin{remarque}
Dans le cas d'une collision, même si le système n'est pas isolé, on a conservation de l'impulsion car la durée de la collision est très faible.
La collision est localisée temporellement et spatialement !
Voir \cite{Seigne2014} p192.
Garder en tête qu'une quantité est négligeable par rapport à une autre : ici $F_\mathrm{ext}\tau \ll p_\mathrm{avant}$.
\end{remarque}

\begin{transition}
Voyons un cas concret de la conservation de l'impulsion.
\end{transition}

\subsubsection{Lancer au curling}

Calculer la vitesse de recul d'un patineur à l'arrêt (analogue à \cite{Bocquet2002} p401) avec :
\begin{itemize}
\item masse de la pierre $m=\unit{20}{\kilo\gram}$
\item masse du lanceur $M=\unit{80}{\kilo\gram}$
\item vitesse de la pierre après le lancer $v=\unit{1}{\meter\per\second}$
\end{itemize}
On trouve $V=\unit{-0{,}25}{\meter\per\second}$.

\begin{transition}
Pas besoin de connaitre les détails de l'interaction, seulement les situations initiale et finale.
Mentionner le ralentissement et refroidissement d'un jet d'atomes par absorptions de photons.
\end{transition}

\subsubsection{Problème à deux corps -- Mobile fictif}

Suivre \cite{Faroux1996} p136 ou \cite{Bocquet2002} p129 :
\begin{itemize}
\item référentiel barycentrique ;
\item mobile fictif.
\end{itemize}

\begin{transition}
Le système à deux corps se résume à un problème à force centrale.
Existe-t-il un invariant particulier dans ce cas ?
\end{transition}

\subsection{Conservation du moment cinétique}

\subsubsection{Loi de conservation}

Comme avant mais avec le TMC au lieu du PFD.
\begin{itemize}
\item système isolé ;
\item résultante nulle ;
\item force centrale
\end{itemize}

Dans un espace isotrope, une invariance par rotation dans l'espace est équivalente à une conservation de l'impulsion.

\begin{remarque}
Attention : si le point d'application du TMC est un point mobile par rapport au système, il faut ajouter un terme.
La forme la plus simple n'est valide que par rapport à un point/axe fixe.
\end{remarque}

\begin{transition}
Voyons un cas concret de la conservation du moment cinétique.
\end{transition}

\subsubsection{Patinage artistique}

Faire les calculs pour une patineuse \cite{Perez2014}.
Dire qu'une fois de plus le détails du passage de la situation initiale à la situation finale n'ont pas d'importance.

\begin{transition}
En revenant au problème à deux corps, que nous apprend la conservation du moment cinétique ?
\end{transition}

\subsubsection{Problème à deux corps -- Planéité du mouvement}

Suivre \cite{Michel2017} p456 pour montrer la planéité du mouvement et introduire la constante des aires.
Mentionner la loi de Kepler ?

\begin{transition}
Une dernière loi de conservation permet de simplifier ce problème : elle porte sur l'énergie.
\end{transition}

\subsection{Conservation de l'énergie}

\subsubsection{Loi de conservation}

Suivre \cite{Salamito2016} p624.
Partir de l'expression infinitésimale du théorème de l'énergie cinétique et faire apparaitre le cas où la force dérive d'une énergie potentielle.
Les forces internes interviennent dans ce bilan.
On aboutit à la conservation de l'énergie mécanique (approximation : frottements, etc...)

\begin{remarque}
La conservation de l'énergie mécanique est toujours une approximation mais la conservation de l'énergie (interne) est vraie.
\end{remarque}

Faire rapidement le cas du pendule pour montrer que s'il n'y a qu'un seul degré de liberté, la résolution du pendule simple est rapide.
\begin{experience}
\textbf{Pendule simple.}
\end{experience}

Pour un système indépendant du temps, l'invariance par translation dans le temps est équivalente à une conservation de l'énergie.

\begin{transition}
Cette conservation est particulièrement utile pour étudier la physique des collisions : on ne connait pas l'interaction au moment du choc mais ce n'est pas un soucis.
\end{transition}

\subsubsection{Choc entre deux pierres au curling}

Suivre par exemple \cite{Seigne2014} p372.
Faire le cas à 1D si le temps manque ou en 2D.
Le problème se résout aussi en 3D mais c'est plus lourd en calcul.

\begin{experience}
\textbf{Collision de deux mobiles ???}
\end{experience}

\begin{remarque}Pour la collision 2D, il n'y a pas assez d'équations pour résoudre entièrement le problème : deux équations pour l'impulsion et une pour l'énergie.
Il faut imposer un angle.

\noindent
Collision directe : les vecteurs avant et après la collision sont colinéaires.

\noindent
Pour l'effet Compton et plus généralement en mécanique relativiste, il faut écrire la conservation du quadri-vecteur énergie impulsion du système.
\end{remarque}

\begin{transition}
Qu'en est-il du problème à deux corps ?
\end{transition}

\subsubsection{Problème à deux corps -- Trajectoires}

Suivre \cite{Michel2017} p459.
Parler du potentiel effectif et distinguer les 3 trajectoires possibles.

\cite{Faroux1996} p132 pour la discussion, collision, etc.

\begin{remarque}
Ce type de problème se résout très bien avec l'invariant de Runge-Lenz.
\end{remarque}

\subsection*{Conclusion}

\begin{slide}
\textbf{Pulsar binaire.}
Conservation de l'énergie : rayonnement d'ondes gravitationnelles comme première preuve indirecte de l'existence des OG
\end{slide}

\begin{funfact}
\href{https://www.youtube.com/watch?v=cnGYMe6GBeQ}{When Conservation of Energy FAILS! (Noether's Theorem)}.
La conservation de l'énergie repose sur une invariance par translation dans le temps du lagrangien... qui n'est pas vraie aux échelles cosmiques.
En effet, le lagrangien de l'univers se décompose en trois termes : un pour la matière, un pour le champ électromagnétique et un pour l'espace temps.
Ce dernier n'est pas invariant en raison de l'expansion de l'univers ce qui ne permet pas la d'appliquer la loi de conservation de l'énergie à l'échelle de l'univers.
Voir aussi l'\href{https://www.pourlascience.fr/sd/cosmologie/lunivers-perd-il-de-lenergie-1774.php}{article Pour la science}.

\noindent
Effet Doppler et conservation de l'énergie : à méditer.
\end{funfact}


\subsection*{Exercices}

\begin{itemize}
\item diffusion de Rutherford : \cite{Faroux1996} p142, \cite{Michel2017} p466.
\item effet Compton : \href{http://supernovae.in2p3.fr/~llg/Enseignements/Agregation/Relativite/}{Polycopié de TD (corrigé)} p60 pour le traitement avec les quadrivecteurs et \cite{Seigne2014}, p381 sans les quadrivecteurs.
\end{itemize}

\newpage