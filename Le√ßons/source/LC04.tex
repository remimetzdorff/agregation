\section{LC04 Synthèse inorganique}

\begin{header}
\begin{tabular}{p{0.4\textwidth} l}
\niveau & \prerequis \\
Lycée (STL - SPCL)   & \textbullet{} Constante d'équilibre \\
        & \textbullet{} Dosages par titrage, étalonnage \\
        & \textbullet{} Structure de Lewis \\
        & \textbullet{} Électrolyse \\
\end{tabular}

\noindent
\objectif
Décrire les interactions matière rayonnement avec les résultats de la mécanique quantique.
\end{header}

{
\footnotesize{\bibliography{source/biblio}}
\bibentry{Buchere2017}
\bibentry{Cachau-Hereillat2011}
\bibentry{Fosset2016}
\bibentry{Fosset2014}
\bibentry{BO}
\bibentry{LivreNunSTL}
\bibentry{ElementariumJavel}
\bibentry{Eurochlor}
}

\paragraph{Expériences :}
\begin{itemize}
\item Synthèse de l'eau de Javel par électrolyse de NaCl \cite{Cachau-Hereillat2011} p.337 ;
\item Révélation de quelques cations métalliques de transition \cite{Buchere2017} p.263 ;
\item Synthèse du complexe $\mathrm{K_3[Fe(C_2O_4)_3],3H_2O}$ \cite{Buchere2017} p.291.
\end{itemize}


\subsection{Introduction}

Par synthèse, on sous-entend le procédé permettant d'obtenir une nouvelle espèce chimique par transformation d'un ou plusieurs réactifs. 
Dans cette leçon on s'intéresse aux synthèses inorganiques, i.e. qui n'impliquent pas de modification d'un squelette carboné (qui relève du domaine de la chimie organique).
Historiquement, c'est ce qu'on appelle la chimie minérale, même si ses frontières sont parfois ténues, notamment comme on le verra quand on s'intéresse à des complexes faisant intervenir des ligands organiques.

On s'intéressera tout d'abord à la synthèse de composés simples à travers l'exemple de la synthèse du dichlore, puis on introduira de nouveaux assemblages atomiques avec les complexes dont on verra un exemple de synthèse.

\subsection{Synthèse du dichlore}

\subsubsection{Synthèse de l'eau de Javel en laboratoire}

Un peu d'histoire :
\begin{itemize}
\item $\sim 1785$ : blanchiment au dichlore ;
\item $\mathrm{Cl_2}$ obtenu par oxydation de l'acide chlorhydrique le dioxyde de manganèse $$\mathrm{MnO_2 + 4HCl \rightarrow MnCl_2 + Cl_2 + H_2O}$$;
\end{itemize} 

\begin{slide}
\textbf{Schéma de la manip.}
\end{slide}

On peut synthétiser le dichlore par électrolyse de la saumure.
Sur la cathode on observe la réduction de l'eau :
\begin{equation*}
\mathrm{H_2O + 2e^- \rightarrow H_2 + 2HO^-}
\end{equation*}
et sur l'anode l'oxydation des ions chlorure :
\begin{equation*}
\mathrm{2Cl^- \rightarrow Cl_2 + 2e^-}
\end{equation*}
L'équation bilan de l'électrolyse est donc :
\begin{equation*}
\mathrm{H_2O + 2Cl^- \rightarrow Cl_2 + H_2 + 2HO^-}
\end{equation*}
Sous agitation, on peut ainsi dissoudre le dichlore dans une solution basique qui conduit par dismutation à :
\begin{equation*}
\mathrm{Cl_2 + 2HO^- \rightarrow Cl^- + ClO^- + H_2O} 
\end{equation*}

\begin{experience}
\textbf{Synthèse du dichlore par électrolyse de la saumure.}
\begin{itemize}
\item lancer l'électrolyse dès le début de la leçon ;
\item mettre en évidence la formation de $\mathrm{ClO^-}$ avec l'iodure de potassium + empois d'amidon ;
\begin{equation*}
\mathrm{ClO^- + H_2O + 2I^- \rightarrow I_2 + Cl^- + 2HO^-} 
\end{equation*}
\item comparer à un prélèvement avant l'électrolyse et un prélèvement de la préparation.
\end{itemize}
\end{experience}

\begin{transition}
Ce processus ne permet pas la production de dichlore à grande échelle.
Qu'en est-il des méthodes de production industrielles ?
\end{transition}

\subsubsection{Synthèse industrielle}

Le dichlore est un composé essentiel dans notre monde actuel.
\begin{itemize}
\item actuellement utilisé pour la synthèse de l'acide chlorydrique, du PVC, de fluides frigorigènes, pour le blanchiment de toiles, de papier, comme désinfectant, etc. ;
\item production actuelle : $70$ million de tonnes en 2017.
\end{itemize}

\begin{slide}
\textbf{Synthèse industrielle de l'eau de Javel.}
La synthèse se fait en séparant les deux cellules : il faut assurer le transport des ions sodium pour la neutralité.
Comparaison des différentes méthodes et un mot sur le réacteur ouvert.
\end{slide}

Insister sur :
\begin{itemize}
\item matières premières ;
\item sous produits ;
\item énergie ;
\item catalyseur ;
\item sécurité.
\end{itemize}

\begin{transition}
On a vu que les méthodes de production s'efforcent d'être plus en accord avec les enjeux environnementaux de notre époque.
Un autre exemple qui illustre cette préoccupation envers les problématiques environnementales est celui de la synthèse de l'ammoniac.
\end{transition}

\subsubsection{Vers des synthèses plus vertes}

Production actuelle : plus de 100 millions de tonnes par an, utilisé dans les engrais, les explosifs, les carburants, polymères, etc. consomme entre 1 et 2 \% de la consommation énergétique mondiale.

Sa synthèse repose sur le procédé Haber-Bosch développé au début du $\mathrm{XX^e}$ siècle, par réaction directe de diazote et dihydrogène en présence d'un catalyseur (Fer $\alpha$), à haute température (\unit{450}{\celsius}) et haute pression (\unit{250}{bar}) :
\begin{equation*}
\mathrm{N_{2(g)}} + 3\mathrm{H_{2(g)}} \rightarrow 2\mathrm{NH_{3(g)}}
\end{equation*}
L'idéal serait de parvenir à s'inspirer de la nature où l'on trouve de nombreuses plantes capables de réaliser cette transformation sans avoir besoin d'une telle énergie, par catalyse enzymatique.

La difficulté est de rompre la triple liaison du diazote. 
Pour cela, certains progrès récents proposent l'utilisation de complexes organométalliques.

\begin{transition}
Que sont les complexes et comment les synthétiser.
\end{transition}

\subsection{Les complexes}

\subsubsection{Mise en évidence}

Un complexe est un édifice polyatomique formé d'un centre métallique (souvent un cation d'un métal de transition) autour duquel sont liés (coordonnées ou coordinés) des molécules ou anions appelés ligands.

\begin{slide}
\textbf{Exemple de complexe.}
\end{slide}

L'ion central est un accepteur d'électrons :
\begin{itemize}
\item fer(II), fer(III) ;
\item cuivre(I), cuivre(II) ;
\item cobalt(II)...
\end{itemize}
alors que les ligands sont donneurs d'électrons, ce qui permet de former une ou plusieurs liaison(s) par partage de doublets non liants :
\begin{itemize}
\item eau $\mathrm{H_2O}$
\item ion cyanure $\mathrm{CN}^-$ ;
\item ion oxalate $\mathrm{C_2O_4}^{2-}$ ;
\item ion thiocyanate $\mathrm{SCN^-}$...
\end{itemize}
Les complexes sont très souvent colorés.

\begin{experience}
\textbf{Révélations de quelques cations métalliques de transition.}
(\cite{Buchere2017} p.263)
\end{experience}

\begin{slide}
\textbf{Révélations de quelques cations métalliques de transition}
\end{slide}

L'indice de coordination est le nombre de liaison(s) entre l'atome central et les ligands.

\begin{slide}
\textbf{Exemple de ligands.}
\end{slide}

Pour quantifier le nombre de liaison que peut former un ligand avec le centre métallique, on parle de denticité du ligand :
\begin{itemize}
\item un ligand est est monodentate s'il ne se lie au centre métallique que par un seul de ses atomes ;
\item au contraire s'il se lie par plusieurs sites de fixation, on dit que le ligand est polydentate.
\end{itemize}

\begin{transition}
Comment peut-on synthétiser les complexes ?
\end{transition}

\subsubsection{Synthèse d'un complexe}

On s'intéresse ici à la synthèse du complexe oxalatofer (III) :
\begin{equation*}
\mathrm{Fe}^{3+} + 3 \mathrm{C_2O_4}^{2-} \rightarrow \mathrm{[Fe(C_2O_4)_3]}^{3-}
\end{equation*}
La constante d'équilibre de cette réaction est appelée constante de formation globale du complexe $\beta$ telle que
\begin{equation*}
\beta = \frac{[\mathrm{[Fe(C_2O_4)_3]}^{3-}](c^0)^3}{[\mathrm{Fe}^{3+}][\mathrm{C_2O_4}^2-]^3},
\end{equation*}
où $c^0 = \unit{1}{\mole\per\liter}$.

\begin{experience}
\textbf{Synthèse du complexe $\mathrm{K_3[Fe(C_2O_4)_3],3H_2O}$.}
\end{experience}

\begin{transition}
On a évoqué le rôle des complexes comme catalyseur, mais ils sont très souvent rencontrés en biochimie.
\end{transition}

\subsection{Complexes bioinorganiques}

\subsubsection{Transport de l'oxygène}

\begin{slide}
\textbf{Transport du dioxygène.}
\end{slide}

\begin{itemize}
\item \href{https://www.rts.ch/decouverte/sante-et-medecine/corps-humain/9852272-comment-s-effectue-le-transport-du-dioxygene-dans-les-hematies-.html}{Comment s'effectue le transport du dioxygène dans les hématies?}
\end{itemize}

\subsubsection{Un complexe en chimiothérapie}

\begin{slide}
\textbf{Le cisplatine en chimiothérapie.}
\end{slide}
\begin{itemize}
\item Chapitre 1, l'activité anticancéreuse du cisplatine (extrait de thèse)
\end{itemize}

\subsection{Conclusion}