\section{LP01 Gravitation}

\begin{header}
\begin{tabular}{p{0.4\textwidth} l}
\niveau & \prerequis \\
CPGE & \textbullet{} Cinématique du point \\
     & \textbullet{} Théorèmes de la dynamique \\
     & \textbullet{} Électrostatique.
\end{tabular}

\noindent
\objectif
Formaliser l'interaction gravitationnelle et étudier le mouvement d'une particule massique dans un champ de pesanteur. 
\end{header}

{
\subsubsection*{Bibliographie}
\footnotesize{}
\begin{itemize}
\item Page Wikipedia sur la \href{https://fr.wikipedia.org/wiki/Constante_gravitationnelle}{constante gravitationnelle} pour la mesure de $G$ ;
\item \cite{Michel2017} chapitre 13
\item \cite{Faroux1996}
\item \cite{Salamito2016}
\item \cite{Bocquet2002}
\end{itemize}
}

\begin{remarque}
Revoir \cite{Michel2017} p472 pour des rappels d'odg sur le système solaire.
Pour ce qui est du niveau : plutôt première année sauf l'analogie avec l'électrostatique qui ne peut être réalisée qu'en deuxième année.
\end{remarque}

\subsection*{Introduction}

La gravitation décrit l'interaction entre des objets de masse non nulle.
Il s'agit d'une des quatre interactions fondamentales.
\begin{slide}
\textbf{Les quatre interactions fondamentales.}
\end{slide}
Elle est de loin la plus faible mais régit l'évolution des astres, ce qui s'explique par la neutralité de la matière et la longue portée des interactions considérées.

Plusieurs modèles des interactions gravitationnelles dans le but d'expliquer le mouvement des objets massifs et notamment celui des astres.
\begin{slide}
\textbf{Plusieurs théories de la gravitation.}
Dire que la théorie de Newton marche très bien dans le cas des masses faibles (incluant le Soleil, ou presque !).
\end{slide}

\begin{transition}
Nous allons décrire dans un premier l'interaction gravitationnelle dans le cadre de la mécanique newtonienne.
\end{transition}

\subsection{Interaction gravitationnelle}

\subsubsection{Force et énergie}

Justifier la forme de l'interaction proposée par Newton \cite{Faroux1996} p146.
Décrire la force gravitationnelle \cite{Michel2017} p337 et \cite{Salamito2016} p569 : à distance, point d'application, attractive, expression.
Principe des actions réciproques.
Parler de $G$ : valeur et mesure \cite{Faroux1996} p148.
Introduire le champ gravitationnel \cite{Salamito2016} p569.

Faire apparaitre l'énergie potentielle gravitationnelle en calculant le travail de la force gravitationnelle \cite{Salamito2016} p622.
Définition d'une force concervative.

\begin{transition}
On remarque une grande similitude entre la force gravitationnelle et l'interaction coulombienne.
Peut-on faire un parallèle entre la gravitation et l'électrostatique ?
\end{transition}

\subsubsection{Champ gravitationnel}

Dresser le parallèle entre la gravitation et l'électrostatique :
\begin{itemize}
\item écrire les forces;
\item montrer les quantités analogues ($m \leftrightarrow q$, $1/4\pi\varepsilon_0 \leftrightarrow -G$);
\item analogie entre ${\cal{G}}$ et $E$.
\end{itemize}
\begin{remarque}
Attention au signe moins devant $G$ ! 
\end{remarque}

Écrire le théorème de Gauss gravitationnel \cite{Faroux1996} p149.
Calculer le champ pour une distribution sphérique de masse volumique uniforme et la tracer en insistant sur les symétries.
Insister sur le fait qu'en dehors de la sphère tout se passe comme si toute la masse était concentrée sur le centre de la sphère.
Théorème de superposition.
Faire l'application numérique pour retrouver l'accélération de pesanteur terrestre.
\begin{remarque}
Le rotationnel de $\overrightarrow{\cal{G}}$ étant nul, on peut aussi définir un potentiel gravitationnel pour le calcul de l'énergie gravitationnelle.
\end{remarque}

Limites du parallèle :
\begin{itemize}
\item force toujours attractive ;
\item pas de masse négative ;
\item pas de champ magnétogravitationnel (en fait si mais compliqué~\cite{Mashhoon2007}, mentionner les ondes gravitationnelles).
\end{itemize}

\begin{transition}
Et le poids dans tout ça ?
Manifestation quotidienne de la gravité.
\end{transition}

\subsubsection{Le poids}

Dire que le poids résulte de la force gravitationnelle et de la force centrifuge (moins de 1\%).

Application de la gravimétrie pour trouver du pétrole etc.

\begin{experience}
\textbf{Mesure de $g$ d'après la période d'oscillation d'un pendule simple.}
Lors des TP de préparation, faire le calcul avec le pendule pesant pour voir si ça améliore la sensibilité.
Plusieurs façons de d'exprimer $T$ : $\sqrt{l/g}$, Borda, intégration.
On pourrait aussi mesurer l'accélération d'une masse en chute libre par analyse d'une vidéo.
\end{experience}

\begin{slide}
\textbf{La Terre homogène ?}
L'unité du graphe est le gal où $\unit{1}{gal} = \unit{1}{\centi\meter\per\second\squared}$.
L'échelle de couleur donne une variation de \unit{100}{\milli gal}, soit $\unit{10^{-3}}{\meter\per\second\squared}$.
\end{slide}

\begin{transition}
Historiquement, les lois de la gravitation ont été établies pour expliquer les observations astronomiques concernant le mouvement des astres dans le système solaire.
\end{transition}

\subsection{Mouvement dans un champ gravitationnel}

Suivre \cite{Michel2017} à partir de p454.
Objectif : étudier le mouvement d'une particule massique dans un champ de pesanteur
Mentionner les trois lois de Kepler et formuler la première : \cite{Michel2017} p460.
On va vérifier cette loi et les autres.

\subsubsection{Position du problème}

On s'intéresse plus particulièrement au mouvement d'une masse ponctuelle autour d'une autre beaucoup plus massive (Terre-Soleil par exemple : le Soleil est fixe).
Position du problème :
\begin{itemize}
\item justifier l'approximation de masse ponctuelle;
\item faire un schéma ;
\item référentiel \cite{Salamito2016} p562 ;
\item bilan des forces, pas de frottements ;
\item PFD et montrer la conservation de la quantité de mouvement du barycentre ;
\item un mot sur le mobile fictif pour la généralisation du problème.
\end{itemize}

C'est un mouvement à force centrale dans un champ newtonien (en $1/r^2$).

\begin{remarque}
L'obtention de l'équation de la trajectoire elliptique n'est pas au programme en PCSI mais elle l'est en MPSI où ils utilisent explicitement les coniques.
La démonstration passe soit par les formules de Binet avec le changement de variable $u=1/r$ où par l'invariant de Runge-Lenz (cf Tout en un MPSI Dunod).

\noindent
Le mobile fictif n'est plus au programme mais si besoin, il est bien fait dans \cite{Bocquet2002} p128.
\end{remarque}

\begin{transition}
Dans ce cadre, étudions le mouvement d'une particule massique dans un champ de gravité.
\end{transition}

\subsubsection{Conservation du moment cinétique}

Établir la conservation du moment cinétique et en déduire :
\begin{itemize}
\item la planéité du mouvement : dans le plan défini par le vecteur vitesse et la force ce qui justifie l'utilisation de coordonnées polaires ;
\item la constante des aires ;
\item la vitesse aréolaire et donc la loi des aires.
\end{itemize}

Formuler la 2ème loi de Kepler : \og le rayon Soleil-planète balaie des aires égales pendant des intervalles de temps égaux \fg{}.

\begin{slide}
\textbf{Vitesse aréolaire.}
Développer le cas circulaire et elliptique et montrer les étoiles au centre de la voie lactée orbitant autour de Sagitarius A : \href{https://www.youtube.com/watch?v=wyuj7-XE8RE}{simulation} ou \href{https://www.youtube.com/watch?v=TF8THY5spmo}{timelapse}.
On l'a illustré dans le cas particulier de trajectoires elliptiques mais c'est aussi valable dans le cas général
\end{slide}

\begin{transition}
La forme des trajectoires est-elle toujours elliptique ?
\end{transition}

\subsubsection{Conservation de l'énergie}

Introduire le potentiel effectif en éliminant $\dot{\theta}$ avec la constante des aires et le tracer en faisant apparaitre les deux contributions.
Problème à un ddl \cite{Bocquet2002} p124.

Faire apparaitre les trois régimes avec le nom des trajectoires.
Retour sur la première loi de Kepler qui est un cas particulier.

\begin{transition}
On souhaite connaitre la période des trajectoires fermées.
\end{transition}

\subsubsection{Période du mouvement}

Dans le cas circulaire, établir l'expression de la vitesse de l'orbite et donner la période du mouvement.
Faire l'application numérique avec la station spatiale internationale et les dires de Thomas Pesquier ($T=\unit{92{,}69}{min}$, cf notebook).

Généraliser aux trajectoires elliptiques pour donner et donner la troisième loi de Kepler \cite{Michel2017} p463.

\begin{slide}
\textbf{Vérification de la troisième loi de Kepler.}
\end{slide}

\begin{transition}
Nous avons bien retrouvé les loi empiriques de Kepler avec l'interaction gravitationnelle proposée par Einstein.
Voyons quelques cas particuliers
\end{transition}

\subsection{Applications et limite}

Toujours dans \cite{Michel2017}.

\subsubsection{Vitesses cosmiques}

Surtout parler de la vitesse de libération et évoquer les trous noirs.

Vitesse de libération sur terre : \unit{11}{\kilo\meter\per\second}.

\subsubsection{Orbite géostationnaire}

Faire les calculs après avoir fait le raisonnement pour montrer qu'elle est unique.
Il y a actuellement environ 300 satellites sur l'orbite, soit un tous les \unit{900}{\kilo\meter}.

\subsubsection{Vers des masses non-ponctuelles}

\cite{Sanz2016} p212 pour la non verticalité de $\overrightarrow{g}$.

Effets de marée, synchronisation des orbites, etc.

\subsection*{Conclusion}

La dernière section peut largement être mise en conclusion si le temps vient à manquer...

On peut remarquer l'équivalence entre masse grave et masse inerte, curiosité de la mécanique newtonienne mais postulat de la RG.
Ouvrir sur la RG avec le retour sur le cas de Mercure dans l'article d'Einstein \cite{Faroux1996} p165.

\begin{remarque}
\href{https://fr.wikipedia.org/wiki/Point_de_Lagrange}{\textbf{Points de Lagrange :}} dans un système à deux corps orbitant l'un autour de l'autre, les cinq points de Lagrange correspondent à des positions où la force centripète due aux deux objets massifs permet à un troisième de masse négligeable de rester fixe par rapport aux deux premiers.
Il y en a deux stables $L_4$ et $L_5$, les trois autres sont instables et généralement occupés par des satellite artificiels.
LISA pathfinder a été envoyé en 2015 en direction du point $L_1$ du système Terre-Soleil.
\end{remarque}

\begin{funfact}
À propos des travaux d'\href{https://fr.wikipedia.org/wiki/Urbain_Le_Verrier}{Urbain Le Verrier}, astronome et mathématicien français spécialisé en mécanique céleste.
Le 31 août 1846, il annonce l'existence, donne les propriétés et la position de Neptune d'après l'étude des irrégularités de l'orbite d'Uranus.
François Arago dira \og M. Le Verrier vit le nouvel astre au bout de sa plume \fg{}. 
Cette découverte est confirmée par Johann Gottfried Galle le 23 septembre de la même année qui observe le ciel dans la direction des prédictions de Le Verrier.
Plus tard il tente d'expliquer l'anomalie de l'orbite de Mercure par l'existence d'une autre planète, Vulcain, qui ne sera jamais confirmée : il faudra attendre la RG pour donner l'explication appropriée.

\noindent
Dans le cadre de la mécanique newtonienne, toute orbite autour d'un point massif est stable : le rayon de l'orbite impose la vitesse de l'objet.
En RG et en particulier pour les trous noirs, ce n'est pas le cas et il existe un rayon minimal en dessous duquel aucune orbite circulaire n'est stable : ISCO pour innermost stable circular orbit.
Le rayon de cette orbite $r_\mathrm{ISCO}$ dépend du spin du trou noir : plus il tourne vite, plus $r_\mathrm{ISCO}$ est faible si le disque d'accrétion tourne dans le même sens.
En observant le disque d'accrétion autour d'un trou noir on peut ainsi déterminer son spin.
\end{funfact}

\newpage