\section{LP19 Diffraction de Fraunhofer}

\begin{header}
\begin{tabular}{p{0.4\textwidth} l}
\niveau & \prerequis \\
Licence & \textbullet{} Modèle scalaire des ondes lumineuses \\
        & \textbullet{} Interférence \\
        & \textbullet{} Transformée de Fourier
\end{tabular}

\noindent
\objectif
Dégager les principales propriétés du phénomène de diffraction dans l'approximation de Fraunhofer et mettre en évidence son rôle dans la formation des images.
\end{header}

{
\subsubsection*{Bibliographie}
\footnotesize{}
\begin{itemize}
\item \cite{Sanz2016}
\item \cite{Olivier2000}
\item \cite{Fruchart2016}
\item \cite{Hecht2002}
\item \cite{Sayrin2019} : \href{http://www.lkb.upmc.fr/cqed/teaching/teachingsayrin/}{TD d'optique de Clément Sayrin}, Diffraction (1) et Difraction (2) : Applications
\end{itemize}
}

\begin{remarque}
Dans le cadre des nouveaux programmes de CPGE, la diffraction est vue essentiellement à travers l'étude des fréquences spatiales de l'objet diffractant et seulement dans l'approximation de Fraunhofer.
On ne peut donc placer cette leçon qu'en licence.
\end{remarque}

\subsection*{Introduction}

\begin{experience}
\textbf{Fente simple éclairée par un laser.}
L'objectif de cette expérience est simplement de mettre en évidence le phénomène de diffraction : on ne cherche pas à être strictement dans les conditions de Fraunhofer même si on les approche largement.
Relire \cite{Fruchart2016} p284 et 319 pour de arguments basés sur le nombre de Fresnel.
\end{experience}

La diffraction apparait dès lors qu'une onde est limitée transversalement quelque soit la nature de l'onde.

\begin{slide}
\textbf{Limite de résolution des instruments optiques liée à la diffraction.}
L'image de droite est pixelisée mais avoir des capteurs plus précis est ici inutile car la diffraction donne des taches plus grandes que la taille d'un pixel.
\end{slide}

Dire que l'on va aussi voir des cas où la diffraction peut être utilisée pour du filtrage dans l'espace de Fourier.

\subsection{Description de la diffraction}

\subsubsection{Principe de Huygens-Fresnel}

On sait que la lumière peut être transmise, réfléchie, réfractée mais aussi diffractée : travaux de Grimaldi du milieu du $\mathrm{XVIIe}$ siècle \cite{Sanz2016} p833.
Donner les principaux noms associés à la diffraction : Huygens, Fresnel, Kirchoff \cite{Sanz2016} p833-834.

Faire un schéma et énoncer le principe d'Huygens-Fresnel et donner son expression mathématique en oubliant le facteur $A(\theta)$ : \cite{Sayrin2019} p1.

\begin{remarque}
La diffraction de Fraunhofer a été nommée en hommage à ce dernier même s'il n'a pas pris part directement au développement de la théorie.

\noindent
Savoir parler du théorème de Fresnel-Kirchhoff, expression plus tardive démontrée à partir de solutions aux équations de Maxwell \cite{Fruchart2016} p296 pour la version courte ou \cite{Hecht2002} p527.
\end{remarque}

\begin{transition}
Ajoutons un objet diffractant.
\end{transition}

\subsubsection{Diffraction par un objet quelconque}

Suivre \cite{Sayrin2019} p2.
Il faut faire attention à bien justifier l'ordre auquel on fait le DL :
\begin{itemize}
\item pour l'amplitude : ordre 0 car des petites variations vont très légèrement modifier l'intensité sur l'écran.
\item pour la phase : ordre 2 car il faut tenir compte des termes qui modifient le chemin optique à l'échelle de la longueur d'onde.
\end{itemize}
Attention aussi à justifier correctement l'omission des termes constants sur la pupille dans l'expression de la phase.

\begin{transition}
On peut encore simplifier cette relation en se plaçant dans le cadre de l'approximation de Fraunhofer.
\end{transition}

\subsubsection{Approximation de Fraunhofer}

Sous certaines conditions, on peut négliger les termes quadratiques restants.
Arriver à l'expression finale \cite{Sayrin2019} p2 et reconnaitre la transformée de Fourier de la transmission de l'objet diffractant \cite{Sayrin2019} (2) p5.
Introduire la notion de fréquence spatiale.

Donner les cas pratiques de l'observation de la diffraction de Fraunhofer \cite{Sayrin2019} p3 en les illustrant par l'experience.

\begin{experience}
\textbf{Montage à une lentille pour la diffraction de Fraunhofer.}
Faire d'abord le montage à deux lentilles et e ramener à une seule.
Il faut être très clair sur cette partie et en conclure que la diffraction intervient dans la formation des images.
\end{experience}

\begin{remarque}
Pour simplifier, on peut peut être mener tout ces calculs en supposant $\alpha_0 = \beta_0 = 0$, c'est-à-dire que le vecteur d'onde incident est normal à la surface de l'objet diffractant et donner à l'oral l'effet qu'aura une incidence quelconque : déplacement de l'image géométrique donc déplacement de la figure de diffraction.

\noindent
Le nombre de Fresnel \cite{Fruchart2016} p284 permet de savoir si l'on est dans le cadre de la diffraction de champ proche (Fresnel) ou bien dans celui de la diffraction de champ lointain (Fraunhofer).

\noindent
Voir le Sextant p139 pour une autre interprétation du montage à une lentille.
\end{remarque}

\begin{transition}
Voyons quelques figures classiques de diffraction. 
\end{transition}

\subsection{Figures de diffractions}

\subsubsection{Fente rectangulaire}

Donner l'expression de l'éclairement sur l'écran obtenu dans le cas d'une fente rectangulaire \cite{Sayrin2019} p4 en faisant directement Intervenir une transformée de Fourier d'une fonction rectangle.
Expliquer l'image obtenue sur l'écran : pas de diffraction dans la direction de la petite fréquence spatiale.

\begin{experience}
\textbf{Largeur de la tache de diffraction associée à une fente fine.}
Mesurer la largeur de la tache centrale en fonction de la largeur de la fente source.
Mettre aussi en évidence que la position de la fente n'influe pas sur la position de la figure d'interférence.
\end{experience}

\begin{transition}
La plupart des instruments optiques ont plutôt une ouverture circulaire.
\end{transition}

\subsubsection{Ouverture circulaire}

Encadrer le diamètre de la tache de diffraction avec deux carrés : \cite{Sayrin2019} p4.
Dire qu'en réalité, il faut faire intervenir une fonction de Bessel de première espèce et donner l'expression du diamètre angulaire en fonction du rayon de l'ouverture.

\begin{experience}
\textbf{Tache d'Airy.}
On peut éventuellement montrer l'expérience avec les spores de lycopode pour parler du théorème de Babinet : dans ce cas il faudrait sans doute aller jusqu'au bout de l'expérience et mesurer la taille de la tache d'Airy pour en déduire la taille des spores.
\end{experience}

\begin{transition}
Revenons sur les limites de résolutions.
\end{transition}

\subsubsection{Limites de résolutions}

\begin{slide}
\textbf{Critère de Rayleigh.}
\end{slide}

\begin{remarque}
On peut parler de beaucoup de choses ici : astronomie, microscopie, etc.
À adapter donc en fonction du temps.
Relire le principe de l'apodisation \cite{Sayrin2019} (2) et se rappeler que c'est le même principe que le fenêtrage effectué lors des mesures de spectres.
C'est normal, on procède également à une transformée de Fourier du signal tronqué.
\end{remarque}

\begin{transition}
La diffraction n'est pas toujours un problème
\end{transition}

\subsection{Filtrage}

\begin{experience}
\textbf{Croix d'Abbe.}
Montrer la figure de diffraction, dire que c'est la TF de l'objet diffractant et montrer l'image de la grille, TF inverse de la figure de diffraction.
Raisonner en terme de fréquence spatiale : les faibles fréquences spatiales donnent des faisceaux proches de l'image géométrique et inversement.
Que se passe-t-il si on élimine les composantes haute fréquence du spectre dans l'espace de Fourier ?
On agit directement sur le spectre et on peut supprimer des composantes de l'image.
\end{experience}

Applications : microscopie de champ sombre, microscopie à contraste de phase, strioscopie, filtrage d'image, etc.

\begin{slide}
\textbf{\href{https://youtu.be/4tgOyU34D44?t=11}{Strioscopie.}}
Voir \cite{Hecht2002} p638 sur la méthode de Schlieren.
\end{slide}

\subsection*{Conclusion}

Faire ressortir que malgré la lourdeur apparente du formalisme mathématique associé au phénomène de diffraction, on peut déduire simplement l'allure de la tache de diffraction dans associé à un objet diffractant en étudiant les fréquences spatiales de l'objet.
Redonner les limitations dues à la diffraction mais aussi rappeler qu'on peut en tirer profit avec l'optique de Fourier : il s'agit d'un filtrage analogique.
Ce n'est pas le seul cas ou le filtrage spectral est utilisé.

\newpage