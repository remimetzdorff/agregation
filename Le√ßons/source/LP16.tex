\section{LP16 Microscopies optiques}

\begin{header}
\begin{tabular}{p{0.4\textwidth} l}
\niveau & \prerequis \\
Template& \textbullet{} Template \\
        & \textbullet{} Template \\
\end{tabular}

\noindent
\objectif
Template
\end{header}

{
\subsubsection*{Bibliographie}
\footnotesize{}
%\begin{itemize}
%\item 
%\end{itemize}
}

\begin{remarque}
L'ouverture numérique $n\sin\theta$ est aussi l'invariant de Lagrange du microscope car il est invariant au cours du trajet dans le microscope.

\noindent
Parler de l'\href{http://www.optique-ingenieur.org/fr/cours/OPI_fr_M03_C03/co/Contenu_03.html}{éclairage de Köhler} et de l'éclairage en général : pas au cœur de la leçon mais 

\noindent
Autres types de microscopie optique : confocale, de champ proche, à deux photons, super résolution.
Être conscient que souvent ce n'est pas l'œil qui est utilisé comme capteur.

\noindent
Parler de lentille de tube dans les microscopies récentes qui permet d'avoir un éclairage parallèle dans le tube pour utiliser des composants optiques variés facilement.
Grâce à ça on peut faire un zoom : adapter le grossissement sans changer d'objectif.

\noindent
Il faut garder du temps pour parler des méthodes récentes : pas passer trop de temps sur la diffraction qui n'est pas propre au microscope.

\noindent
Pour parler de microscopie confocale, il faut parler de profondeur de champ.
\end{remarque}


\subsection*{Introduction}

\subsection{Template}

\subsubsection{Template}

\subsection*{Questions (JH)}

\begin{enumerate}
\item Vous nous avez parlé de la longueur du tube. Est ce que la courte focale de l'objectif vient de la longueur du tube ? Est ce que ça relaxe la condition d'avoir un objet proche du foyer ?
\item Pour déplacer la mise au point, il faut déplacer la platine d'après vous. Dans la réalité, c'est le microscope qu'on déplace. Commentez.
\item De quoi dépend la mise au point ? Et si on n'y est pas ? Ça donne une image floue ?
\item Sur la photo du microscope, l'axe de l'oculaire n'est pas celui de l'axe optique. Commentez.
\item Qu'est ce qui fait qu'on reste proche de la mise au point quand on change d'objectif ? Quelles conditions sur la géométrie de l'ensemble ?
\item Que se passe-t-il si l'objet est au foyer objet de l'objectif ?
\item Dans certains microscope on a un faisceau parallèle dans le tube. Comment ça marche ?
\item Est ce qu'on peut relier les diaphragmes aux pupille et lucarne de l'appareil ?
\item Qu'est ce qu'on peut dire des bords du diaphragme d'ouverture ?
\item Est ce vous pouvez me parler de comment ça marche un objectif à immersion ?
\item Comment voit-on apparaître nsin(u) ?
\item En quoi le microscope à contraste de phase augmente la résolution ?
\item Pouvez-vous commenter l'illumination du microscope en général ?
\item Si vous aviez présenté le microscope en champ sombre, auriez-vous dit la même chose ?
\item Où placer la lame quart d'onde ?
\end{enumerate}

\newpage