\section{LP16 Microscopies optiques}

\begin{header}
\begin{tabular}{p{0.4\textwidth} l}
\niveau & \prerequis \\
Licence & \textbullet{} Optique géométrique \\
		& \textbullet{} Optique ondulatoire \\
        & \textbullet{} Diffraction \\
        & \textbullet{} Interférences \\
        & \textbullet{} Polarisation
\end{tabular}

\noindent
\objectif
Comprendre le fonctionnement et les caractéristiques du microscope à deux lentilles, ses limitations et présenter une technique moderne.
\end{header}

{
\subsubsection*{Bibliographie}
\footnotesize{}
\begin{itemize}
\item \href{http://www.agregation-physique.org/index.php/annales-des-epreuves-ecrites/54-sujets-et-corriges-annee-2015}{Sujet de l'agrégation externe de 2015} : Microscopies
\item \href{http://www.lkb.upmc.fr/cqed/teaching/teachingsayrin/}{TD d'optique de Clément Sayrin} : Optique géométrique et Diffraction(2) : Applications
\item \cite{Hecht2002}
\item \cite{Perez2017}
\item \cite{Salamito2016}
\item \cite{Sanz2016}
\item \cite{Wastiaux1994}
\item \href{https://www.nikonsmallworld.com/techniques}{Nikon Small World}
\item \href{https://www.microscopyu.com/}{MicroscopyU}
\item \cite{Kastler1948} \href{bupdoc.udppc.asso.fr/consultation/article-bup.php?ID_fiche=3494}{BUP}
\item \cite{Faget1962} \href{bupdoc.udppc.asso.fr/consultation/article-bup.php?ID_fiche=15314}{BUP}
\item \cite{Zanier2015} : des manips de malade sur le filtrage dans l'espace de Fourier.
\end{itemize}
}

\begin{remarque}
La leçon est placée niveau licence à cause de la dernière partie qui n'est pas explicitement au programme de CPGE.
Il me semble toutefois que le programme de CPGE fournit tous les pré-requis nécessaires à la compréhension de la microscopie à contraste de phase et de la microscopie confocale.

\noindent
Une technique moderne est explicitement attendue par le jury.
Il faut garder du temps pour la présenter correctement.

\noindent
Deux façons possibles de faire cette leçon :
\begin{enumerate}
\item présenter le microscope à deux lentilles pour illustrer les caractéristiques d'un microscope et présenter plus qualitativement une technique moderne ;
\item supposer en pré-requis que le microscope à deux lentilles est connu et présenter directement des méthodes modernes.
Probablement trop ambitieux...
Et pour le coup, difficile de caser la manip quantitative dans la leçon.
\end{enumerate}
\end{remarque}

\subsection*{Introduction}

Le pouvoir de résolution de l'œil est d'une minute d'arc : cela correspond à \unit{100}{\kilo\meter} à la surface de la lune ou \unit{0{,}1}{\milli\meter} au puctum proximum.
On veut voir mieux : ici on s'intéresse à l'observation de tous petits objets.

\begin{remarque}
La limite de résolution de l'œil est donnée par la densité de cône sur la rétine, mais elle correspond aussi à la limite de diffraction par la pupille.
\end{remarque}

\begin{slide}
\textbf{Petite histoire du microscope moderne.}
Voir la \href{https://fr.wikipedia.org/wiki/Microscope_optique}{page Wikipedia} sur la microscopie optique pour l'historique.
\end{slide}

\subsection{Microscope à deux lentilles}

Dans cette partie on dégage les propriétés du microscope à deux lentilles tout en le construisant en live.
Faire des schéma propres !
Les calculs sont fait dans le TD de Clément Sayrin.

\begin{experience}
\textbf{Modélisation de l'oeil.}
Faire l'image d'un objet par une lentille de focale \unit{33}{\centi\meter} sur un écran.
Introduire le punctum proximum (pp) qui limite la taille de l'objet que l'on peut voir et donner sa valeur.
\end{experience}

L'idée du microscope est d'obtenir une image agrandie que l'on observe ensuite à la loupe.
C'est un dispositif optique centré.

\begin{transition}
Voyons une première partie du microscope : l'oculaire.
\end{transition}

\subsubsection{L'oculaire}

Dire que l'oculaire agit comme une loupe, donner son grossissement commercial.
L'image à l'infini permet une observation sans fatigue car l'œil n'accommode pas.

\begin{experience}
\textbf{Grossissement par une loupe.}
Utiliser une lentille de focale \unit{20}{\centi\meter} avec le faux œil.
\end{experience}

Montrer l'oculaire du microscope de la collection et donner ses caractéristiques.
à comparer à une lentille simple : les grossissements commerciaux accessibles sont limités à 2 ou 3.
Des focales plus courtes présentes des aberrations importantes.

\begin{remarque}
Lire \cite{Hecht2002} p222 pour les raisons qui limitent le grossissement d'une simple loupe.
\end{remarque}

\begin{transition}
On peut augmenter l'efficacité du dispositif en formant au préalable une image agrandie de l'objet, c'est le rôle de l'objectif.
\end{transition}

\subsubsection{L'objectif}

\begin{experience}
\textbf{Modifier le montage pour y insérer l'objectif de focale \unit{10}{\centi\meter}.}
Expliquer en même temps le principe.
\end{experience}

Exprimer le grandissement de l'objectif.
En plaçant le foyer objet de l'objectif proche de l'objet, on obtient un fort grandissement.
La focale de l'objectif doit être courte pour obtenir des dispositifs compacts.
L'obtention de courtes focales propres est délicate : l'objectif est un système complexe.

\begin{transition}
On a tout pour faire un microscope.
\end{transition}

\subsubsection{Caractéristiques du microscope}

Établir l'expression du grandissement commercial du microscope et l'exprimer en fonction des caractéristiques de l'objectif et de l'oculaire.
Introduire la longueur de tube (typiquement \unit{160}{\milli\meter}).
Donner des valeurs classiques : oculaire $\times 10$ et objectif $\times 10$ jusqu'à $\times 100$.

\begin{experience}
\textbf{Mesure du grossissement commercial du microscope de la collection.}
Montrer le cercle oculaire.
\end{experience}

\begin{slide}
\textbf{Microscope à deux lentilles.}
Parler des diaphragmes de champ et d'ouverture en réalité, le diaphragme d'ouverture est tellement proche de l'objectif qu'il s'agit de la lentille elle même.
La puissance intrinsèque est le rapport de l'angle sous lequel on voit l'objet à travers le microscope et la taille de l'objet : voir question 10 du sujet.
La profondeur de champ définie ici est propre à l'observation à l'œil, mais être conscient que souvent ce n'est pas l'œil qui est utilisé comme capteur.
La profondeur de champ est alors définie de manière classique.
Donner l'ordre de grandeur : quelques microns.
\end{slide}

\begin{remarque}
Transition vers les microscopes actuels : lentille de tube qui permet d'avoir des rayons parallèles dans le tube pour utiliser des composants optiques variés facilement : on n'est plus sensible à la position longitudinale.
Grâce à ça on peut notamment faire un zoom : adapter le grossissement sans changer d'objectif.
\end{remarque}

\begin{transition}
La conception de microscopes performants est soumise à de nombreuses contraintes.
Il faut tenir compte des phénomènes qui peuvent limiter la qualité de l'image obtenue pour les atténuer.
\end{transition}

\subsection{Limitations}

\subsubsection{Résolution latérale}

Pas passer trop de temps sur cette partie qui n'est pas propre au microscope.

\begin{slide}
\textbf{Critère de Rayleigh.}
\end{slide} 

Justifier que le diaphragme d'ouverture est l'objet diffractant : question 34 du sujet.
Établir l'expression de résolution latérale minimale : question 35 et 36.
Donner la résolution maximale accessible : de l'ordre de la longueur d'onde.
\begin{remarque}
D'après cette relation, la solution la plus efficace pour améliorer la résolution est de diminuer la longueur d'onde.
Le gain est toutefois limité si l'on souhaite rester dans l'optique : la microscopie électronique a été développée justement parce qu'en jouant sur l'énergie des électrons, on peut descendre à des longueurs de de Broglie très faibles :
\begin{equation}
\lambda_\mathrm{deBroglie} = \frac{h}{p} = \frac{h}{\gamma mv}
\end{equation}
où $\gamma$ est le facteur de Lorentz.
Avec une énergie de \unit{100}{keV}, on trouve $\lambda_\mathrm{deBroglie}=\unit{3{,}7}{\pico\meter}$.
\end{remarque}

\begin{slide}
\textbf{Objectif à immersion.}
\end{slide}

\begin{remarque}
La quantité $nAB\sin\theta$ est appelé invariant de Lagrange-Helmohltz et est conservée au cours du trajet dans le microscope.
Voir condition du sinus d'Abbe qui traduit mathématiquement l'aplanétisme du système.

\noindent
Les techniques permettant d'aller au delà de la limite de diffraction sont dite super-résolues.
Elles sont basées sur l'utilisation de molécules fluorescentes et utilisent des méthodes d'éclairage très particulières.
On peut citer la méthode \href{https://science.sciencemag.org/content/313/5793/1642}{PALM} par exemple, qui a été récompensée par le prix Nobel de chimie 2014.
Voir aussi la \href{https://fr.wikipedia.org/wiki/Microscopie_PALM}{page Wikipedia} ou mieux \href{https://www.microscopyu.com/techniques/super-resolution/single-molecule-super-resolution-imaging}{MicroscopyU}.

\noindent
Il existe aussi une limite de diffraction longitudinale : voir \href{https://www.microscopyu.com/techniques/super-resolution/the-diffraction-barrier-in-optical-microscopy}{MicroscopyU}.
\end{remarque}

\begin{transition}
Augmenter l'ouverture numérique n'est pas sans conséquence sur le choix des optiques du microscope.
\end{transition}

\subsubsection{Aberrations}

\begin{slide}
\textbf{Aberrations.}
Si le temps le permet, relire \cite{Hecht2002} p266.
\end{slide}

\begin{remarque}
Il existe des doublets achromatiques qui corrigent les aberrations chromatiques pour deux longueurs d'ondes et géométriques pour une, mais il existe aussi des \href{https://fr.wikipedia.org/wiki/Triplet_apochromatique}{triplets apochromatiques} qui corrigent les abérations chromatiques pour trois longueurs d'onde et géométriques pour deux.
\end{remarque}

\begin{transition}
L'éclairage est un élément important pour la qualité des observations.
\end{transition}

\subsubsection{Éclairage}

Sauter cette section si manque de temps.

Un bon éclairage est fondamental pour obtenir des images de qualité : il faut un éclairage intense pour obtenir une image lumineuse mais aussi uniforme.
Si l'on focalise les rayons issus de la source, on fait l'image du filament sur la préparation : le filament sera aussi visible à travers le microscope.
On peut utiliser des diffuseurs mais le mieux est de réaliser un \href{http://www.optique-ingenieur.org/fr/cours/OPI_fr_M03_C03/co/Contenu_03.html}{éclairage de Köhler}.

\begin{remarque}
Dans l'éclairage de Köhler, le diaphragme d'ouverture permet non seulement d'ajuster la luminosité de l'image mais aussi le degré de cohérence spatiale de l'image.
Si le diaphragme est quasiment fermé, on se rapproche d'une source ponctuelle et l'éclairage est cohérent.
La cohérence de l'éclairage modifie la résolution, le contraste et la profondeur de champ.
Voir dans le Taillet ou Houard s'il y a plus d'infos. 
\end{remarque}

\begin{transition}
Le principe du microscope présenté jusqu'ici repose sur l'utilisation d'objets faisant varier l'intensité du faisceau d'illumination.
De nombreux objets sont transparent et leur observation par des méthodes classiques donne lieu à des images faiblement contrastées.
\end{transition}

\subsection{Microscopie à contraste de phase}

Relire le raisonnement de \cite{Kastler1948} et \href{https://www.microscopyu.com/techniques/phase-contrast/introduction-to-phase-contrast-microscopy}{Introduction to phase contrast microscopy}.

Commencer par faire un schéma de principe de la diffraction de Fraunhofer en ajoutant la lentille de l'oculaire \cite{Faget1962} : éclairement par une source ponctuelle au foyer objet d'une lentille et observation de l'image de la préparation sur un écran.
Mettre en évidence le plan de Fourier et expliquer qualitativement le principe en raisonnant en transformée et transformée inverse de Fourier (fait dans le Taillet).

Faire le calcul du Bruhat.
On éclaire la préparation, considérée comme un objet de phase, en $A\sin(\omega t)$.
Après la préparation, l'amplitude de l'onde est 
\begin{equation}
A\sin(\omega t +\varphi) = A\varphi\sin(\omega t) + A\cos(\omega t),
\end{equation}
si les variations de phase introduites par l'objet sont suffisamment faibles.
Le premier terme porte l'information sur la phase et correspond au faisceau diffracté, le deuxième étant seulement responsable de l'éclairement.
Deux options dans le plan de Fourier :
\begin{itemize}
\item cacher le centre du plan de Fourier pour éliminer la composante d'éclairement.
L'image de la préparation est donc en $\varphi^2$ ce qui n'est pas idéal car pas d'information sur le signe de la phase et image très peu lumineuse.
\item déphaser l'un des deux faisceaux de $\pi/2$ avec une lame quart d'onde.
On obtient alors une image en $1+2\varphi$ au premier ordre en $\varphi$.
\end{itemize}
Dans les deux cas, on a bien transformé des variations de phase en variation d'intensité.

Méthode mise au point par Zernike dans la première moité du $\mathrm{XX^e}$ siècle qui lui a valu le prix Nobel de physique en 1953.
Donner les avantages : pas besoin de marquer la préparation et donc souvent de la tuer, observations en temps réel...

\begin{remarque}
Dans le cas où l'on ne fait que bloquer le faisceau non diffracté, cette méthode se rapproche de la microscopie en fond noir.
Dans ce cas le contraste est très grand mais l'image est très peu lumineuse.
\end{remarque}

\begin{slide}
\textbf{Contraste de phase.}
Pas sûr de comprendre pourquoi on utilise en pratique des ouvertures sous forme d'anneau : il semble que ce soit pour augmenter le contraste.
Les objets masquants sont partiellement opaques et déphasants : c'est la microscopie de Zernike.
\end{slide}

\subsection*{Conclusion}

On récapitule la leçon et on ouvre sur d'autres techniques modernes avec des jolies photos !

\begin{slide}
\textbf{Éclairage.}
\end{slide}

\begin{slide}
\textbf{Microscopie à contraste de phase.}
\end{slide}

\begin{slide}
\textbf{Et bien d'autres...}
\end{slide}

\begin{funfact}
Prix Nobel associés à la microscopie optique :
\begin{itemize}
\item 1953 (physique) : Frits (Frederik) Zernike pour la microscopie à contraste de phase ;
\item 1984 (physiologie ou médecine) : Georges J.F. Köhler pour ses travaux en rapport avec le système immunitaire ;
\item 2014 (chimie) : Eric Betzig pour la microscopie à fluorescence super-résolue.
\end{itemize}
Il y en a encore bien d'autres pour les microscopies autres qu'optique.
\end{funfact}

\subsection*{Questions (JH)}

\begin{enumerate}
\item Vous nous avez parlé de la longueur du tube. Est ce que la courte focale de l'objectif vient de la longueur du tube ? Est ce que ça relaxe la condition d'avoir un objet proche du foyer ?
\item Pour déplacer la mise au point, il faut déplacer la platine d'après vous. Dans la réalité, c'est le microscope qu'on déplace. Commentez.
\item De quoi dépend la mise au point ? Et si on n'y est pas ? Ça donne une image floue ?
\item Sur la photo du microscope, l'axe de l'oculaire n'est pas celui de l'axe optique. Commentez.
\item Qu'est ce qui fait qu'on reste proche de la mise au point quand on change d'objectif ? Quelles conditions sur la géométrie de l'ensemble ?
\item Que se passe-t-il si l'objet est au foyer objet de l'objectif ?
\item Dans certains microscope on a un faisceau parallèle dans le tube. Comment ça marche ?
\item Est ce qu'on peut relier les diaphragmes aux pupille et lucarne de l'appareil ?
\item Qu'est ce qu'on peut dire des bords du diaphragme d'ouverture ?
\item Est ce vous pouvez me parler de comment ça marche un objectif à immersion ?
\item Comment voit-on apparaître nsin(u) ?
\item En quoi le microscope à contraste de phase augmente la résolution ?
\item Pouvez-vous commenter l'illumination du microscope en général ?
\item Si vous aviez présenté le microscope en champ sombre, auriez-vous dit la même chose ?
\item Où placer la lame quart d'onde ?
\end{enumerate}

\newpage