\section{LP09 Conversion de puissance électromécanique}

\begin{header}
\begin{tabular}{p{0.4\textwidth} l}
\niveau & \prerequis \\
CPGE (PSI) & \textbullet{} Induction \\
           & \textbullet{} Force de Laplace \\
           & \textbullet{} Énergie électromagnétique \\
           & \textbullet{} Électrocinétique \\
           & \textbullet{} Notion de couple, puissance mécanique
\end{tabular}

\noindent
\objectif
Aborder la conversion réciproque d'énergie électromécanique par l'intermédiaire de l'énergie magnétique et mettre en évidence le rôle des ferromagnétiques.
\end{header}

\subsection*{Bibliographie}
{
\footnotesize{}
En gros : Polycopié de Jérémy et Dunod PSI (apparemment la dernière édition est la meilleure)
\begin{itemize}
\item \cite{Faroux1998}
\item \cite{Salamito2016}
\item \cite{Neveu2019}
\item \cite{Cardini2017}
\item \cite{Naval}
\end{itemize}
}

\begin{remarque}
Dans cette leçon le jury attend explicitement une approche énergétique pour le calcul du couple du moteur qui permet de mettre en avant le rôle de l'énergie magnétique. Cela n'empêche pas de faire l'approche avec le moment magnétique que je trouve plus parlante.
Le rôle du milieu ferromagnétique doit être mis en avant : canalisation parfaite des lignes de champ et $\overrightarrow{B}$ radial dans l'entrefer dans le cas idéal permettant un recouvrement parfait entre les champs statoriques et rotoriques.
J'imagine qu'on peut alors discuter des pertes si le temps le permet : avoir en tête les pertes de flux, par hystérésis, courants de Foucault, effet Joule et les façons de les diminuer.
Comme dans l'induction, il faut faire attention aux conventions. 
\end{remarque}

\subsection*{Introduction}

Électricité à la base de notre société : permet de transporter de l'énergie qu'il faut convertir en lumière, chaleur ou travail.
Les moteurs et générateurs sont des exemples de machines tournantes très fréquemment rencontrés depuis la production électrique jusqu'aux moyens de transport.
Ces machines sont constituées d'un stator et d'un rotor (définir).

\subsection{Induction et force de Laplace}

\begin{remarque}
Ne pas faire cette partie probablement par manque de temps et redondance avec le moteur continu.
Je l'ai écrite pour rien ???
Oui sans doute...
\end{remarque}

On peut probablement la faire rapidement mais en insistant lourdement sur la correspondance entre énergie méca et elec.

On s'intéresse à une barre en mouvement dans un champ magnétique uniforme.
On trouve :
\begin{equation}
e = lBv
\end{equation}
Explication physique : l'opérateur met en mouvement les particules chargées du conducteur qui sont déviées par le champ magnétique.
On peut alors assimiler cet effet à une force électromotrice $e$.

La puissance électrique est donnée par 
\begin{equation}
{\mathcal{P}}_e = ilBv
\end{equation}
Ici un déplacement crée une force électromotrice, mais on sait qu'un champ magnétique variable en crée aussi ce qui est important dans les machines électromécaniques : soit un conducteur se déplace dans un champ magnétique soit le champ magnétique est variable.
Faire le schéma équivalent électrique.

Si l'induction transforme l'énergie mécanique en énergie électromagnétique, la force de Laplace fait l'inverse.

Seule la force de Laplace est à prendre en compte :
\begin{equation}
F_L = -ilB
\end{equation}
dont la puissance est
\begin{equation}
{\mathcal{P}}_L = -ilBv
\end{equation}

On montre immédiatement l'égalité entre les puissances d'induction et de Laplace ce qui est l'essentiel de la conversion électromécanique.

\begin{remarque}
On peut partir du résultat ${\mathcal{P}}_L+{\mathcal{P}}_e = 0$ pour les considération énergétiques dans le cas où la partie n'est pas traitée.
\end{remarque}

\begin{transition}
Il existe des applications ou l'on souhaite faire des translations mais souvent une rotation plus utile.
\end{transition}

\subsection{Machine à courant continu}

\subsubsection{Cas d'une spire -- Laplace}

Suivre le cours de Jérémy

\begin{slide}
\textbf{Modèle de la machine à courant continu.}
\end{slide}

\begin{slide}
\textbf{Couple de la machine à courant continu.}
\end{slide}

\begin{transition}
On a un circuit mobile dans un champs magnétique constant ce qui conduit au phénomène d'induction.
\end{transition}

\subsubsection{Cas d'une spire -- Induction de Lorentz}

Retrouver l'égalité des puissance mécanique (Laplace) et de l'induction en fonction du couple, de la vitesse de rotation, de la force électromotrice et du courant.

Dire un mot sur la réversibilité de ces machine mais sera développé après.

\begin{transition}
Le couple d'un tel système n'est pas acceptable, il faut utiliser plusieurs spires et des matériaux ferromagnétiques pour optimiser le couplage entre les champs rotoriques et statoriques.
\end{transition}


\subsubsection{Généralisation aux machines réelles}

\begin{slide}
\textbf{Machine à courant continu réelle}
\end{slide}

Introduire la constante de couplage électromécanique et établir l'expression de la vitesse de rotation en fonction de la force contre électromotrice.
Insister sur le fait que les moteurs CC sont facilement contrôlables (vitesse, piles).

\begin{experience}
\textbf{Caractérisque $\omega(E)$ du moteur à courant continu.}
\end{experience}

\begin{transition}
Ces machines sont essentielles mais présentent des inconvénients :
\begin{itemize}
\item pour la production : le transport d'électricité en CC n'est pas rentable, il vaut mieux utiliser le triphasé.
\item pour les moteurs : contacts glissants qui s'abîment, puissances modérées. 
\end{itemize}
\end{transition}

\subsubsection{Machine synchrone}

Avant on avait un moment magnétique variable dans un champ magnétique permanent, maintenant on va voir un moment magnétique constant dans un champ magnétique variable.

\subsubsection{Création d'un champ tournant}
\label{sec:lp09_rotating_field}

Faire le cas avec un champ diphasé pour avoir l'idée puis montrer le cas réel avec 3 bobines.

\begin{remarque}
On peut raisonner par analogie avec les polarisations circulaires qui se décomposent en somme de polarisation linéaires.
\end{remarque}

\begin{slide}
\textbf{Création d'un champ tournant.}
\end{slide}

En plaçant un dipôle au centre, celui-ci sera soumis à un couple qui peut le mettre en rotation.

\begin{experience}
\textbf{Boussole dans un champ magnétique tournant.}
\end{experience}

\begin{transition}
En pratique les champs obtenu grâce à de simples spires n'est pas très intense et le couplage entre les champs statorique et rotorique n'est pas optimal.
Il faut utiliser un milieux ferromagnétique doux pour canaliser le champ magnétique.
\end{transition}

\subsubsection{Champ statorique}

\begin{slide}
\textbf{Modèle de la machine synchrone.}
Parler des hypothèses et insister sur le rôle du matériau ferromagnétique.
\end{slide}

\begin{slide}
\textbf{Champ statorique.}
\end{slide}

Calculer le champs statorique avec le théorème d'Ampère dans les milieux comme dans le bouquin de PSI en faisant attention à bien justifier les symétries, etc.

Aboutir sur l'expression du champ glissant (la donner par analogie avec la partie~\ref{sec:lp09_rotating_field} si manque de temps).

Donner l'expression du champ rotorique par analogie avec le cas d'une seule spire.

\subsection{Couple de la machine synchrone}

\begin{remarque}
Il faut soigner l'introduction du couple comme une dérivée angulaire de l'énergie magnétique.
On peut raisonner par analogie avec la façon dont s'exprime le moment sur un dipôle magnétique, qui vient de l'énergie d'interaction entre le dipôle et le champ. 
\end{remarque}

Suivre le livre de PSI pour déterminer l'expression de l'énergie électromagnétique dans l'entrefer et en déduire l'expression du couple.
Faire apparaitre la condition de synchronisme puis discuter des différents régimes de fonctionnement. 

\begin{slide}
\textbf{Caractéristique $\Gamma(\alpha)$}
\end{slide}

Faire un bilan de puissance comme dans le poly de Jérémy et insister sur les conventions.

Discuter des avantages et inconvénients de la machine synchrone (ouvrir sur les problèmes liés au démarrage et éventuellement montrer la cage d'écureuil)

\subsection*{Conclusion}

Faire un bilan de puissance avec les systèmes mécanique et électrique et montrer le couplage entre les deux par l'induction et la force de Laplace.
Parler des pertes si cela n'a pas été fait dans la leçon.

\newpage