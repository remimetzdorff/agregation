\section{LP14 Ondes acoustiques}

\begin{header}
\begin{tabular}{p{0.4\textwidth} l}
\niveau & \prerequis \\
CPGE    & \textbullet{} Mécanique des fluides \\
        & \textbullet{} Thermodynamique \\
        & \textbullet{} Ondes électromagnétiques \\
\end{tabular}

\noindent
\objectif
Décrire les ondes acoustiques dans différents milieux, leur propagation et ainsi expliquer le principe de fonctionnement de plusieurs instruments de musique. 
\end{header}


\subsection*{Bibliographie}
{
\footnotesize{}
\begin{itemize}
\item \cite{Chaigne2008}
\item \cite{Brebec2004}
\item \cite{Morse1986}
\item \cite{Sanz2016}
\item \cite{Landau}
\item \cite{Metzdorff2017}
\item \cite{Stanford2014}
\end{itemize}
}

\subsection*{Introduction}

Montrer l'extrait de la vidéo~\cite{Stanford2014}.
Dans cette vidéo, on a vu de nombreux exemples qui montrent le caractère vibratoire des ondes acoustiques et leur lien avec le son, la musique.
L'objectif de cette leçon va être de décrire les ondes acoustiques, principalement dans les fluides et de voir comment leur manipulation peut conduire à la fabrication d'instruments, mais aussi à la compréhension du comportement de nombreux objets.

\subsection{Description d'une onde acoustique}

Les ondes acoustiques sont des ondes mécaniques.
Elles correspondent à la propagation d'une déformation dans un milieu matériel.
Insister sur la nécessité d'un milieu matériel, qui peut être fluide ou solide.
Ici on va principalement s'intéresser aux ondes acoustiques dans l'air.

\subsubsection{Approximation acoustique}

Les ondes acoustiques résultent d'un couplage entre des variations de pression et le déplacement des particules de fluide.
On va donc s'intéresser à ces deux grandeurs principalement.
Cependant, le fluide est compressible et il va aussi y avoir des variations de volume donc de masse volumique.
D'autres grandeurs (température, etc.) sont également amenées à varier ce qui va conduire à effectuer certaine hypothèse, que l'on pourra vérifier ensuite.

Dans un premier temps, on s'intéresse à un fluide au repos :
\begin{itemize}
\item de vitesse moyenne nulle ;
\item de pression moyenne $P_0$ ;
\item de masse volumique moyenne $\mu_0$.
\end{itemize}

L'onde sonore correspond à une faible perturbation du fluide par rapport à cet état de repos :
\begin{itemize}
\item $\overrightarrow{v}(M, t) = \overrightarrow{v_1}(M, t)$, petite devant la vitesse du son $c=\lambda\nu$ ;
\item $P(M, t) = P_0 + p_1(M, t)$, où $p_1(M, t) \ll P_0$ ;
\item $\mu(M, t) = \mu_0 + \mu_1(M,t)$ où $\mu_1(M, t) \ll \mu_0$;
\end{itemize}
et sera traité comme tel.
On négligera ainsi tous les termes d'ordre deux dans les équations.
C'est l'approximation acoustique.

Dans le cadre de cette leçon, on considère l'écoulement comme parfait en négligeant la viscosité du fluide.
Ceci conduit à une déformation élastique rapide du fluide, c'est à dire réversible, ce qui nous permettra de formuler une hypothèse thermodynamique.

\begin{transition}
On vient de définir le cadre de l'étude des ondes acoustiques dans un fluide.
On peut maintenant déterminer l'équation qui régit l'évolution de ces ondes en exploitant les outils de la mécanique des fluides et de la thermodynamique.
\end{transition}

\subsubsection{Équation de propagation}

On peut tout d'abord utiliser l'équation de la conservation locale de la masse :
\begin{equation*}
\frac{\partial \mu}{\partial t} + \div(\mu\overrightarrow{v}) = 0,
\end{equation*}
qui conduit après linéarisation à
\begin{equation*}
\frac{\partial \mu_1}{\partial t} + \mu_0\div\overrightarrow{v_1} = 0.
\end{equation*}
Cette équation fait apparaitre un premier lien entre $\mu_1$ et $\overrightarrow{v_1}$ alors qu'on préfèrerai un lien entre $p_1$ et $\overrightarrow{v_1}$.

On peut faire ce lien à travers un coefficient thermodynamique.
La transformation associée au passage de l'onde est rapide donc on la supposera adiabatique et réversible, c'est à dire isentropique.
Dans ce cas on utilise le coefficient de compressibilité isentropique $\chi_S$
\begin{equation*}
\chi_S = -\frac{1}{V} \left( \frac{\partial V}{\partial P} \right)_S = \frac{1}{\mu} \left( \frac{\partial \mu}{\partial V} \right)_S.
\end{equation*}
Un développement de Taylor donne ainsi la relation $\mu_1 = \chi_S\mu_0 p_1$.
En l'injectant dans l'équation de conservation de la masse, on obtient après linéarisation
\begin{equation}
\chi_S \frac{\partial p_1}{\partial t} + \div  \overrightarrow{v_1} = 0.
\label{eq:mass_conservation}
\end{equation}

L'écoulement étant parfait, on utilise l'équation d'Euler en négligeant la gravité :
\begin{equation*}
\mu \left( \frac{\partial \overrightarrow{v}}{\partial t} + \left( \overrightarrow{v}\grad\right)\overrightarrow{v}\right) = -\grad P.
\end{equation*}
L'hypothèse $v_1$ petite conduit à négliger le terme non linéaire de l'équation d'Euler
\begin{equation*}
\left|\left|\left( \overrightarrow{v}\grad\right)\overrightarrow{v}\right|\right| \ll \left|\left|\frac{\partial\overrightarrow{v}}{\partial t}\right|\right|
\end{equation*}
ce qui est vrai si $||v_1|| \ll c$, où $c=\lambda\nu$ est la vitesse de l'onde acoustique.
Cette condition peut être vérifiée à posteriori.
Avec ces hypothèses, on aboutit à l'équation linéarisée
\begin{equation}
\mu_0 \frac{\partial \overrightarrow{v_1}}{\partial t} = -\grad p_1.
\label{eq:euler}
\end{equation}

En dérivant l'équation de conservation de la masse~\ref{eq:mass_conservation} par rapport au temps et en prenant la divergence de l'équation d'Euler~\ref{eq:euler}, on obtient l'équation de d'Alembert pour la surpression $p_1$
\begin{equation}
\frac{\partial^2 p_1}{\partial t^2} - \frac{1}{c^2}\Delta p_1 = 0,
\label{eq:dalembert_p}
\end{equation}
où $c=1/\sqrt{\chi_S \mu_0}$ est la vitesse de l'onde acoustique.
De même, en dérivant Euler par rapport au temps et en prenant le gradient de la conservation de la masse, on obtient l'équation de d'Alembert pour la vitesse $\overrightarrow{v_1}$
\begin{equation}
\frac{\partial^2 \overrightarrow{v_1}}{\partial t^2} - \frac{1}{c^2}\Delta \overrightarrow{v_1} = 0.
\label{eq:dalembert_v}
\end{equation}
\begin{remarque}
Pour l'équation de d'Alembert sur la vitesse, il faut de plus supposer l'écoulement irrotationnel, ce qui est raisonnable dans l'hypothèse d'un écoulement parfait et en appliquant le théorème de Kelvin.
\end{remarque}

\begin{transition}
Pour établir ces équations de propagation, on a fait plusieurs hypothèses qu'il faut vérifier.
\end{transition}

\subsubsection{Retour sur les hypothèses}

\begin{slide}
\textbf{Quelques ordres de grandeur.}
L'intensité sera définie proprement ensuite.
Même pour des sons très intenses, les hypothèses de perturbations faibles sont vérifiées, donc l'approximation acoustique est valide.
\end{slide}

La deuxième hypothèse réalisée est celle d'une transformation adiabatique réversible.
Pour une évolution isentropique du fluide, on utilise la loi de Laplace $PV^\gamma = \mathrm{cte}$, où $\gamma=c_p/c_v$ est le rapport des capacité calorifique à pression et volume constant.
On trouve ainsi $\chi_S = 1/\gamma P_0$ et donc
\begin{equation*}
c = \sqrt{\frac{\gamma P_0}{\mu_0}}.
\end{equation*}
Si le fluide peut-être considéré comme un gaz parfait, on obtient finalement en utilisant l'équation d'état des gaz parfaits
\begin{equation}
c = \sqrt{\frac{\gamma RT_0}{M}}.
\end{equation}
\begin{remarque}
Le plus simple pour redémontrer cette relation est de différencier $\ln(PV^\gamma)$ pour exprimer la compressibilité isentropique. 
\end{remarque}

On peut vérifier expérimentalement cette relation en mesurant la vitesse du son dans l'air.
\begin{experience}
\textbf{Mesure de la vitesse du son dans l'air avec une onde ultra-sonore.}
On pourrait remonter à cette vitesse en mesurant le temps de vol d'une impulsion brève entre un émetteur et un récepteur ultra-sonore mais cette mesure est sujette à une incertitude importante car on ne connait pas exactement leur géométrie.
Je préfère ici mesurer la longueur d'onde en déplaçant de plusieurs longueur d'onde le récepteur devant l'émetteur sur un banc optique.
\end{experience}

\begin{remarque}
L'air n'est pas un milieu dispersif pour les ondes acoustiques.
\end{remarque}
\begin{remarque}
Pour une transformation isotherme, on utilise le coefficient de compressibilité isotherme $\chi_T$ et on trouve
\begin{equation*}
c = \sqrt{\frac{RT_0}{M}},
\end{equation*}
ce qui n'est pas en accord avec les observations expérimentales.
\end{remarque}

\begin{transition}
Les résultats que l'on a obtenu jusqu'à maintenant semblent expliquer convenablement les observations expérimentales.
Montrer la vidéo~\cite{Metzdorff2017}.
Cette vidéo met en évidence que les ondes sonores transportent de l'énergie.
\end{transition}

\subsection{Aspects énergétiques}

Dans cette partie, on fait directement le parallèle avec les résultats obtenus pour les ondes électromagnétiques.

\subsubsection{Conservation de l'énergie}

La puissance $\d\cal{P}$ transférée par l'onde acoustique à travers une surface orientée $\overrightarrow{\d S}$ correspond à la puissance des forces de pression soit
\begin{equation*}
\d {\cal{P}} = (P_0+p_1) \overrightarrow{\d S} \cdot \overrightarrow{v_1}.
\end{equation*}
Comme $P_0$ est constante, elle donnera avec $\overrightarrow{v_1}$ un terme de moyenne temporelle nulle qu'il n'est pas nécessaire de considérer.
On peut ainsi définir le vecteur de Poynting sonore $\overrightarrow{\Pi}$
\begin{equation}
\overrightarrow{\Pi} = p_1\overrightarrow{v_1},
\end{equation}
qui correspond aux transferts d'énergie dû à la surpression donc aux ondes acoustiques.

Par ailleurs on souhaite exprimer la densité d'énergie du milieu liée au passage de l'onde acoustique.
Comme il s'agit d'une onde de vitesse et de pression, on retrouve deux contributions :
\begin{itemize}
\item cinétique $e_c$ liée à la vitesse $\overrightarrow{v_1}$
\begin{equation*}
e_c = \frac{1}{2}\mu_0 v_1^2 ;
\end{equation*}
\item potentielle $e_p$ lié à la compression du fluide et analogue à l'énergie potentielle d'un ressort comprimé
\begin{equation*}
e_p = \frac{1}{2} \chi_S p_1^2.
\end{equation*}
\end{itemize}
La densité volumique d'énergie associée à l'onde acoustique $e$ est donc
\begin{equation}
e = \frac{1}{2}\mu_0v_1^2 + \frac{1}{2}\chi_S p_1^2.
\end{equation}

Le cadre de la description des ondes acoustiques nous a conduit à négliger les phénomènes dissipatifs.
Au niveau local, une variation d'énergie ne peut être due qu'à son transport par l'intermédiaire des forces de pression, si bien qu'on peut retrouver l'équation locale de conservation de l'énergie
\begin{equation}
\frac{\partial e}{\partial t} + \div\overrightarrow{\Pi} = 0.
\end{equation}
Dans ce modèle, une onde plane n'est pas atténuée et l'atténuation d'une onde sphérique n'est due qu'à un facteur géométrique de dilution dans l'espace.
\begin{remarque}
Pour une onde plane progressive harmonique, $e_c = e_p$.
\end{remarque}

\begin{transition}
Les flux de puissance dûs aux ondes acoustiques sont généralement très faibles, si bien qu'il est souvent utile d'utiliser l'intensité acoustique. 
\end{transition}

\subsubsection{Intensité d'une onde acoustique}

L'intensité sonore $I$ est définie comme la moyenne temporelle de la puissance reçue par unité de surface, soit en utilisant le vecteur de Poynting
\begin{equation*}
I = \left\langle \overrightarrow{\Pi}\cdot\overrightarrow{n} \right\rangle.
\end{equation*}
Pour plus de commodité, il est d'usage de l'exprimer en décibel (dB)
\begin{equation}
I_\mathrm{dB} = 10\log\frac{I}{I_0},
\end{equation}
avec $I_0 = \unit{10^{-12}}{\watt\cdot\meter^{-2}}$ qui correspond au seuil d'audibilité.

\begin{slide}
\textbf{Audition humaine.}
On entend bien les sons entre \unit{20}{\hertz} et \unit{20}{\kilo\hertz}.
L'oreille est très sensible à une grande diversité d'intensité sonores ce qui justifie l'utilisation d'une échelle logarithmique.

Ici c'est propre à l'Homme mais certains animaux sont capables de produire et percevoir des infra-sons (éléphant, girafe) et ultra-sons (cétacés).
\end{slide}

\begin{transition}
Terminons le parallèle avec l'électromagnétisme en s'intéressant à la notion d'impédance acoustique, qui exprime un lien simple entre la vitesse du fluide et la surpression.
\end{transition}

\subsubsection{Impédance acoustique}

Ici on s'intéresse à un type de solutions particulières de l'équation de d'Alembert : les ondes planes progressives harmoniques de la forme
\begin{equation*}
p_1 = P_0 \cos \left( \omega t - \overrightarrow{k}.\overrightarrow{r} + \varphi \right),
\end{equation*}
analogues aux OPPH électromagnétiques, où $\overrightarrow{k}=2\pi/\lambda$ est dans la direction de propagation de l'onde.
Comme les équations qui décrivent le phénomène sont linéaires, on peut utiliser la notation complexe
\begin{equation}
\underline{p_1} = \underline{P_0} e^{i\left(\omega t - \overrightarrow{k}.\overrightarrow{r}\right)}.
\end{equation}

En utilisant l'équation d'Euler~\ref{eq:euler}, on trouve
\begin{equation*}
\mu_0 i\omega \underline{\overrightarrow{v_1}} = i\overrightarrow{k}\underline{p_1},
\end{equation*}
d'où
\begin{equation}
\underline{\overrightarrow{v_1}} = \frac{1}{\mu_0 c} \underline{p_1} \overrightarrow{n}.
\end{equation}

Ce lien entre la vitesse et la surpression peut être exprimé comme en électromagnétisme à l'aide de l'impédance acoustique du milieu $Z$ définie comme
\begin{equation}
Z = \frac{p_1}{v_1},
\end{equation}
exprimé en $\kilogram \cdot \meter^{-2} \cdot s^{-2}$.
Dans le cas d'une onde plane progressive harmonique, ce rapport vaut
\begin{equation}
Z = \mu_0 c = \sqrt{\frac{\mu_0}{\chi_S}}.
\end{equation}
Plus la masse volumique du fluide est grande et plus la compressibilité du fluide est faible, plus l'impédance est grande, d'où
\begin{equation*}
Z_\mathrm{solide} > Z_\mathrm{liquide} \gg Z_\mathrm{gaz}.
\end{equation*}
\begin{remarque}
L'impédance électromagnétique est définie comme
\begin{equation*}
Z = \frac{E}{H} = \frac{\mu}{\epsilon}.
\end{equation*}
\end{remarque}

De la même façon qu'en électromagnétisme, la propagation d'une onde acoustique à travers un dioptre donne naissance à une onde réfléchie et une onde transmise.
\begin{slide}
\textbf{Réflexion et transmission sur un dioptre.}
On voit qu'un changement brutal et important d'impédance ($Z\rightarrow 0$ ou $Z\rightarrow \infty$) conduit à une réflexion totale de l'onde acoustique incidente.
On peut parler du gel pour l'échographie, de l'écho contre un mur, etc.
\end{slide}

\begin{transition}
Des réflexions multiples peuvent conduire à l'établissement d'ondes stationnaires.
Leur maitrise permet de fabriquer des cavités résonantes en vue de réaliser des instruments de musique par exemple.
\end{transition}

\subsection{Quelques exemples}

\subsubsection{Tuyau sonore}

Schéma du tuyau.
On s'intéresse à une onde acoustique se propageant dans le tuyau.
Les extrémités du tuyau imposent des conditions aux limites :
\begin{itemize}
\item une extrémité ouverte impose que la pression doit être $P_0$, mais ne donne pas de restriction sur la vitesse ;
\item à l'inverse, une extrémité fermée impose une vitesse nulle (impénétrabilité) mais rien sur la pression.
\end{itemize}
Les extrémités du tuyau correspondent donc à des sauts d'impédance
\begin{equation}
Z_\mathrm{fermé} = \infty \quad , \quad Z_\mathrm{ouvert} = 0,
\end{equation}
qui vont donner lieux à des réflexions totales.
Dans le tuyaux, on a donc une superposition d'ondes contra-propageantes qui donne naissance à une onde stationnaire de la forme
\begin{equation}
p_1 = P_0 \cos(\omega t) \left[ A\cos(kx) + B\sin(kx) \right].
\end{equation}
Pour un tuyau ouvert aux deux extrémités, on trouve par exemple $A=0$ et
\begin{equation}
k_n = n\frac{\pi}{L}.
\end{equation}
\begin{slide}
\textbf{Mode d'un tuyau sonore.} Comparaison avec un tuyau fermé à une extrémité
\end{slide}
\begin{slide}
\textbf{Orgue de la cathédrale Saint Étienne.}
\end{slide}
\begin{slide}
\textbf{Micro-pilier.} Le micro-pilier est un analogue solide du tuyau sonore.
On retrouve les mêmes conditions aux limites, mais il faut utiliser le module d'Young.
Bien détailler ce qu'on voit à l'écran : la membrane permet de tenir le pilier etc.
\end{slide}

\subsubsection{Plaque vibrante}

\begin{slide}
\textbf{Plaque de Chladni.}
Présentation rapide du système, mêmes équations qui interviennent, modes obtenus en tenant compte des conditions aux limites, etc.
Cette fois-ci il s'agit d'ondes transverses et pas longitudinales.
\end{slide}

\subsection*{Conclusion}

\begin{slide}
Récapitulatif et ouverture avec l'exemple du train :
\begin{itemize}
\item onde dans la caténaires (il a fallu retendre les caténaire lors des records de vitesse du TGV) ;
\item ambiance sonore dans les cabines ;
\item propagation des vibrations dans les structures.
\end{itemize}
\end{slide}

\newpage