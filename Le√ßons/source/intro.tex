\chapter*{Introduction}
\addcontentsline{toc}{chapter}{Introduction}

\section*{Code couleur}

\begin{header}
Cadre de la leçon.
\end{header}

\begin{experience}
\textbf{Expérience.}
Les manips et l'expérience quantitative propre à l'agrégation externe spéciale.
\end{experience}

\begin{slide}
\textbf{Slide.}
Le contenu à projeter à l'écran : slides, vidéos, ressources internet, etc.
\end{slide}

\begin{transition}
Les transitions indispensables à la fluidité du discours !
\end{transition}

\begin{funfact}
Fun facts : moins important que les remarques mais quand même.
\end{funfact}

\begin{remarque}
Les remarques sur des points non essentiels mais qu'il est bon de grader en tête.
\end{remarque}

Les leçons LC01 à LC09 sont placées au niveau lycée.
Les leçons LC10 à LC19 sont placées en CPGE.