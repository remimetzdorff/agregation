\section{LC08 Molécules de la santé}

\begin{header}
\begin{tabular}{p{0.4\textwidth} l}
\niveau & \prerequis \\
Lycée   & \textbullet{} Synthèse organique \\
        & \textbullet{} Méthodes de caractérisation \\
        & \textbullet{} Réaction d'oxydoréduction \\
        & \textbullet{} Dosages \\
        & \textbullet{} Spectroscopie RMN
\end{tabular}

\noindent
\objectif
Le but de cette leçon est de voir quelles sont les molécules de la santé, de comprendre comment elles agissent et enfin de découvrir quelques méthodes d'obtention de principes actifs.
\end{header}

{
\subsection*{Bibliographie}
\footnotesize{}
\begin{itemize}
\item \cite{Bataille2010}
\item \cite{Prevost2017}
\item \cite{Azan2011}
\item \cite{Dulaurans2012}
\item \cite{Mesplede2002}
\item \href{https://www.pourquoidocteur.fr/Articles/Question-d-actu/24371-Levothyrox-nombre-inattendu-d-effets-secondaires-deja-connus-Elements-d-explication}{Levothyrox}
\item \href{http://www.guidepharmasante.fr/chiffres-cles/les-chiffres-cles-du-marche-du-medicament-1}{Les chiffres-clés du marché du médicament}
\end{itemize}

}

\paragraph{Expériences :}
\begin{itemize}
\item Synthèse du paracétamol \cite{Mesplede2002}, p145 ;
\item Dosage du diiode de la bétadine \cite{Dulaurans2012}, p468 ;
\item Solubilité de différentes formulation de l'aspirine \cite{Bataille2010}, p117 ;
\item Catalyse de la dismutation de $\mathrm{H_2O_2}$.
\item Extraction de l'eugénol.
\end{itemize}

\subsection*{Introduction}

Les progrès de la médecine ont permis de rallonger considérablement notre espérance de vie.
Jusqu'au $\mathrm{XVIII^{eme}}$ siècle, on se contentait essentiellement de ce que la nature pouvait apporter, mais à partir du $\mathrm{XIX^{eme}}$ siècle, les connaissances en chimie ont permis d'améliorer les substances utilisées.
La chimie est ainsi réellement au cœur de ces développements comme nous allons le voir dans cette leçon.
Actuellement l'industrie pharmaceutique est le sixième marché économique mondial derrière le pétrole, la nourriture, et les trafics de stupéfiant, d'arme et d'être humain.

L'objectif sera de voir quelles sont les molécules de la santé et quels sont les procédés d'obtention de ces composés.
Nous verrons aussi le mode d'actions de certains de ces composés pour comprendre quels peuvent être leurs effets.

\begin{remarque}
\href{http://www.guidepharmasante.fr/chiffres-cles/les-chiffres-cles-du-marche-du-medicament-1}{Les chiffres-clés du marché du médicament en France} :
\begin{itemize}
\item la France est au cinquième rang des marchés pharmaceutiques ;
\item 8500 embauches par an ;
\item $54{,}1$ milliards d'euros de chiffre d'affaire ;
\item 510 \euro{} pour la consommation moyenne par habitant.
\end{itemize}
\end{remarque}

\subsection{La chimie au service de la santé}

\subsubsection{Action thérapeutique : les médicaments}

\begin{slide}
\textbf{Le paracétamol.}
Introduire les différentes définitions avec l'extrait de la notice.
\end{slide}

Toutes ces définitions sont tirées de \cite{Prevost2017}, p35.

La définition du mot médicament est fixée par une loi du 26/02/07 : \og On entend par médicament toute substance ou composition présentée comme possédant des propriétés curatives ou préventives à l'égard des maladies humaines ou animales [...] \fg{}

Un médicament contient au moins une substance active, appelée principe actif, connue pour prévenir ou guérir une maladie.

Les autres constituants d'un médicament sont appelés excipients.
Ils servent à donner sa forme, son aspect, son goût mais aussi souvent à faciliter l'assimilation du principe actif.

\begin{slide}
\textbf{Développement d'un médicament.}
Le brevet donne à celui qui le dépose une exclusivité de 20 ans sur l'exploitation du principe actif.
Il faut entre 10 et 15 ans pour que le médicament arrive sur le marché ce qui donne entre 5 et 10 ans d'exclusivité au dépositaire du brevet pour la commercialisation du princeps avant que les génériques ne soit accessibles.
On compte en général entre 8 et 10 ans de recherches et entre 1 et 3 ans pour l'autorisation de mise sur le marche (AMM).
\end{slide}

Pour un même principe actif, il existe souvent différentes formes d'assimilation appelées formes galéniques.
La formulation du médicament est choisie en vue d'une meilleure assimilation du principe actif.
Elle dépend principalement des excipients.

\begin{slide}
\textbf{pH du système digestif.}
\end{slide}

\begin{experience}
\textbf{Solubilité de différentes formulations d'aspirine à différents pH.}
\cite{Bataille2010}, p117.
\end{experience}

L'exemple de l'aspirine est assez banal mais il existe des cas où des changements de formulation ont eu des conséquences importantes.
C'est le cas du \href{https://www.pourquoidocteur.fr/Articles/Question-d-actu/24371-Levothyrox-nombre-inattendu-d-effets-secondaires-deja-connus-Elements-d-explication}{Levothyrox} dont un changement de la composition des excipients a entrainé une augmentation de la fréquence d'effets secondaires insupportables selon les patients.
Ce médicament contient une hormone thyroïdienne et est prescrit dans le cas d'une déficience en thyroxine naturelle.

\begin{remarque}
Le principe d'action du paracétamol n'est pas parfaitement connu mais, comme l'aspirine, il agirait en inhibant au niveau du système nerveux central la production de prostaglandines.
Ce sont des métabolites impliqués dans les processus de la douleur et de la fièvre.
L'aspirine agit sur l'hypothalamus, thermostat de la température corporelle.
\end{remarque}

\begin{transition}
L'apport de la chimie à la santé ne se limite pas seulement au développement de médicaments.
On utilise souvent des substances destinées à l'assainissement.
\end{transition}

\subsubsection{Hygiène : antiseptiques et désinfectants}

Définitions tirées de \cite{Azan2011}, p128.

Il s'agit de composés chimiques qui éliminent certains micro-organismes (virus, bactéries, champignons, spores), ou du moins qui ralentissent leur prolifération.
Ils agissent par oxydation.
On distingue les antiseptiques, qui empêchent la prolifération de ces germes dans les tissus vivants ou à leur surface, des désinfectants qui eux, tuent les germes présents en dehors de l'organisme :
\begin{itemize}
\item antiseptique : liquide de Dakin ($\mathrm{ClO^-, MnO_4^-}$), Bétadine ($\mathrm{I_2}$), eau oxygénée ($\mathrm{H_2O_2}$) ;
\item désinfectant : eau de Javel ($\mathrm{ClO^-}$).
\end{itemize}

\begin{remarque}
Le diiode est obtenu par réduction par le dioxyde de souffre des ions iodate $\mathrm{IO_3^-}$ contenus dans le minerai de caliche, sous forme d'iodate de calcium.
Il peut être obtenu par oxydation de ions iodure issus des saumures extraites lors de l'exploitation de puits de pétrole.

\noindent
Les ions hypochlorite sont obtenus à partir de la dismutation du dichlore dans la soude, lui même issu du procédé chlore soude (cf LC04).

\noindent
Le peroxyde d'hydrogène est obtenu avec le procédé Riedl-Pfeiderer (1936) par barbotage d'air comprimé dans un dérivé d'anthraquinone.
\end{remarque}

\begin{experience}
\textbf{Propriétés oxydantes du diiode.}
\cite{Dulaurans2012}, p468.
Montrer la décoloration d'une solution de diiode (Bétadine) par une solution de thiosulfate de sodium ($\mathrm{S_2O_3^{2-}}$).
\begin{equation}
\mathrm{I_2 + 2S_2O_3^{2-} = 2I^- + S_4O_6^{2-}}
\end{equation}
Comme la leçon ne contient pas beaucoup de réaction, c'est peut-être le bon moment pour écrire proprement l'équation d'oxydoréduction.
Les produits de la réaction sont les ions iodure et les ions tétrathionate. 
\end{experience}

\begin{experience}
\textbf{Catalyse de la dismutation du peroxyde d'hydrogène par les ions Fe(II).}
$\mathrm{H_2O_2}$ appartient à deux couples redox :
\begin{itemize}
\item $\mathrm{H_2O_2/H_2O}$ ($E^0 = \unit{1{,}78}{\volt}$) : $\mathrm{H_2O_2 + 2H^+ + 2e^- = 2H_2O}$ ;
\item $\mathrm{O_2/H_2O_2}$ ($E^0 = \unit{0{,}697}{\volt}$) : $\mathrm{O_2 + 2H^+ + 2e^- = H_2O_2}$.
\end{itemize}
La réaction de dismutation est thermodynamiquement favorable mais cinétiquement lente.
Elle est catalysée par les ions $\mathrm{Fe^{2+}}$.
En plus de son action oxydante, le dégagement gazeux rapide en présence d'un catalyseur (Fe(II) contenu dans l'hémoglobine, enzymes), l'eau oxygénée a une action mécanique pour le nettoiement des plaies.
\end{experience}

\begin{remarque}
Les désinfectants agissent souvent par dénaturation des protéines.
L'éthanol dénature les protéines cytoplasmiques et membranaires, et inhibe la synthèse  des acides nucléiques et des protéines.
Les oxydants produisent des radicaux libres qui interagissent avec les lipides, les protéines et l'ADN.
\end{remarque}

\begin{transition}
Il existe de nombreuses façons d'obtenir le principe actif d'un médicament.
\end{transition}

\subsection{Obtention du principe actif}

\subsubsection{Extraction de principes actifs}

C'est la façon la plus simple d'obtenir un principe actif, parce qu'elle exploite directement les composés présents dans la nature.
C'est donc la première méthode employée par l'homme : l'acide salicylique est ainsi employé depuis l'antiquité en l'extrayant de l'écorce de saule blanc.
La seule difficulté est d'isoler le produit.
\begin{slide}
\textbf{De multiples méthodes d'extraction.}
On peut citer :
\begin{itemize}
\item expression, macération, infusion, décoction ;
\item l'hydrodistillation pour l'obtention d'huile essentielles, avec l'exemple de la lavande ;
\item extraction par solvant \cite{Prevost2017} : elle est réalisée en solubilisant l'espèce chimique à extraire dans un solvant. 
\end{itemize}
\end{slide}

\begin{slide}
\textbf{Extraction de l'eugénol.}
\cite{Prevost2017}.
\end{slide}

\begin{experience}
\textbf{Extraction de l'eugénol dans l'éther.}
En préparation, on aura réalisé une décoction ou une hydrostillation de clous de girofle.
Pendant la leçon on présente la phase d'extraction liquide-liquide de l'eugénol dans l'éther.
\end{experience}

\begin{transition}
L'extraction de principes actifs disponibles dans la nature présente plusieurs limitations :
\begin{itemize}
\item on est limité aux composé produits par la nature ;
\item il peut être difficile de s'approvisionner en matière première.
\end{itemize}
Pour améliorer l'efficacité des principes actifs il est souvent nécessaire de transformer des molécules pour en synthétiser de nouvelles.
\end{transition}

\subsubsection{Synthèse du paracétamol}

Le paracétamol est obtenu par addition nucléophile à partir du 4-aminophénol (ou paraaminophénol ou 4-hydroxyaniline) et de l'anhydride acétique.
La réaction produit aussi de l'acide acétique, utilisé comme solvant dans cette synthèse.

\begin{slide}
\textbf{Synthèse du paracétamol -- Équation de réaction.}
\end{slide}

\begin{remarque}
Le paraaminophénol est obtenu par nitration du phénol.
Le phénol est obtenu grâce au procédé au cumène (Hock 1944), à partir de benzène, de propylène et du dioxygène de l'air.
Les réactifs alors dérivés de la pétrochimie (le cumène peut être lui même obtenu de la pétrochimie) et le procédé forme aussi de l'acétone.
\end{remarque}

\begin{experience}
\textbf{Synthèse du paracétamol.}
\cite{Mesplede2002}, p145.
Lancer le goutte à goutte d'anhydride acétique au début de la leçon et mettre dans la glace au début de la deuxième partie.
On présente la phase d'essorage sur buchner.
Les contrôles seront effectués plus tard.
\end{experience}

\begin{slide}
\textbf{Synthèse du paracétamol -- Montage.}
\end{slide}

\begin{transition}
On souhaite vérifier que le produit synthétisé est le bon et qu'il est pur.
Dans cette synthèse par exemple le paraaminophénol est toxique et cancérigène.
Pour éviter des problèmes sanitaires, il est importants d'effectuer des contrôles qualité. 
\end{transition}

\subsection{Contrôle qualité}

\subsubsection{Identification, vérification de la pureté}

En préparation, on a mesuré la température de fusion du produit obtenu après synthèse et séchage.
La température obtenue est plus faible que la température tabulée ce qui indique la présence d'impuretés.
Une recristallisation a été réalisée pour purifier le produit.

\begin{experience}
\textbf{Contrôle du paracétamol synthétisé.}
Montrer une plaque CCM réalisée en préparation avec produit synthétisé, paracétamol commercial, paraaminophénol et codépôt.
Mesure de la température de fusion du produit recristallisé (impossible d'utiliser le produit récupéré à la phase précédente car il nécessite un séchage).
On peut aussi acquérir le spectre IR du produit et le comparer aux spectres tabulés ou du produit commercial.
\end{experience}

\begin{transition}
La pureté n'est pas le seul critère qui importe.
Le dosage est aussi fondamental pour éviter des erreurs de posologie.
\end{transition}

\subsubsection{Dosage}

On souhaite vérifier l'information données par le fabricant sur la concentration en diiode dans une solution commerciale de diiode.

\begin{experience}
\textbf{Dosage du diiode contenu dans la Bétadine par les ions thiosulfate.}
\cite{Dulaurans2012}, p468.
Faire le dosage complet à partir des solutions réalisées en préparation.
Le suivi de l'avancement se fait par colorimétrie.
\end{experience}

\begin{slide}
\textbf{Dosage du diiode de la Bétadine.}
\end{slide}


\subsection*{Conclusion}

\begin{slide}
\textbf{La chimie au service de la santé.}
\end{slide}

On peut ouvrir sur l'importance de la configuration spatiale des molécules avec l'exemple de la thalidomide commercialisée dans les années 1950 :
\begin{itemize}
\item la forme (R) protège contre les nausées, les tumeurs et les syndromes inflammatoires ;
\item la forme (S) est tératogène (source de malformation fœtales).
\end{itemize}

\newpage