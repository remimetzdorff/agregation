\section{LP12 Traitement d'un signal, étude spectrale}

\begin{header}
\begin{tabular}{p{0.4\textwidth} l}
\niveau & \prerequis \\
CPGE (MP, PSI) & \textbullet{} Electrocinétique, fonction de transfert \\
			   & \textbullet{} Filtre RC
\end{tabular}

\noindent
\objectif
Mettre en évidence l'importance du traitement de signal dans les protocoles de communication actuels.
Discuter des avantages et inconvénients des traitements numériques et analogiques.
Se familiariser avec l'étude spectrale.
\end{header}

{
\footnotesize{}
\subsection*{Bibliographie}
\begin{itemize}
\item \cite{Augier2014}
\item \cite{Cardini2017}
\item \cite{Salamito2017}
\item \cite{Neveu2019}

\end{itemize}
}

\begin{remarque}
Cette leçon se prête particulièrement bien à des expériences qualitatives rapides sur l'oscilloscope, aussi bien dans le filtrage avec l'étude de l'action du RC sur un créneau par exemple que pour montrer le traitement numérique.
Il faut bien reprendre en main l'oscilloscope, notamment sur la FFT pour montrer rapidement les effets mentionnés dans la leçon. 
\end{remarque}

\subsection*{Introduction}

Dans le cadre des télécommunications, on s'intéresse à la transmission d'un signal complexe qui doit être traité afin d'être transporté sur de longues distances.
On distingue les méthodes analogiques qui conservent la nature continue du signal des méthodes numériques qui nécessite sa discrétisation.
Ces deux méthodes présentent chacune des avantages et inconvénients que l'on discutera au cours de cette leçon.
L'étude spectrale permet de s'affranchir de la redondance temporelle de la plupart des signaux et constitue donc un outils puissant pour leur étude.

\subsection{Spectre d'un signal et filtrage}

\begin{remarque}
L'objectif de cette partie est de justifier le passage à l'étude de spectre et de comprendre comment agit un système simple (filtre linéaire) sur le signal.
Par la suite il est possible d'étudier directement les spectres.
\end{remarque}

Le signal est une grandeur physique dons les variations temporelles encodent une information.

\subsubsection{Signal périodique}

Suivre \cite{Salamito2017}.
Tout signal périodique $s(t)$ de fréquence $f_s$ s'exprime comme la superposition de fonctions sinusoïdales de fréquence multiple de $f_s$ :
\begin{equation}
s(t) = A_0 + \sum_{n=1}^\infty A_n \cos(2\pi n f_s t + \varphi_n).
\end{equation}
Toutes les informations sur $s(t)$ sont contenues dans les $A_n$ et $\varphi_n$ : on peut les représenter en fonction de la fréquence sous forme de spectres en amplitude et en phase.
Faire les schémas $s(t)$, $A_n(f)$ et $\varphi_n(f)$.

\begin{slide}
\textbf{Spectre d'un signal en créneau.}
\end{slide}

\begin{experience}
\textbf{FFT d'un signal en créneau sur l'oscilloscope.}
\end{experience}

Donner les interprétations physiques de $A_0$, $A_1$ et $A_n$ : les premières harmoniques décrivent la forme générale du signal mais les détails sont donnés par les hautes fréquences, c'est-à-dire par les harmoniques de rang élevé.

\begin{slide}
\textbf{Spectre du signal sonore émis par un verre.}
Parler de la transformée de Fourier pour les signaux non périodiques. 
\end{slide}

\begin{transition}
Cette description permet de s'affranchir de la redondance d'un signal périodique et est particulièrement utile pour décrire la réponse des filtres linéaires.
\end{transition}

\subsubsection{Filtrage linéaire}

\begin{remarque}
Attention à la cohérence dans cette partie avec les pré-requis : si les filtres électroniques ont déjà été vus, il n'est pas utile d'introduire les notions de diagramme de Bode etc.
Il faut plutôt je pense rattacher la description générale au cas particulier des filtres électroniques.
\end{remarque}

Faire une schéma d'un filtre avec $e$ et $s$.
Définition d'un filtre : système pour lequel il existe une relation à coefficients constants entre les dérivées temporelles de l'entrée et celles de la sortie :
\begin{equation}
b_0 s(t) + b_1 \frac{\d s}{\d t} + ... + b_m \frac{\d^m s}{\d t^m} = a_0 e(t) + a_1 \frac{\d e}{\d t}+ ...
\end{equation}
$m$ est appelé l'ordre du filtre.
Ce type de système est linéaire ce qui permet d'utiliser le principe de superposition.
Il permet d'agir sur le spectre d'un signal sans modifier le contenu fréquentiel (ne fait pas apparaitre de nouvelle fréquences, ce qui traduirait une non-linéarité).

On suppose un signal d'entrée de la forme :
\begin{equation}
e(t) = E\cos(\omega t+\varphi_e)
\end{equation}
et on utilise la notation complexe :
\begin{equation}
\underline{e}(t) = E e^{j(\omega t + \varphi_e)}.
\end{equation}
En faisant de même pour la sortie, on définit la fonction de transfert d'un filtre linéaire comme le rapport :
\begin{equation}
\underline{H}(j\omega) = \frac{\underline{s}}{\underline{e}}.
\end{equation}

Calculs du filtre RC en utilisant les impédances complexes.
On obtiendrait le même résultat en établissant l'équation temporelle du circuit.

\begin{slide}
\textbf{Filtre passe-bas.}
\end{slide}

\begin{experience}
\textbf{Diagramme de Bode d'un circuit RC.}
Montrer l'effet du filtre sur un signal sinusoïdal en modifiant la fréquence pour mettre en évidence le caractère passe bas.
Faire les calculs puis acquérir et ajuster le diagramme de Bode obtenu par le modèle.
\end{experience}

\begin{remarque}
Prendre le temps de discuter l'altération du spectre en fonction de la différence de fréquence entre le signal et le filtre dans le cas simple du passe-bas.
\end{remarque}

Mettre en évidence le comportement intégrateur du passe-bas par le calcul dans le cas $\omega\gg\omega_c$ et par l'expérience.

De manière générale les filtres permettent d'isoler une partie du spectre d'un signal pour en extraire l'information essentielle dont la nature varie avec l'application : musique, télécommunications, mesures de précisions...
Il existe de nombreux types de filtres linéaires : passe-bas, passe-haut, passe-bande, etc. et pas seulement d'ordre 1.
Ils peuvent agir dans des domaines différents (mécanique, électronique, etc.).

\begin{transition}
Ces systèmes linéaires sont nécessaires pour le traitement du signal dans les télécommunications mais pas suffisants.
\end{transition}

\subsection{Transmission analogique : la radio AM}

\subsubsection{Modulation}

On s'intéresse à la transmission d'une musique pour expliquer le fonctionnement de la radio par exemple.
Il est exclu de les transmettre via des ondes mécaniques pour des raisons évidentes : on préfère utiliser des ondes électromagnétiques.
Il faut donc premièrement transformer le signal sonore en signal électrique grâce à un microphone par exemple.
La fréquence du signal à transmettre est de l'ordre du kHz, ce qui correspond à une longueur d'onde de \unit{300}{\kilo\meter}.
Comme les antennes sont du même ordre de grandeur que le signal à détecter il faudrait des antennes gigantesques !
Pour éviter cela on va utiliser une porteuse de haute fréquence pour transporter l'information base fréquence, ce qui nécessite une modulation.

Cette opération nécessite un composant non linéaire : le multiplieur.
Faire un schéma
\begin{equation}
s_{AM}(t) = (1+\alpha s(t))s_p(t)
\end{equation}
Faire le calcul pour montrer que 
\begin{equation}
s_{AM}(t) = A_p \left[ \cos(\omega_p t) + \frac{m}{2}\cos(\omega_p-\omega)t + \frac{m}{2}\cos(\omega_p+\omega)t \right] 
\end{equation}
On voit immédiatement que le signal modulé contient trois fréquences.

\begin{slide}
\textbf{Spectre du signal modulé en amplitude.}
\end{slide}

Parler du problème de la porteuse qui nécessite beaucoup d'énergie.

\begin{transition}
On obtient donc un signal qui peut être transporté mais qui n'est pas utilisable en tant que tel.
Il est nécessaire de le démoduler.
\end{transition}

\subsubsection{Démodulation synchrone}

Parler rapidement de la détection d'enveloppe.

Faire le calcul de la démodulation synchrone et insister sur l'importance du filtrage.
 
\begin{slide}
\textbf{Spectre du signal démodulé en amplitude.}
\end{slide}

\begin{transition}
Si les télécommunication analogiques sont encore utilisées actuellement, le numérique devient de plus en plus présent.
Inconvénients de la transmissions analogique : sujet au bruit, difficile à stocker.
\end{transition}

\subsection{Traitement numérique}

\subsubsection{Numérisation}

Avantages du numériques \cite{Cardini2017} p136.
Définition de l'échantillonnage \cite{Cardini2017} p127.

\begin{slide}
\textbf{Numérisation d'un signal.}
\end{slide}

Il y a une perte d'information mais facilement mémorisable.

\begin{slide}
\textbf{Critère de Shanon.}
\end{slide}
Parler des DFT, FFT.
Tracer le spectre d'un signal échantillonné et faire apparaitre la création de nouvelle fréquence en raisonnant par analogie avec la modulation vue avant.
Repliement de spectre.
Amener la nécessité de filtrer avant l'acquisition pour éviter le repliement de spectre.

\begin{transition}
\textbf{Template}
\end{transition}

\subsubsection{Filtrage}

Présenter la méthode de discrétisation en prenant l'exemple du passe bas.

\begin{slide}
\textbf{Filtrage numérique.}
\end{slide}

Le traitement du signal peut être fait a posteriori mais aussi en temps réel : parler des micro-contrôleurs qui permettent de réaliser des filtres complexes sans modifier le hardware.
Il y a donc plus de flexibilité !

\subsection*{Conclusion}

\begin{slide}
\textbf{Traitement du signal pour la détection des ondes gravitationnelles.}
Il faut garder du temps pour traiter convenablement cette partie : elle permet de récapituler le contenu de la leçon :
\begin{itemize}
\item étude d'un spectre ;
\item filtrage analogique : expliquer les systèmes de suspension des miroirs filtrage passe-bande pour éliminer hautes et basses fréquence du signal à la sortie de l'interféromètre ;
\item filtrage numérique : traitement du signal pour éliminer les harmoniques, analyse des données (massives) et automatique.
\end{itemize}
\end{slide}

On peut aussi parler des asservissements.

\newpage