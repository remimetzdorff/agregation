\section{LX12 Application du premier principe de la thermodynamique à la réaction chimique}

\begin{header}
\begin{tabular}{p{0.4\textwidth} l}
\niveau & \prerequis \\
Template& \textbullet{} Template \\
        & \textbullet{} Template \\
\end{tabular}

\noindent
\objectif
Template
\end{header}

{
\subsubsection*{Bibliographie}
\footnotesize{}
%\begin{itemize}
%\item 
%\end{itemize}
}

\subsection*{Introduction}

\subsection{Template}

\subsubsection{Template}

\begin{experience}
\textbf{Template}
\end{experience}

\begin{slide}
\textbf{Template}
\end{slide}

\begin{transition}
\textbf{Template}
\end{transition}

\begin{remarque}
\textbf{Template}
\end{remarque}

\newpage