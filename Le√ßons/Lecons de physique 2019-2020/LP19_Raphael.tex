\section{LP19 Diffraction de Fraunhofer (Raphaël) }

\paragraph{Bibliographie :}
\begin{itemize}
\item 
\end{itemize}

\paragraph{Niveau : L3} 

\paragraph{Pré-requis :}
\begin{itemize}
\item Otique géométrique
\item Interférences à deux ondes
\item Transformée de Fourier
\item Produit de convolution
\end{itemize}

\subsection{Introduction}

\emph{Exp : diffraction d'un laser par une fente? D'abord avec une fente large (tache du faisceau) et en diminuant on observe l'élargissement.}
Pas explicable avec l'optique géométrique... optique ondulatoire

\subsection{Phénomène de diffraction}

\subsubsection{Principe d'Huygens Fresnel 1815}

\emph{Diapo : vagues à l'entrée d'un port}

Rappel : modèle scalaire, vibration lumineuse en un point M $s(M)$.

\paragraph{Principe :}
\begin{itemize}
\item chaque élément $d\Sigma$ atteint par la lumière se comporte comme une source secondaire qui va émettre une onde sphérique ;
\item cette ondelette est proportionnelle à la vibration lumineuse incidente et à l'élément de surface $d\Sigma$
\item toutes les sources secondaires sont cohérentes entre elles et vont interférer pour donner naissance à la figure de diffraction.
\end{itemize}
Ce principe est démontrable avec les équations de Maxwell

\emph{Schéma classique avec objet diffractant et système d'axes $(xOy)$ pour le plan d'observation et $(XOY)$ pour l'objet.}

Facteur de transmission
\begin{equation}
t(X,Y) = \frac{s(X,Y, z=0^-)}{s(X,Y, z=0^+)}
\end{equation}

Le pricipe de HF s'écrit :
\begin{equation}
\d s(M) = t(X,Y) A s(P) e^{ikPM} \frac{1}{PM} dX dY
\end{equation}
déroulement du calcul en intégrant...
Hypothèse champs lointain donc $d, D \gg x, y, X, Y$.
Après calcul, DL, on trouve
\begin{equation}
PM \approx \left( 1 + \frac{x^2+y^2}{2D^2} - \frac{xY}{D^2} - \frac{yY}{D^2}\right) 
\end{equation}
\begin{equation}
SP \approx \left( 1 + \frac{x_0^2+y_0^2}{2d^2} - \frac{x_0Y}{d^2} - \frac{y_0Y}{d^2}\right) 
\end{equation}
\begin{equation}
s(M) = \frac{A's_0}{dD} \iint t(X,Y)e^{i \frac{2\pi}{\lambda} \left( -(\alpha - \alpha_0)X - (\beta-\beta_0) Y \right)} \d X \d Y
\end{equation}
On fait donc apparaitre la transformée de Fourier du facteur de transmission :
\begin{equation}
s(M) = K \times \mathrm{TF}[t(X,Y)]\left(\frac{\alpha-\alpha_0}{\lambda}, \frac{\beta-\beta_0}{\lambda} \right)
\end{equation}

\subsubsection{Validité de l'approximation}

Une onde plane incidente est équivalente à une diffraction à l'infini

\emph{Schéma : Montage à deux lentilles puis justification du montage à une lentille.}
 
\subsection{Quelques figures de diffraction}

\subsubsection{Fente rectangulaire}

Largeur $a$ selon $X$ et $b$ selon $Y$ d'où
\begin{equation}
t(X,Y) = \mathrm{rect}_a(X) \mathrm{rect}_b(Y)
\end{equation}
\begin{equation}
s(M) \propto \sinc\left(\frac{\pi a x}{\lambda f'}\right) \sinc\left(\frac{\pi b y}{\lambda f'}\right)
\end{equation}
\begin{equation}
I(M) \propto \sinc^2\left(\frac{\pi a x}{\lambda f'}\right) \sinc^2\left(\frac{\pi b y}{\lambda f'}\right)
\end{equation}

\emph{Schéma du $\sinc^2$}

\emph{Diapo : fentes d'Young}

\subsection{Limitations/applications}

\subsubsection{Limite de résolution angulaire}

Observation de deux étoiles proches grâce à une lunette astronomique. On modélise l'ouverture de la lentille par ue fente de largeur $a$

\emph{Schéma : deux étoiles proches avec deux incidence différentes de faisceaux parallèles.}

Les figures de diffraction de chaque étoile sont centrées sur les images géométriques.
Si les étoiles sont trop proches, on ne peut résoudre les deux étoiles.
On choisit un critère : le critère de Rayleigh.
On calcul l'angle limite $\theta_l$ tel que le maximum d'intensité d'une figure est au même endroit que la première annulation de l'autre figure.
\begin{equation}
\theta_l = \frac{\lambda}{a}
\end{equation}

\subsubsection{Expérience d'Abbe}

\emph{Schéma de l'expérience.}

\emph{Expérience d'Abbe.}

\subsection{Conclusion}

Filtrage spatial, diffraction par des réseaux périodiques.

\subsection{Questions}