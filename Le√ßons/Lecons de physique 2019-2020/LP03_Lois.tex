\section{LP03 Notion de viscosité d'un fluide et écoulement visqueux}

\paragraph{Bibliographie :}
\begin{itemize}
\item 
\end{itemize}

\paragraph{Niveau : CPGE} 

\paragraph{Pré-requis :}
\begin{itemize}
\item Analyse vectorielle ;
\item Calcul différentiel ;
\item Mécanique en référentiel non galiléen ;
\item Phénomène de diffusion.
\end{itemize}

\subsection{Introduction}

\paragraph{Exp :} Cuve en rotation, arrêt de la rotation de la cuve, le liquide ralenti sa rotation.
Ceci est dû à la viscosité du fluide.

\subsection{Interprétation de l'expérience 2'}

On suit une particule de fluide dans la cuve en rotation.
La quantité de mouvement de cette particule est donnée par :
\begin{equation}
p = \rho Hrd\theta dr v_\theta(r,t)
\end{equation}
On s'intéresse à la variation de quantité de mouvement de cette particule de fluide :
\begin{equation}
\frac{dp}{dt} = \rho Hrd\theta dr \frac{dv\theta}{dt}
\end{equation}
La force exercée par le fluide en $r$ est :
\begin{equation}
F_1 = -\eta \frac{\partial v_\theta}{\partial r}(r) Hrd\theta
\end{equation}
\begin{equation}
F_2 = \eta \frac{\partial v_\theta}{\partial r}(r+dr) Hrd\theta
\end{equation}
On applique le PFD et on obtient
\begin{equation}
\frac{dv_\theta}{dt} = -\frac{\eta}{\rho} \frac{\partial^2v_\theta}{\partial r^2}
\end{equation}
On introduit le coefficient de viscosité cinématique $\nu=\eta/\rho$.
Par analyse dimensionnelle et en utilisant le temps d'arrêt de l'eau lors de l'expérience préliminaire, on trouve $\nu \approx 2.10^{-5} m^2/s$

\subsection{Équation de Navier Stokes}

\begin{equation}
\frac{\partial \overrightarrow{v}}{\partial t} = \nu \Delta \overrightarrow{v} - \frac{1}{\rho} \overrightarrow{v}\overrightarrow{\nabla}\overrightarrow{v}-\frac{1}{\rho} \overrightarrow{\nabla}P+\overrightarrow{g}+...
\end{equation}
On trouve l'équation de Navier Stokes
\begin{equation}
\frac{D\overrightarrow{p}}{Dt} = \eta\Delta \overrightarrow{v} - \overrightarrow{\nabla} P + \rho \overrightarrow{g}+...
\end{equation}

\subsubsection{Nombre de Reynolds}

Introduit en faisant apparaitre l'opérateur gradient dimensionné, etc.
\begin{equation}
Re = \frac{\rho L U}{\eta}
\end{equation}

\begin{itemize}
\item $Re<1$ : régime laminaire ;
\item $Re>2000$ : régime turbulent.
\end{itemize}

\paragraph{Vidéo :} 2coulement laminaire puis turbulent dans un tube.

\subsection{Transport d'Eckmann 20'}

\paragraph{Diapo :} Carte de la température de l'océan et la vitesse du vent au niveau de la côté ouest de l'Amérique du sud.

Quelques hypothèses :
\begin{itemize}
\item latitude $\lambda<0$ ;
\item invariant selon x et y ;
\item surface $z=0$ ;
\item stationnaire et incompressible ;
\item horizontal.
\end{itemize}

On applique Navier Stokes, projette sur les trois axes

\subsection{Conclusion}

