\section{LP2 Loi de conservation en dynamique (Richard)}

\paragraph{Bibliographie :}
\begin{itemize}
\item 
\end{itemize}

\paragraph{Niveau : CPGE} 

\paragraph{Pré-requis :}
\begin{itemize}
\item Théorème de la mécanique (PFD, TMC, TEM)
\item Pendule simple oscillateur harmonique
\end{itemize}

\subsection{Introduction}

\subsection{Lois de conservation}

\subsubsection{Conservation de la quantité de mouvement (1')}

Dans un référentiel galiléen \cal{R} :
\begin{equation}
\frac{d\overrightarrow{p}}{dt} = \sum \overrightarrow{F_\mathrm{ext}}.
\end{equation}
En l'absence de force extérieure, on obtient la conservation de la quantité de mouvement
\begin{equation}
\overrightarrow{p} = \overrightarrow{cte}.
\end{equation}
Application à un pécheur qui se déplace sur une barque : conservation de la qiunatité de mouvement du système pécheur + barque pour déterminer la position du système après mise en mouvement du pécheur.

\subsubsection{Conservation du moment cinétique (6')}

\begin{equation}
\frac{d\overrightarrow{L_0}}{dt} = \sum \overrightarrow{M_0}(\overrightarrow{F})
\end{equation}
La conservation du moment cinétique donne $\overleftarrow{L_0} = \overrightarrow{cte}$

\paragraph{Vidéo :} Astronaute qui tourne dans la station spatiale internationale.
On modélise l'astronaute à l'état initial par un cylindre : calcul des moments d'inertie bras pliés et bras dépliés (bras ouvert = 4 bras fermés).
L'astronaute divise sa vitesse de rotation par 4 en ouvrant les bras.
La connaissance de l'état initial et de la quantité conservée permet une fois de plus de trouver le mouvement d'un système pas forcément simple.

\subsubsection{Conservation de l'énergie mécanique 12'}

En l'absence de forces non conservatives, on trouve que l'énergie mécanique est conservée puisque
\begin{equation}
\frac{dE_m}{dt} = \sum W(\overrightarrow{F_\mathrm{NC}}).
\end{equation}

Application au pendule simple pour retrouver l'équation du mouvement
\begin{equation}
\theta^{(2)} + \frac{g}{l}\theta = 0
\end{equation}
dans le cas de petites oscillation.
On pose $\omega_0^2=g/l$
Ici la conservation de l'énergie mécanique permet de donner rapidement une équation qui est suffisante car le problème n'a qu'un seul degré de liberté.

\paragraph{Diapo :} Portrait de phase de l'oscillateur harmonique avec discussion des différents régimes.
La connaissance des paramètres initiaux du système permet de déduire les propriétés du mouvement.

\subsection{Application au problème à deux corps}

\subsubsection{Position et réduction du problème (20')}

Dans un référentiel {\cal{R}} supposé galiléen, on considère le système formé des deux points $M_1$ et $M_2$.
On exprime le PFD pour chacune des masses
\begin{equation}
m_1\frac{d^2\overrightarrow{OM_1}}{dt^2} = \overrightarrow{F_{2\rightarrow 1}}.
\end{equation}
et de même pour 2.
En sommant les deux équations, on trouve l'équation du mouvement du centre de masse du système.
Ensuite :
\begin{equation}
\frac{-(1)}{m_1}+\frac{(2)}{m_2} = \frac{d\overrightarrow{M_1M_2}}{dt^2} = \left( \frac{1}{m_1} + \frac{1}{m_2} \right) \overrightarrow{F_{2\rightarrow 1}}.
\end{equation}
On pose
\begin{equation}
\mu = \frac{m_1m_2}{m_1+m_2}
\end{equation}

\subsubsection{Conservation du moment cinétique (26')}

En supposant que la force est une force centrale, on peut utiliser la conservation du moment cinétique.
Le vecteur 
\begin{equation}
\overrightarrow{L_0} = \mu \overrightarrow{r} \wedge \overrightarrow{v}
\end{equation}
est donc conservé ce qui impose un mouvement plan.
On élimine le vecteur nul qui correspond à une évolution triviale du système.

\subsubsection{Conservation de l'énergie mécanique (31')}

L'énergie mécanique se conserve
\begin{equation}
E_m = \frac{1}{2} \mu \dot{r}^2 + \left(\frac{L^2}{2mr} - \frac{K}{r}\right) = cte.
\end{equation}
On trace le potentiel effectif et on discute des différentes évolutions possibles en fonction de l'énergie mécanique initiale.
