\section{LP10 Induction électromagnétique (Martin)}

\paragraph{Bibliographie :}
\begin{itemize}
\item Perez
\item Dunod
\item Cours Jérémy
\item BFR
\end{itemize}

\paragraph{Niveau : L3} 

\paragraph{Pré-requis :}
\begin{itemize}
\item Electrocinétique ;
\item Magnétostatique ;
\item Forces de Laplace ;
\item Equations de Maxwell ;
\item Potentiels ;
\end{itemize}

\subsection{Introduction}

Qu'est ce que l'induction ?
\paragraph{EXP : } aimant vers une bobine : apparition d'un courant.
C'est l'induction à partir d'un champ B on obtient des courant

\paragraph{Diapo :} exp de Faraday. Il penssait que le courant été créé par un champ mag. mais c'est en allumant la bobine : variation du champ B
Manip : apparition de courant quand l'aimant se déplace dans la bobine.
Pas de i quand B est statique

\subsection{Les lois de l'induction}

\subsubsection{La loi de Faraday 2'30}

On se place dans l'ARQS, contour ABCD, chamb $B$ et on donne une vitesse $v$ au circuit.
On calcule la force de Lorentz (équation), on en déduit la force électromatrice en multipliant par dl (énergie pot) et en divisant par la charge.
On intègre sur tout le circuit.
On utilise les potentiels scalaire et vecteur pour réécrire $E$.
Le terme en $grad$ est nul car contour fermé.
Avec le thm de Stokes Ampère, on transforme $dA/dt$...
\begin{equation}
e = \iint \frac{\partial B}{\partial t} dS + \int (v\times B) dl
\end{equation}
On obtient :
\begin{itemize}
\item Neumann (temporel)
\item Lorentz (vitesse)
\end{itemize}

On écrit la loi de Faraday dans le cas d'un circuit indéformable.
On obtient la formule avec la dérivée totale du flux :
\begin{equation}
e = \frac{D\Phi}{Dt}
\end{equation}
Le signe moins traduit la loi de Lenz

\subsubsection{Loi de Lenz 10'}

Les effets de l'induction s'opposent à la cause qui les a produit.
On peut intuiter les effets de l'induction à partir de cette loi.

\paragraph{Exp : } chute d'un aimant dans un tube plexiglas puis cuivre.
Explication avec la loi de Lenz

\subsection{Induction de Neumann (B variable) 12'}

\subsubsection{Auto-induction}

Schéma d'une bobine parcourue par un champ i.
Importance de l'agébrisation : l'orientation du courant défini l'orientation des surfaces.
Calcul du flux propre si on suppose le champs B créé par la bobine est celui d'un solénoide infini.
Calcul du flux à travers une spire, puis $N$ spires.
\begin{equation}
\Phi_B = \mu_0 N^2 S i /l
\end{equation}
On fait apparaitre l'inductance de la bobine.
Convention générateur/récepteur pour faire le lien entre $e$ et $U_L$.
Puissance stockée dans la bobine.

\paragraph{Diapo, EXP 19'30: } Vérification de la dépendance en $N^2$ de l'inductance de plusieurs bobines en mesurant l'inductance d'après la fonction de transfert d'un circuit RL.
Comparaison entre les valeurs mesurées et les valeurs déduites de la géométrie des bobines.
Fin de l'expérience et des analyses : 25'

\subsubsection{Inductance mutuelle}

Dans le schéma de l'expérience de Faraday.
Coefficients d'auto inductance pour chaque bobine.
Inductance mutuelle qui traduit le couplage entre les deux circuits.
\begin{equation}
e_2 = -L\frac{di_2}{dt} - M\frac{Di_1}{dt}
\end{equation}

On peut transferrer de l'énergie d'un circuit à l'autre : le transformateur.
Application transport de l'énergie à haute tension pour diminuer les pertes dues au transfert puis transfromateur pour abaisser la tension aux maison.

\subsection{Induction de Lorentz (circuit mobile) 28'30}

\subsubsection{Rail de Laplace}

Deux termes : 
\begin{itemize}
\item force électromotrice
\item force de Laplace
\end{itemize}
Schéma du dispositif, mécanique et électrique.

Mise en équation :
\begin{itemize}
\item équation électrique
\item équation mécanique : principe fondamental de la dynamique projeté selon x
\end{itemize}
Discussion qualitative avec la loi de Lenz

33' : on résout les équation pour avoir l'équation du mouvement dans le cas de la tige.
Il faut ajouter un frottement solide pour arrêter la tige.

Conversion électromécanique :
Schéma des échanges ($P_laplace$ et $P_induction$) pour faire le lien entre les pertes par effet Joule et la variation d'énergie cinétique.
On a $P_laplace + P_induit = 0$

\paragraph{Diapo :} Freinage par induction. courant de Foucault, freinage des trains, des poids lourd, Roue de Barlow comme générateur de courant. L'induction est ainsi à la base des méthodes de production d'électricité actuels.


\subsection{Conclusion 39'}

Rappel et applications : générateurs, chauffage par induction, micro, chargement à distance..

\paragraph{Vidéo :} Railgun de l'espace

\subsection{Question}

\begin{enumerate}
\item Approche historique avec l'exp de Faraday : c'est quoi un galvanomètre ?
\item Lois de l'induction : trois ingrédients : force de Lorentz, E = -gradV - dA/dT, $v\times B$. Commenter les termes. On repart de Maxwell pour exprimer E en fonction des potentiels vecteur et scalaire.
\item Que représentent V et A ? Cas du solénoide avec une spire autour, Ahramov Bohm. On mesure pas V on mesure des ddp. Pour A ?
\item Comment pourrait-on préciser l'introduction de $e$ ? Lien entre le travail et la ddp (le travail de la force de lorentz permet d'introduire la ddp)
\item dans la force de lorentz, Que représente v ? C'est la vitesse des porteurs.
\item quel est le référentiel dans lequel est défini v ?
\item vitesse du circuit = vitesse des charges ? composition des vitesses $v_e = v_circuit+v_electron$. Il n'apparait que la vitesse du circuit car les électrons se déplacent le long du circuit
\item Comment évolue la vitesse de chute de l'aimant en fonction du matériau du tube ?
\item Justifier l'approximation du solénoide infini pour les bobines ?
\item Pourquoi sommer les champs B ? Linéarité des eq de Maxwell
\item Pourquoi se placer en convention générateur ?
\item Puissance stockée dans la bobine. Pourquoi iUL et pas ei : convention récepteur/géné
\item manip : Quelle est la valeur de la résistance de la bobine ? justifier le choix de la résistance.
\item Incertitudes ? viennent des fits. Comment sont calculées les erreurs ?
\item Possible de les rajouter à la main ? elles sont petites venant de l'oscilloscope
\item Pourquoi avoir choisi de mesurer L d'après le diagramme de Bode ? Autres méthodes : umax/2, temps de montée, etc. Justifiée par une approche pédagogique.
\item Valeur d'inductance comparée à quoi ?
\item d'où vient l'incertitude sur l'inductance théorique ? inhomogénéité du champ B dans toutes les spires
\item Inductance mutuelle ? pourquoi mettre des dérivée ronde ? (erreur)
\item Une seule équation : couplage de 1 vers 2. Qu'est ce qui se passe dans l'autre sens ?
\item L'inductance mutuelle est elle la même de 12 ou 21 ? Oui
\item Equivalensce des puissance de laplace et induit ? Oui car sinon fuck la physique.
\item Roue de Barlow pour générer du courant ?

\subsection{Commentaires}

Débit de parole important. Faire quelques pauses.
Bien expliqué
Point noir de la leçon : comment amener les lois de l'induction ?
On peut écrire que la vitesse des électrons est la composition des électrons dans le circuit + la vitesse du circuit.
Une autre possibilité serait de faire l'approche historique entière et introduire le dphi directement.
Bonne utilisation des couleurs, mais défaut de diction : pas dire mon/ma tout le temps (mesurer mon inductance...)
Bq d'applications : bien
Faire des pauses, poser des questions ouvertes, pendant l'écriture du titre tu te tais.
A n'est pas mesurable car il y a une jauge près mais ddp mesurable et la circulation de A aussi.



\end{enumerate}