\section{LP04 Synthèse inorganique}

\paragraph{Bibliographie :}
\begin{itemize}
\item ;
\end{itemize}

\paragraph{Niveau :} CPGE

\paragraph{Pré-requis :}
\begin{itemize}
\item Réaction d'oxydo-réduction ;
\item Réaction acido-basique ;
\item Titrage indirect.
\end{itemize}

\paragraph{Objectifs de la leçon :}
\begin{itemize}
\item
\end{itemize}

\paragraph{Expériences :}
\begin{itemize}
\item
\end{itemize}


\subsection{Introduction}

\paragraph{Diapo :} Rappels sur les réactions acide/base et oxydoréduction, formule de Nernst, zone de prédominance d'une espèce d'un couple acide/base.

\subsection{Diagramme potentiel-pH}

\paragraph{EXP :} mélange des ions iodate avec thiocyanate : une réaction d'oxydoréduction peut modifier le pH d'une solution.

Il est nécessaire d'utiliser un diagramme à deux dimensions : diagramme de Pourbaix

\subsubsection{Nécessité d'un diagramme à deux dimensions}

\begin{equation}
\mathrm{IO_3^-+6S_2O_3^{2-} + H^+ = I^- + 3S_4O_6^{2-}+3H_2O}
\end{equation}
On écrit l'égalité des potentiels pour les couples, on trouve l'équation de la droite sur le diagramme.

\subsubsection{Diagramme potentiel-pH du Fer}

\paragraph{Diapo :} diagramme E-pH du Fer et de l'eau.
Réflexion autour du digramme : quelle espèce réagit avec quoi etc. 

\paragraph{Exp : } clou dans de l'eau

\subsection{Application industrielle : obtention de l'alumine}

Aluminium : métal le plus présent à la surface de la Terre mais utilisé tardivement en raison de sa présence sous forme oxydée.
Procédé d'obtention de l'alu attribuée à Bayer à partir de la bauxite (mélange de Fer et Aluminium oxydés)

\paragraph{Exp :} Dissolution d'hydroxyde d'aluminium en milieu acide. Ajout de base : précipitation puis avec plus de base nouvelle dissolution.

\paragraph{Diapo} Diagramme E-pH de l'alu, fer et eau

\subsection{Application au contrôle de la qualité de l'eau}

Analyse de la qualité de l'eau à Montrouge.

\paragraph{Diapo :} Diagramme E-pH de Winkler.

\subsection{Conclusion}

\subsection{Commentaires de l'enseignant}

(Clément Guibert : clementguilbert@sorbonne-universite.fr)

La leçon peut être centrée sur l'utilisation du diagramme potentiel pH : superposition de plusieurs diagrammes potentiels pH.

Contextualisation !!!