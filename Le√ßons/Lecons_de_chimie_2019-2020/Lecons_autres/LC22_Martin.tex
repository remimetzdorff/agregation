\section{LC22 Évolution et équilibre chimique}

\niveau CPGE

\prerequis
\begin{itemize}
\item Premier principe ;
\item Second principe ;
\item Potentiel chimique ;
\item Grandeur de réaction.
\end{itemize}

\objectif Template

\footnotesize{\bibliography{biblio}}
\bibentry{Template}

\subsection{Introduction}

\begin{experience}
\textbf{Étude de l'équilibre $\mathrm{N_2O_{4(g)} = 2NO_{2(g)}}$}.
Pour trois températures (0 dans un bain de glace, 25 à température ambiante et \unit{50}{\degree C} dans un bain marie), on compare la couleur du mélange contenu dans un erlenmeyer.

On l'a synthétisé en préparation.
Explication de la manipulation.
Le dioxyde d'azote est roux alors que l'autre est incolore.
On montre les trois erlenmeyer qui ont des couleurs différentes.
Cela montre qu'on a trois équilibre différents.
\end{experience}

\subsection{Évolution d'un système vers l'équilibre}
\subsubsection{Équilibre d'un système (2')}

On fait plusieurs hypothèses :
\begin{itemize}
\item on se place à l'équilibre thermodynamique ;
\item on considère un système fermé ;
\item on s'intéresse à des transformations isothermes et isobares.
\end{itemize}

Pour caractériser le système, on s'intéresse à l'entropie, avec les variables naturelle ($V, S, n_i$).
On préfère contrôler la température et la pression donc on utilise l'enthalpie libre $G$ avec les variables naturelles ($P, T, n_i$).

\begin{transition}
On peut décrire l'évolution d'un système, mais dans notre système, on a une réaction chimique.
\end{transition}

\subsubsection{Évolution d'une réaction chimique (7')}

On fait un tableau d'avancement pour la réaction du NO2.


\begin{transition}
On vient de caractériser l'évolution du système vers l'état d'équilibre mais pas l'équilibre lui même.
\end{transition}

\subsubsection{Caractérisation de l'état d'équilibre ( 11'30'')}

\begin{experience}
\textbf{Mesure du $pK_a$ de l'acide éthanoïque.}
\end{experience}


\begin{transition}
On peut maintenant décrire l'état d'équilibre et l'évolution du système vers cet état. Est-il possible de jouer sur la composition chimique de l'état d'équilibre ?
\end{transition}

\subsection{Déplacement d'équilibre}

\subsubsection{Influence de la température (21')}



\begin{transition}
Est-il possible de déplacer l'équilibre avec d'autres facteurs ?
\end{transition}

\subsubsection{Influence de la pression (28'30'')}

\begin{experience}
\textbf{Évolution de la réaction $\mathrm{N_2O_{4(g)} = 2NO_{2(g)}}$ en fonction de la pression.}
\end{experience}

\begin{transition}
Il existe d'autres façons de déplacer l'équilibre.
\end{transition}

\subsubsection{Rupture d'équilibre (33')}

\begin{experience}
\textbf{Synthèse de l'ester de poire.}
\end{experience}

\subsection{Conclusion (38')}

\begin{slide}
Stylé
\end{slide}

\begin{remarque}
Pompélopie
\end{remarque}
\note{Trop de la bombe}

\paragraph{Question :}
\begin{itemize}
\item La manip introductive : l'auriez vous réalisé avec des étudiants ?
Non car le gaz est irritant.
Mais on peut préparer le gaz à l'avance et montrer les résultats avec les différentes température.
\item Définir l'éq thermo ?
Il est possible de définir des variables d'état.
\item Un système fermé ?
Ne peut pas écnager de matière.
\item Sur la mesure de pH pour $K$, quelles sont les incertitudes ?
L'autre bécher.
\item Y a-t-il un exemple où on veux déplacer l'équilibre vers les réactifs ?
Acidification des océans.
\item Considérez vous l'approximation d'Ellingham comme un prérequis ?
Non on le verra quand on définit les grandeurs de réaction.
\item Sur la loi de Van't Hoff : 
\item Pour la pression et la seringue : qu'est ce qui varie ?
En faisant varier le volume on chnage la pression (hypothèse gaz parfait).
\item Peut-on considérer le gaz comme parfait ? Comment le vérifier ?
Faire des détentes de Joule Gay-Lussace, tracer $PV$ en fonction du volume, mesurer la température et la pression pour plusierus volumes etc.
\item Le procédé Haber-Bosch : vous avez dit 200 atmosphères, est-ce la bonne unité ?
Non il faut travailler en Pa.
\item Sur la synthèse de l'ester de poire : le rendement de l'estérification est toujours de 67\% ?
Pour cette réaction oui mais dépend des réactif pour d'autres estérifications.
\item Quelles sont les activités des constituants pour le calcul de la constante d'équilibre ?
Il faut utiliser le coefficient d'activité.
\item Est-il nécessaire de mettre de Dean-Satrk dans les prérequis ? Est-ce une bonne occasion de l'introduire ou vaut-il mieux le faire de façon expérimentale ?
De manière expérimentale, il y a plus d'interaction avec les élève.
C'est aussi la première fois que cet appareillage est nécessaire.
\item Y a-t-il d'autres manières de mesurer le rendement de la réaction ?
Oui cf protocole. Il aurait peut-être fallu utiliser une garde CaCl2.
\item Peux-tu dessiner le profil réactionnel dans le cas endo-exothermique ?
\item Peux-tu réécrire $G = H-TS$ ?
Il y avait une erreur de signe.
\item Variable de Degonder
\item Est-ce que la pression influe sur $K$ ?
\item La réaction d'estérification est-elle exo ou endothermique ?
Elle est athermique.
\item Pourquoi chauffe-t-on ? Cinétique, Arrhénius, etc.
\item Dans le cas d'une réaction catalysée que devient l'énergie d'activation.
\item Utilise-t-on l'ajout de réactif pour déplacer l'équilibre ?
Oui dans le procédé Haber-Bosch.
\item Avec le Dean-Stark, qu'est ce qui s'évapore quand on chauffe ?
Un mélange
\item Donne un exemple de réaction qui est favorisée à basse température ?
Dissolution du calcaire, dissolution du CO2
\item Fonctionnement du pH-mètre ?
\item Comment neutraliser le gaz ? Dans l'eau
\item Vous avez dit MON pH-mètre : il vous a couté combien ? LOOOOOOOOL
\item Comment lutter contre l'inégalité Homme/Femme en Science ?
Mentionner des femmes qui ont fait avancer la science.
Lutter contre les comportements genre je ne veux pas être avec elle car elle nulle.
Les femmes n'avaient pas le droit d'accéder au milieu académique c'est pour ça qu'il y en a peut.
Ne pas genrer les métier : la préparatrice.
\end{itemize}

\paragraph{Remarques :}
\begin{itemize}
\item La leçon est pas facile, mais tu t'en es bien tiré, très clair, avec des applications, des calculs etc. C'est très bien !
\item Lors des questions, il faut pas hésiter à justifier le choix : ce n'est pas frocément une critique.
\end{itemize}