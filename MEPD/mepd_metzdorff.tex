\documentclass[12pt,a4paper]{article}
\usepackage[utf8]{inputenc}
\usepackage[french]{babel}
\usepackage[T1]{fontenc}
\usepackage{amsmath}
\usepackage{amsfonts}
\usepackage{amssymb}
\usepackage{graphicx}
\usepackage[left=2cm,right=2cm,top=2cm,bottom=2cm]{geometry}
\usepackage[thinspace,thinqspace,amssymb]{SIunits}

\title{Mise en perspective didactique d'un dossier de recherche}
\author{Rémi Metzdorff}
\date{\today}

\begin{document}

\maketitle

\section{Parcours universitaire et scientifique}

\subsection{Formation et doctorat}

\paragraph{2015--2019 Doctorat :}
Thèse en optomécanique réalisée au sein de l'équipe Optomécanique et Mesures Quantiques du laboratoire Kastler Brossel (LKB, Paris) sous la direction de Pierre-François Cohadon intitulée \textbf{Refroidissement de résonateurs macroscopiques proche de leur état quantique fondamental}, soutenue le 23 Juillet 2019.

\paragraph{2014--2015 Master 2 :}
Parcours LuMI (Lumière, Matière, Interactions) du master OMP (Optique, Matière, Plasma) de l'UPMC (Paris).
Stage de recherche au LKB (6 mois) sur l'étude de l'évolution des pertes dans des cavités Fabry Perot de grande finesses en fonction de leur longueur.

\paragraph{2013--2014 Master 1 :}
Parcours Physique Générale du master Physique et Applications de l'UPMC (Paris).
Stage au LKB (4 mois) sur le développement de collimateurs compacts pour les faisceaux d'un piège magnéto-optique à trois dimensions (3D-MOT) sur une expérience de métrologie.

\paragraph{2012--2013 Licence 3 :}
Parcours Physique-Chimie de l'UPMC (Paris).

\paragraph{2010--2012 CPGE :} PCSI et PC$^*$ au Lycée Louis-le-Grand (Paris).

\subsection{Enseignements, diffusion et vulgarisation}

\paragraph{2015--2018 Monitorats :}
\begin{itemize}
\item travaux pratiques d'électromagnétisme (L1) ;
\item travaux dirigés et travaux pratiques de physique expérimentale (L2-L3) : programmation, électronique, Arduino ;
\item travaux pratiques d'optique (M1) : laser, diode laser.
\end{itemize}

\paragraph{2015--2019 Conférences :} COLOQ, CLEO, ... 

\paragraph{Vulgarisation :}
\begin{itemize}
\item \textbf{Un chercheur, Une manip (2019) :} Présentations au Palais de la Découverte autour de la découverte des ondes gravitationnelles ;
\item \textbf{Fêtes de la science :} Présentation du LKB au grand public, conférence sur la détection d'ondes gravitationnelles, animations grand public ;
\item \textbf{E=M6 :} Casser un verre avec la voix ;
\item \textbf{Olympiades de physique :} Conférence sur la découverte des ondes gravitationnelles.
\end{itemize}

\section{Travaux de recherche}

\subsection{Contexte}

De l'obtention de références de temps ultra stables (horloges atomiques) à l'interférométrie gravitationnelle (Virgo, LIGO), les mesures les plus précises réalisées actuellement reposent sur des méthodes interférométriques.
Le principe de ces mesures est de mesurer un déphasage entre deux ondes qui parcourent des chemins différents

Les propriétés de la lumière en font un outil de choix pour de nombreuses mesures de déplacements



\section{Enseignement}

\end{document}