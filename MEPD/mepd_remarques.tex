\documentclass[12pt,a4paper]{article}
\usepackage[utf8]{inputenc}
\usepackage[french]{babel}
\usepackage[T1]{fontenc}
\usepackage{amsmath}
\usepackage{amsfonts}
\usepackage{amssymb}
\usepackage{graphicx}
\usepackage[left=2cm,right=2cm,top=2cm,bottom=2cm]{geometry}
\usepackage[thinspace,thinqspace,amssymb]{SIunits}
\usepackage{eurosym}

\usepackage{hyperref}
\hypersetup{
    colorlinks=true,
    urlcolor=theme
}

\title{Remarques sur le dossier de mise en perspective didactique d'un dossier de recherche}
\author{\textbf{Rémi Metzdorff}}
\date{Concours externe spécial de l'agrégation de physique-chimie option physique Session 2020}
%\date{2020}

\renewcommand{\d}{\mathrm{d}}
\newcommand{\uroc}{\micro RoC}

%%%%%%%%%%%%%%%%%%%%%%%%%%%%%%  For CV  %%%%%%%%%%%%%%%%%%%%%%%%%%%%%%
\usepackage{array}
%\newcolumntype{L}[1]{>{\raggedright\let\newline\\\arraybackslash\hspace{0pt}}m{#1}}
%\newcolumntype{C}[1]{>{\centering\let\newline\\\arraybackslash\hspace{0pt}}m{#1}}
%\newcolumntype{R}[1]{>{\raggedleft\let\newline\\\arraybackslash\hspace{0pt}}m{#1}}
%%%%%%%%%%%%%%%%%%%%%%%%%%%%%%%%%%%%%%%%%%%%%%%%%%%%%%%%%%%%%%%%%%%%%%

\usepackage{xcolor}
\definecolor{theme}{RGB}{56,115,179}
\usepackage[font=small]{caption}

%%%%%%%%%%%%%%%%%%%%%%%%%%%%%%%%%%%%%%%%%%%%%%%%%%%%%%%%%%%%%%%%%%%%%%
\usepackage[framemethod=tikz]{mdframed}

\mdfdefinestyle{s_mep}{
	linecolor=theme!,
	outerlinewidth=3pt,
	frametitlerule=false,
	topline=false,
	bottomline=false,
	rightline=false,
	backgroundcolor=white,
	innertopmargin=5pt,
	roundcorner=0pt,
	%skipabove=5pt,
	skipbelow=5pt
}
\newmdenv[style=s_mep]{mep_env}

\newenvironment{mep}{%\stepcounter{exa}%
%\newenvironment{myenv}{\begin{adjustwidth}{2cm}{}}{\end{adjustwidth}}
\addcontentsline{ldf}{figure}{0}%
\begin{mep_env}
\small}
{\end{mep_env}}
%%%%%%%%%%%%%%%%%%%%%%%%%%%%%%%%%%%%%%%%%%%%%%%%%%%%%%%%%%%%%%%%%%%%%%

\begin{document}

\maketitle

\section{Parcours universitaire et scientifique}

\section{Refroidissement vers l'état quantique fondamental}

\subsection{Mesure de déplacements et optomécanique}
\label{sec:intro}

\subsubsection{Les mesures de déplacements et leurs applications}

Revoir comment sont extraits les paramètres de la source d'OG d'après le signal : localisation, masses, énergie émise.

\subsubsection{L'optomécanique : entre limite de sensibilité...}

\subsubsection{... Et contrôle d'oscillateurs mécaniques}

\subsection{Les oscillateurs mécaniques}
\label{sec:mechanical_oscillators}

\paragraph{Resonant approximation : lorentzian shape\\}
La susceptibilité mécanique de l'oscillateur amorti s'écrit
\begin{equation}
\chi[\Omega] = \frac{1}{m_\mathrm{eff}\left[\Omega_m^2-\Omega^2 - i\Gamma_m\Omega\right]} = \frac{1}{m_\mathrm{eff}\left[(\Omega_m+\Omega)(\Omega_m-\Omega) - i\Gamma_m\Omega\right]}.
\end{equation}
Si $\Gamma_m \ll \Omega_m$, on peut s'intéresser à l'évolution de la susceptibilité proche de la résonance, soit pour $\Omega \approx \Omega_m$.
On a alors aussi $\Omega_m+\Omega \approx 2\Omega_m$, si bien que
\begin{equation}
\chi[\Omega] \approx \frac{1}{m_\mathrm{eff}\left[2\Omega_m(\Omega_m-\Omega) - i\Gamma_m\Omega_m\right]} = \frac{1}{m_\mathrm{eff}\Gamma_m\Omega_m\left[\frac{\Omega_m-\Omega}{\frac{\Gamma_m}{2}} - i\right]}.
\end{equation}
Pour la spectre de déplacement, on trace la densité spectrale de puissance du bruit de position, proportionnelle au carré du module de la susceptibilité mécanique
\begin{equation}
\left|\chi[\Omega]\right|^2 = \frac{\frac{1}{(m_\mathrm{eff}\Gamma_m\Omega_m)^2}}{\left[\left(\frac{\Omega_m-\Omega}{\frac{\Gamma_m}{2}}\right)^2 + 1\right]},
\end{equation}
ce qui est bien une lorentzienne.

\paragraph{Oscillateur harmonique quantique\\}
Refaire le calcul (TD de MQ d'Arnaud) et marquer les différences avec le puits infini rectangulaire.
Différentes manifestations de la quantification :
\begin{itemize}
\item niveaux d'énergie discrets ;
\item énergie minimale non nulle,
\end{itemize}
qui se traduisent sur la mesure :
\begin{itemize}
\item fluctuations de point zéro ;
\item asymétrie des bandes latérales mécaniques.
\end{itemize}

Les phonons sont des bosons.

\subsection{Une cavité optique pour augmenter la sensibilité}
\label{sec:cavity}

\subsection{Couplage optomécanique et refroidissement}
\label{sec:optomechanics}

\paragraph{Approche thermodynamique du self-cooling\\}
Voir Aspelmeyer 2014.

\paragraph{Approche quantique : processus Stokes et anti-Stokes\\}
Voir Aspelmeyer 2014.


\subsection{Principaux résultats obtenus}
\label{sec:results}

\subsection{Vers des mesures sous la limite quantique standard}
\label{sec:prospects}

\section{Enseignement, diffusion et vulgarisation}

\subsection{Valorisation des travaux de recherche}

\paragraph{Transferts thermiques\\}

\subsubsection{Filtrage, micro-contrôleurs et asservissements}
\label{sec:controls}

\paragraph{Super-atténuateurs de Virgo\\}

Il s'agit d'un système complexe composé de sept oscillateurs couplés.
Il faut relire l'article \emph{The seismic Superattenuators of the Virgo gravitational waves interferometer}, section 3.
La cascade de pendules permet d'avoir une dépendance forte de la réponse en déplacement u miroir final avec la fréquence.
Au delà de la plus haute des fréquences de résonance $f_n$ associées aux modes propres du système, la réponse d'un pendule composé de $N$ étages est donnée par
\begin{equation}
\frac{C}{f^{2N}},
\end{equation}
où $C = \prod_{n=1}^N f_n^2$.
Le résultat est similaire dans le cas des lames utilisées pour l'isolation dans la direction verticale.

Dans le cas des pendules, le couplage est inertiel.
Dans le cas des lames, le couplage est élastique.


\subsubsection{Un interféromètre de Michelson imprimé en 3D}

\subsection{Enseignements}

\subsection{Vulgarisation : résonance et ondes gravitationnelles}

\end{document}