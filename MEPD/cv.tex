\documentclass[11pt,a4paper]{moderncv}
\moderncvtheme[blue]{classic}                
\usepackage[utf8]{inputenc}
\usepackage[francais]{babel}
\usepackage[top=1.1cm, bottom=1.1cm, left=2cm, right=2cm]{geometry}
% Largeur de la colonne pour les dates
\setlength{\hintscolumnwidth}{2.5cm}

\firstname{Rémi}
\familyname{Metzdorff}
\title{Étudiant en Master de Physique}              
\address{177-179 Bd. Maxime Gorki}{94800 VILLEJUIF}    
\email{remi.metzdorff@etu.upmc.fr}                      
\mobile{06 33 23 76 69} 
\extrainfo{22 ans -- Permis B}

\begin{document}

\maketitle

\section{Formation}
% \cventry{Début -- Fin}{}{}{}{}{}
  \cventry{2014 -- 2015}{Master 2 Optique, Matière, Plasma}{Université Pierre et Marie Curie (75)}{}{}{Parcours Lumière, Matière, Interactions}
  \cventry{2013 -- 2014}{Master 1 de Sciences et Technologies}{Université Pierre et Marie Curie (75)}{}{}{Spécialité Physique générale}
  \cventry{2013}{Obtention du diplôme de Licence en Sciences et technologies}{Université Pierre et Marie Curie (75)}{Mention Très Bien}{}{Spécialité Physique -- Chimie}
  %\cventry{2012 -- 2013}{Licence de Sciences et Technologies}{Université Pierre et Marie Curie (75)}{}{}{Spécialité Physique -- Chimie}
  \cventry{2010 -- 2012}{Classe préparatoire aux grandes écoles}{Lycée Louis le Grand (75)}{}{}{Spécialité Physique -- Chimie (PCSI puis PC*)}
  %\cventry{2011 -- 2012}{Classe préparatoire aux grandes écoles}{Lycée Louis le Grand (75)}{}{}{Spécialité Physique -- Chimie (PC*)}
  %\cventry{2010 -- 2011}{Classe préparatoire aux grandes écoles}{Lycée Louis le Grand (75)}{}{}{Spécialité Physique -- Chimie -- Sciences de l'ingénieur (PCSI)}
  \cventry{2010}{Obtention du diplôme du Baccalauréat}{Lycée Marguerite de Navarre (18)}{Mention Très Bien}{}{Série Scientifique, Spécialité Physique -- Chimie}

\section{Expériences professionnelles}
  \cventry{Avril 2015 -- Actuellement}{Stage au laboratoire Kastler -- Brossel (75) (4 mois)}{}{\'Etude de dégénérescences de modes dans une micro-cavité optique}{}{Optique, électronique, interfaçage}
  \cventry{Decembre 2014 -- Mars 2015}{Correcteur et interrogateur pour les Cours Valin}{}{Corrections d'examens blancs et interrogations orales en physique-chimie pour des classes de terminales scientifiques}{}{}
  \cventry{Mai 2014 -- Juillet 2014}{Stage au laboratoire Kastler -- Brossel (75) (3 mois)}{}{Optimisation d'un collimateur pour 3D-MOT et divers projets autour d'une expérience d'atomes froids}{}{Simulation numérique, prototypage, optique, électronique}
  %\cventry{Juin 2013}{Stage au laboratoire Kastler -- Brossel (75) (1 mois)}{}{Réalisation d’une détection homodyne pour la mesure de déplacement d’un micro-oscillateur harmonique}{}{Électronique, laser, dessin de pièces mécaniques}
  \cventry{Mars 2013 -- Mai 2013}{Stage à la Cité des sciences et de l'industrie (3 mois)}{Médiation scientifique}{}{}{Création et présentation d’un atelier sur les remèdes de grand-mère, lors de la nuit des musées}
  \cventry{Janvier 2013}{Stage à l’Institut de Minéralogie et de Physique des Milieux Condensés (75) (1 mois)}{}{Les oxydes colorants du chrome pour les décors de porcelaine du $XIX^{ème}$  siècle à nos jours}{}{Spectroscopie UV-Vis-NIR, Diffraction des rayons X}
  % \cventry{Aout 2011}{Ouvrier paysagiste chez TARVEL (18)}{}{}{}{Mise en place d’un système d’arrosage automatique, entretien d'espaces verts}

\section{Compétences}
  \cvitem{Langages}{Python, C, Fortran, Mapple, \LaTeX}{}{}{}{}
  \cvitem{Systèmes}{Windows, Linux, Mac OS X}{}{}{}{}
  \cvitem{Logiciels}{Catia, Eagle, OSLO}{}{}{}{}
%  \cvitem{Joomla}{Maintenance du site http://www.ena-aikido.com/}{}{}{}{}
  \cvitem{Anglais}{Lu, écrit et parlé}{}{}{}{}
  \cvitem{Allemand}{Notions}{}{}{}{}

\section{Centres d'intérêts}
  \cvitem{Aïkido}{Pratique depuis six ans au dojo Shinkai (Paris), dont quatre ans comme assistant au cours enfant}
  \cvitem{Randonnée}{Deux semaines sur les Chemins de Compostelle durant l'été 2013}

\clearpage

\recipient{\bigskip \bigskip \bigskip \bigskip A l'attention de M. Frédéric Chevy,}{Enseignant-chercheur au laboratoire Kastler Brossel\\
\bigskip
\bigskip
} % Letter recipient
\date{} % Letter date \today
\opening{\textbf{Objet : Candidature pour une mission doctorale d'enseignement.} \\ \bigskip \bigskip \bigskip \bigskip \bigskip
 Monsieur Frédéric Chevy,} % Opening greeting
\closing{Afin de vous assurer de ma motivation, je me tiens à votre disposition et vous prie d'agréer, Monsieur, mes sincères salutations.}
% \enclosure[Attached]{curriculum vit\ae{}} % List of enclosed documents
\makelettertitle % Print letter title

\setlength\rightskip{-\leftskip}

Au cours de ma première année de thèse à l'UPMC, je suis vivement intéressé par la possibilité d'effectuer une mission doctorale d'enseignement dans un établissement prestigieux comme l'ENS. L'occasion de partager et d'échanger mes connaissances avec des étudiants motivés et intéressés est pour moi très stimulante. 

Actuellement en stage de Master 2 au Laboratoire Kastler Brossel, j'y débuterai ma thèse l'année prochaine dans l'équipe Optomécanique et mesures quantiques sous la direction de Pierre-François Cohadon sur le sujet \emph{Squeezed light to beat quantum limits in optomechanical systems}. Au cours de ma formation comme dans le cadre d'activités extra scolaires, j'ai déjà assuré différentes facettes du travail de l'enseignant face à des publics variés, qu'il s'agisse de diffusion et de transmission de savoirs ou d'évaluations d'élèves.

Ainsi, je serai très heureux de pouvoir mettre à disposition mes compétences pédagogiques et mes connaissances au service des étudiants pour les aider à acquérir les notions et à remplir les objectifs fixés par leur parcours. Cette expérience serais par ailleurs très bénéfique pour mon projet professionnel que je dirige vers l'enseignement et la recherche.

\makeletterclosing % Print letter signature

\end{document}