\documentclass[12pt,a4paper]{article}
\usepackage[utf8]{inputenc}
\usepackage[french]{babel}
\usepackage[T1]{fontenc}
\usepackage{amsmath}
\usepackage{amsfonts}
\usepackage{amssymb}
\usepackage{graphicx}
\usepackage[left=2cm,right=2cm,top=2cm,bottom=2cm]{geometry}
\usepackage[thinspace,thinqspace,amssymb]{SIunits}
\usepackage{eurosym}

\usepackage{hyperref}
\hypersetup{
    colorlinks=true,
    urlcolor=theme
}

\title{Notes pour la présentation du dossier de mise en perspective didactique d'un dossier de recherche}
\author{\textbf{Rémi Metzdorff}}
\date{Concours externe spécial de l'agrégation de physique-chimie option physique Session 2020}
%\date{2020}

\renewcommand{\d}{\mathrm{d}}
\newcommand{\uroc}{\micro RoC}

%%%%%%%%%%%%%%%%%%%%%%%%%%%%%%  For CV  %%%%%%%%%%%%%%%%%%%%%%%%%%%%%%
\usepackage{array}
%\newcolumntype{L}[1]{>{\raggedright\let\newline\\\arraybackslash\hspace{0pt}}m{#1}}
%\newcolumntype{C}[1]{>{\centering\let\newline\\\arraybackslash\hspace{0pt}}m{#1}}
%\newcolumntype{R}[1]{>{\raggedleft\let\newline\\\arraybackslash\hspace{0pt}}m{#1}}
%%%%%%%%%%%%%%%%%%%%%%%%%%%%%%%%%%%%%%%%%%%%%%%%%%%%%%%%%%%%%%%%%%%%%%

\usepackage{xcolor}
\definecolor{theme}{RGB}{56,115,179}
\usepackage[font=small]{caption}

\definecolor{gray_f}{RGB}{68,84,106}
\definecolor{gray_c}{RGB}{214,220,229}
\definecolor{bleu_f}{RGB}{91,155,213}
\definecolor{bleu_c}{RGB}{222,235,247}
\definecolor{red_f}{RGB}{204,0,0}
\definecolor{red_c}{RGB}{245,204,204}
\definecolor{orange_f}{RGB}{237,125,49}
\definecolor{orange_c}{RGB}{251,229,214}
\definecolor{green_f}{RGB}{112,173,71}
\definecolor{green_c}{RGB}{226,240,217}
\definecolor{yellow_f}{RGB}{255,192,0}
\definecolor{yellow_c}{RGB}{255,242,204}

%%%%%%%%%%%%%%%%%%%%%%%%%%%%%%%%%%%%%%%%%%%%%%%%%%%%%%%%%%%%%%%%%%%%%%
\usepackage[framemethod=tikz]{mdframed}

\mdfdefinestyle{s_mep}{
	linecolor=theme!,
	outerlinewidth=3pt,
	frametitlerule=false,
	topline=false,
	bottomline=false,
	rightline=false,
	backgroundcolor=white,
	innertopmargin=5pt,
	roundcorner=0pt,
	%skipabove=5pt,
	skipbelow=5pt
}
\newmdenv[style=s_mep]{mep_env}

\newenvironment{mep}{%\stepcounter{exa}%
%\newenvironment{myenv}{\begin{adjustwidth}{2cm}{}}{\end{adjustwidth}}
\addcontentsline{ldf}{figure}{0}%
\begin{mep_env}
\small}
{\end{mep_env}}
%%%%%%%%%%%%%%%%%%%%%%%%%%%%%%%%%%%%%%%%%%%%%%%%%%%%%%%%%%%%%%%%%%%%%%

%%%%%%%%%%%%%%%%%%%%%%%%%%%%%%%%%%%%%
% Transition

\mdfdefinestyle{s_trans}{%
	linecolor=yellow_f!,
	outerlinewidth=3pt,%
	frametitlerule=false,
	topline=false,
	bottomline=false,
	rightline=false,
	backgroundcolor=yellow_c,
	innertopmargin=8pt,
	roundcorner=0pt,
	nobreak=true
}
\newmdenv[style=s_trans]{transition2_env}
\newenvironment{transition}
{%\stepcounter{exa}%
	\addcontentsline{ldf}{figure}{0}%
	\begin{transition2_env}}
	{\end{transition2_env}}

%%%%%%%%%%%%%%%%%%%%%%%%%%%%%%%%%%%%%

%%%%%%%%%%%%%%%%%%%%%%%%%%%%%%%%%%%%%
% Experience

\mdfdefinestyle{s_experience}{%
	linecolor=bleu_f!,
	outerlinewidth=3pt,%
	frametitlerule=false,
	topline=false,
	bottomline=false,
	rightline=false,
	backgroundcolor=bleu_c,
	innertopmargin=8pt,
	roundcorner=0pt,
	nobreak=true
}
\newmdenv[style=s_experience]{experience_env}
\newenvironment{experience}
{%\stepcounter{exa}%
	\addcontentsline{ldf}{figure}{0}%
	\begin{experience_env}}
%	\begin{experience_env}[]{\noindent\colorbox[rgb]{0.1 0.1 0.53}{\textbf{\color{white} Expérience : }}\\}}
	{\end{experience_env}}

%%%%%%%%%%%%%%%%%%%%%%%%%%%%%%%%%%%%%

%%%%%%%%%%%%%%%%%%%%%%%%%%%%%%%%%%%%%
% Slide

\mdfdefinestyle{s_slide}{%
	linecolor=green_f!,
	outerlinewidth=3pt,%
	frametitlerule=false,
	topline=false,
	bottomline=false,
	rightline=false,
	backgroundcolor=green_c,
	innertopmargin=8pt,
	roundcorner=0pt,
	nobreak=true
}
\newmdenv[style=s_slide]{slide_env}

\newenvironment{slide}
	{%\stepcounter{exa}%
%		\newenvironment{myenv}{\begin{adjustwidth}{2cm}{}}{\end{adjustwidth}}
		\addcontentsline{ldf}{figure}{0}%
		\begin{slide_env}}
		{\end{slide_env}
	}

%%%%%%%%%%%%%%%%%%%%%%%%%%%%%%%%%%%%%

\begin{document}

\maketitle

\section*{Code couleur}

\begin{slide}
Les concepts essentiels à faire passer avec la slide : ce qu'il faut retenir !
\end{slide}

\begin{experience}
Les simulations pour agrémenter la présentation.
\end{experience}

\begin{transition}
Les transitions pour le lien entre les slides.
\end{transition}

\section{Mise en perspective didactique d’un dossier de recherche}

\begin{slide}
Présentation de l'exposé en trois parties :
\begin{itemize}
\item le parcours universitaire ;
\item les travaux de thèse ;
\item la séquence pédagogique.
\end{itemize}
\end{slide}

Bonjour à tous.
Je vais vous présenter la mise en perspective didactique de mon dossier de recherche.
Au cours de ma thèse, je me suis intéressé aux mesures de petits déplacements et j'ai utilisé les deux oscillateurs mécaniques que vous voyez.
On y reviendra un peu plus tard.

\begin{transition}
Je vais commencer par vous présenter mon parcours universitaire.
\end{transition}

\section{Parcours universitaire}

\begin{slide}
Suivre la slide.
Faire le lien entre les différentes étapes : le parcours doit être logique !
\end{slide}

Après le baccalauréat, j'ai été accepté à LLG où j'ai pu bénéficier d'une formation de grande qualité.
J'ai ensuite choisi d'aller en licence pour garder une formation pluridisciplinaire et s'orienter vers l'enseignement et la recherche.
Après, j'ai suivi un master général et me suis spécialisé en optique.
Au cours de ma formation, j'ai fait plusieurs stages dès la L3 ce qui m'a permis de découvrir plusieurs sujets de recherche et plusieurs équipes.
J'ai ensuite effectué ma thèse au LKB, durant laquelle j'ai eu la chance d'avoir une mission d'enseignement pendant trois ans.
Cela a fini de me convaincre que je voulais enseigner et cette année j'ai préparé le concours au centre de Montrouge.

\begin{transition}
Je vais maintenant vous présenter mes travaux de thèse sur le refroidissement de résonateurs mécaniques proches de leur état quantique fondamental.
\end{transition}

\section{Tester la mécanique quantique}

\begin{slide}
\begin{itemize}
\item Deux mondes apparemment distincts : quantique et classique ;
\item Une frontière floue : deux approches complémentaires pour l'explorer ;
\item Masse de Planck et oscillateurs.
\end{itemize}
\end{slide}

Ma thèse s'inscrit dans une démarche qui vise à explorer les limites entre les descriptions classiques et quantiques.
D'un côté la mécanique quantique décrit très bien le comportement parfois surprenant des entités microscopique.
De l'autre, la mécanique classique s'applique aux objets macroscopiques.
Les deux approches sont très différentes, l'une probabiliste, l'autre parfaitement déterministe et la frontière entre les deux n'est pas bien définie.

Pour explorer cette limite, deux approches complémentaires :
\begin{itemize}
\item bottom-up : tester les propriétés quantiques d'objets de plus en plus massifs ;
\item top-bottom : observer des écarts aux prédictions classiques sur des objets de plus en plus petits, comme des oscillateurs mécaniques.
\end{itemize}
La limite quantique/classique est attendue proche de la masse de Planck, au delà de laquelle les propriétés quantiques s'estompent.

\begin{transition}
C'est dans cette deuxième démarche que s'inscrit mon sujet de thèse où l'on s'intéresse au comportements d'oscillateurs mécaniques de masse proche de la masse de Planck, une fois refroidis à très basse température. 
\end{transition}

\section{Mouvement brownien -- État quantique fondamental d'un oscillateur mécanique}

\begin{slide}
\begin{itemize}
\item Oscillateur mécanique classique et quantique ;
\item Mouvement brownien ;
\item État quantique fondamental : fluctuations de point zéro à basse température.
\end{itemize}
\end{slide}

En l'absence de dissipation, l'oscillateur mécanique à un degré de liberté peut être décrit par un système masse-ressort auquel on associe un potentiel harmonique.
Ce problème est très bien connu et on peut le résoudre classiquement et quantiquement.
La résolution quantique aboutit à une discrétisation des niveaux d'énergie accessible à l'oscillateur et à l'existence d'un état fondamental d'énergie non nulle.

Maintenant, si l'oscillateur est dans un environnement chaud, l'agitation thermique microscopique se traduit à l'échelle macroscopique comme des fluctuations de position de l'oscillateur : le mouvement brownien.
À haute température, c'est à dire si l'énergie thermique est bien plus grande que l'écart entre les niveaux de l'oscillateur quantique, les deux théories donnent les mêmes résultats et l'amplitude des fluctuations dépend de la température.

En revanche, à basse température c'est à dire lorsque l'énergie thermique est comparable à l'écart entre les niveaux de l'oscillateur quantique, la mécanique classique prédit que l'oscillateur va s'immobiliser au font du puits.
En revanche la mécanique quantique, du fait de l'état d'énergie minimale non nulle prévoit des fluctuations de position résiduelles : les fluctuations de point zéro.

Pour observer cette différence de comportement, on souhaite refroidir un oscillateur mécanique proche de la température quantique définie comme le rapport entre l'écart entre deux niveaux d'énergie et $k_B$ : préparer un oscillateur mécanique dans son état quantique fondamental.

\begin{transition}
Un tel refroidissement n'est pas simple à obtenir.
Il faut entre autre disposer d'oscillateurs avec de bonne propriétés.
\end{transition}

\section{Le micro-pilier en quartz en quelques chiffres}

\begin{slide}
\begin{itemize}
\item Masse proche de la masse de Planck ;
\item Présentation de la géométrie : matériau, mode de compression ;
\item Fréquence imposée par la masse et l'allure de la structure : donne la température quantique à atteindre ;
\item Très bon facteur de qualité. 
\end{itemize}
\end{slide}

Je vais vous présenter un seul des deux oscillateurs que l'on voyait sur la première slide.

\begin{transition}
La fréquence et le facteur de qualité sont déterminés en mesurant la réponse de l'oscillateur à une excitation forcée.
La masse elle est accessible grace à la mesure du mouvement brownien.
\end{transition}

\section{Mouvement brownien et thermométrie}

\begin{slide}
\begin{itemize}
\item Spectre des fluctuations de position du mouvement brownien ;
\item Calibration de la masse effective à température ambiante ;
\item Thermométrie à basse température ;
\item Amplitude des fluctuations de position très faible !
\end{itemize}
\end{slide}

Comme les fluctuations de position de l'oscillateur sont concentrées au voisinage de sa fréquence propre, on peut s'intéresser au spectre de ses  fluctuations plutôt qu'à l'évolution temporelle des déplacements de l'oscillateur.

On exploite une propriété particulièrement intéressante de ce spectre : l'aire sous la courbe dépend des caractéristiques de l'oscillateur et en particulier du rapport $T/m_\mathrm{eff}$.
À température ambiante, on connait précisément la température de l'oscillateur grâce à un thermomètre situé à proximité et on peut en déduire la masse effective de l'oscillateur.

Pour le refroidissement, l'oscillateur est placé dans un cryostat et une fois refroidi, la température est beaucoup plus dure à mesurer avec des thermomètres conventionnels.
Comme la masse effective ne change pas, la mesure du mouvement brownien donne accès à la température de l'oscillateur.

L'échelle n'est pas facile à interpréter alors pour se donner une idée de l'amplitude des déplacement de l'oscillateur on peut regarder l'amplitude des fluctuations de point zéro : $\unit{8\times10^{-18}}{\meter}$ !

\begin{transition}
Les mesures de déplacement doivent être extrêmement sensibles : on utilise des méthodes interférométriques.
\end{transition}

\section{Mesures interférométriques de petits déplacements}

\begin{slide}
\begin{itemize}
\item Mesure interférométrique ;
\item Cavité pour augmenter la sensibilité ;
\item Grande finesse.
\end{itemize}
\end{slide}

Dans un interféromètre de Michelson, un déplacement d'un miroir provoque un déphasage du faisceau réfléchi proportionnel à ce déplacement.
En exploitant le phénomène d'interférence, ce déphasage est converti en variatiation d'intensité mesurable grâce à une photodiode.
En mesurant l'intensité transmise par l'interféromètre, on peut déterminer le déplacement d'un miroir.

La sensibilité de cette mesure ne dépend que de la longueur d'onde du laser utilisé.
On peut augmenter l'effet d'un petit déplacement en utilisant une cavité Fabry-Perot de grande finesse.
Si la cavité est résonante, le déphasage est amplifié d'un facteur $\cal{F}$ qui correspond environ au nombre d'aller-retour de la lumière dans la cavité.
En comparant le faisceau réfléchi par la cavité à un faisceau de référence on peut à nouveau convertir le déphasage en variation d'intensité.

Il est alors possible de mesurer des déplacements très faibles en utilisant des cavité de grande finesse.

\begin{transition}
L'obtention de grandes finesses nécessite d'avoir des miroirs de grandes qualité, ce qui n'a rien d'évident.
\end{transition}

\section{La cavité optomécanique}

\begin{slide}
\begin{itemize}
\item Faible dimension du miroir des oscillateurs : micro-cavité ;
\item Fabrication des miroirs d'entrée ;
\item Cavité optomécanique.
\end{itemize}
\end{slide}

Pour mesurer la position de nos oscillateurs on dépose un miroir plan à l'une des extrémité du pilier.
Compte tenu de la faible dimension de l'oscillateur le miroir a un diamètre de l'ordre de \unit{100}{\micro\meter}.
Pour obtenir une cavité de grande finesse, il faut utiliser comme miroir d'entrée de la cavité un miroir concave avec un rayon de courbure de l'ordre du millimètre.
Il est obtenu par photoablation...

Il a ensuite fallu développer un ensemble de pièces mécaniques qui permet un alignement rapide et stable des cavités.
L'ensemble est ensuite refroidit dans un cryostat à dilution.

On obtient une cavité optomécanique : une cavité optique dont l'un des miroirs est mobile.

\begin{transition}
La température minimale du cryostat est encore bien trop élevée pour atteindre l'état quantique fondamental.
On va utiliser le couplage optomécanique
\end{transition}

\section{Couplage optomécanique et refroidissement}

\begin{slide}
\begin{itemize}
\item Couplage optomécanique avec l'effet de ressort optique par l'intermédiaire de la pression de radiation;
\item Retard : force proportionnelle à $v$.
\item Friction froide
\end{itemize}
\end{slide}

\begin{transition}
Le problème de cette méthode est que le refroidissement est limité par les paramètres optiques et mécaniques de la cavité et la puissance laser injectable.
Il n'est pas possible de l'augmenter indéfiniment.
\end{transition}

\section{Refroidissement par rétroaction}

\begin{slide}
\begin{itemize}
\item Méthode active en mesurant les déplacements de l'oscillateur et appliquant une force proportionnelle à $v$ uniquement ;
\item Utilisation de la pression de radiation ;
\item Rétroaction, refroidissement plus important en augmentant le gain ;
\item Limitation due au bruit de détection.
\end{itemize}
\end{slide}

\begin{transition}
On a implémenté cette méthode ce qui nous a permis de refroidir notre échantillon.
\end{transition}

\section{Résultats}

\begin{slide}
\begin{itemize}
\item On mesure le spectre des fluctuations de position pour plusieurs valeurs du gain de la rétroaction ;
\item On extrait la température de chaque spectre ;
\item Gain optimal : \unit{1{,}0}{\kelvin} pour 5{,}5 phonons pour un objet massif !
\end{itemize}
\end{slide}

\begin{transition}
C'est le bruit de phase qui limite la sensibilité de la mesure et le refroidissement.
\end{transition}

\section{Vers une amélioration de la sensibilité des mesures de déplacement}

\begin{slide}
\begin{itemize}
\item Monde parfait mesure sans bruit et on mesure seulement les déplacements de l'oscillateur.
\item Cependant il y a des bruits de mesure.
\item Réduction du bruit thermique : situation analogue aux interféromètres gravitationnels.
\end{itemize}
\end{slide}

\section{Photo}

\begin{slide}
\begin{itemize}
\item Grande diversité des thèmes abordés.
\end{itemize}
\end{slide}

C'est la transition entre la partie scientifique et la partie mise en perspective.

\begin{transition}
Je vais maintenant m'attarder sur les concepts liés à l'oscillateur en vous présenter une séquence pédagogique que je proposerai en première année de CPGE.
\end{transition}

\section{Valorisation -- Oscillateurs et micro-pilier }

\begin{slide}
\end{slide}

\begin{transition}
\end{transition}

\section{Valorisation -- Cours et TD}

\begin{slide}
\end{slide}

\begin{experience}
\end{experience}

\begin{transition}
\end{transition}

\section{Valorisation -- TP : caractérisation mécanique}

\begin{slide}
\end{slide}

\begin{transition}
\end{transition}

\section{Valorisation -- TP : fonction de transfert}

\begin{slide}
\end{slide}

\begin{transition}
\end{transition}

\section{Vidéo}

\begin{slide}
\end{slide}

\begin{transition}
\end{transition}

\section{Valorisation -- TP : étude spectrale}

\begin{slide}
\end{slide}

\begin{transition}
\end{transition}

\section{Valorisation -- Évaluation}

\begin{slide}
\end{slide}

\begin{transition}
\end{transition}

\section{Merci pour votre attention !}

\end{document}