\documentclass[12pt,a4paper]{article}
\usepackage[utf8]{inputenc}
\usepackage[french]{babel}
\usepackage[T1]{fontenc}
\usepackage{amsmath}
\usepackage{amsfonts}
\usepackage{amssymb}
\usepackage{graphicx}
\usepackage[left=2cm,right=2cm,top=2cm,bottom=2cm]{geometry}
\usepackage[thinspace,thinqspace,amssymb]{SIunits}
\usepackage{eurosym}

\usepackage{hyperref}
\hypersetup{
    colorlinks=true,
    urlcolor=theme
}

\title{Notes pour la présentation du dossier de mise en perspective didactique d'un dossier de recherche}
\author{\textbf{Rémi Metzdorff}}
\date{Concours externe spécial de l'agrégation de physique-chimie option physique Session 2020}
%\date{2020}

\renewcommand{\d}{\mathrm{d}}
\newcommand{\uroc}{\micro RoC}

%%%%%%%%%%%%%%%%%%%%%%%%%%%%%%  For CV  %%%%%%%%%%%%%%%%%%%%%%%%%%%%%%
\usepackage{array}
%\newcolumntype{L}[1]{>{\raggedright\let\newline\\\arraybackslash\hspace{0pt}}m{#1}}
%\newcolumntype{C}[1]{>{\centering\let\newline\\\arraybackslash\hspace{0pt}}m{#1}}
%\newcolumntype{R}[1]{>{\raggedleft\let\newline\\\arraybackslash\hspace{0pt}}m{#1}}
%%%%%%%%%%%%%%%%%%%%%%%%%%%%%%%%%%%%%%%%%%%%%%%%%%%%%%%%%%%%%%%%%%%%%%

\usepackage{xcolor}
\definecolor{theme}{RGB}{56,115,179}
\usepackage[font=small]{caption}

\definecolor{gray_f}{RGB}{68,84,106}
\definecolor{gray_c}{RGB}{214,220,229}
\definecolor{bleu_f}{RGB}{91,155,213}
\definecolor{bleu_c}{RGB}{222,235,247}
\definecolor{red_f}{RGB}{204,0,0}
\definecolor{red_c}{RGB}{245,204,204}
\definecolor{orange_f}{RGB}{237,125,49}
\definecolor{orange_c}{RGB}{251,229,214}
\definecolor{green_f}{RGB}{112,173,71}
\definecolor{green_c}{RGB}{226,240,217}
\definecolor{yellow_f}{RGB}{255,192,0}
\definecolor{yellow_c}{RGB}{255,242,204}

%%%%%%%%%%%%%%%%%%%%%%%%%%%%%%%%%%%%%%%%%%%%%%%%%%%%%%%%%%%%%%%%%%%%%%
\usepackage[framemethod=tikz]{mdframed}

\mdfdefinestyle{s_mep}{
	linecolor=theme!,
	outerlinewidth=3pt,
	frametitlerule=false,
	topline=false,
	bottomline=false,
	rightline=false,
	backgroundcolor=white,
	innertopmargin=5pt,
	roundcorner=0pt,
	%skipabove=5pt,
	skipbelow=5pt
}
\newmdenv[style=s_mep]{mep_env}

\newenvironment{mep}{%\stepcounter{exa}%
%\newenvironment{myenv}{\begin{adjustwidth}{2cm}{}}{\end{adjustwidth}}
\addcontentsline{ldf}{figure}{0}%
\begin{mep_env}
\small}
{\end{mep_env}}
%%%%%%%%%%%%%%%%%%%%%%%%%%%%%%%%%%%%%%%%%%%%%%%%%%%%%%%%%%%%%%%%%%%%%%

%%%%%%%%%%%%%%%%%%%%%%%%%%%%%%%%%%%%%
% Transition

\mdfdefinestyle{s_trans}{%
	linecolor=yellow_f!,
	outerlinewidth=3pt,%
	frametitlerule=false,
	topline=false,
	bottomline=false,
	rightline=false,
	backgroundcolor=yellow_c,
	innertopmargin=8pt,
	roundcorner=0pt,
	nobreak=true
}
\newmdenv[style=s_trans]{transition2_env}
\newenvironment{transition}
{%\stepcounter{exa}%
	\addcontentsline{ldf}{figure}{0}%
	\begin{transition2_env}}
	{\end{transition2_env}}

%%%%%%%%%%%%%%%%%%%%%%%%%%%%%%%%%%%%%

%%%%%%%%%%%%%%%%%%%%%%%%%%%%%%%%%%%%%
% Experience

\mdfdefinestyle{s_experience}{%
	linecolor=bleu_f!,
	outerlinewidth=3pt,%
	frametitlerule=false,
	topline=false,
	bottomline=false,
	rightline=false,
	backgroundcolor=bleu_c,
	innertopmargin=8pt,
	roundcorner=0pt,
	nobreak=true
}
\newmdenv[style=s_experience]{experience_env}
\newenvironment{experience}
{%\stepcounter{exa}%
	\addcontentsline{ldf}{figure}{0}%
	\begin{experience_env}}
%	\begin{experience_env}[]{\noindent\colorbox[rgb]{0.1 0.1 0.53}{\textbf{\color{white} Expérience : }}\\}}
	{\end{experience_env}}

%%%%%%%%%%%%%%%%%%%%%%%%%%%%%%%%%%%%%

%%%%%%%%%%%%%%%%%%%%%%%%%%%%%%%%%%%%%
% Slide

\mdfdefinestyle{s_slide}{%
	linecolor=green_f!,
	outerlinewidth=3pt,%
	frametitlerule=false,
	topline=false,
	bottomline=false,
	rightline=false,
	backgroundcolor=green_c,
	innertopmargin=8pt,
	roundcorner=0pt,
	nobreak=true
}
\newmdenv[style=s_slide]{slide_env}

\newenvironment{slide}
	{%\stepcounter{exa}%
%		\newenvironment{myenv}{\begin{adjustwidth}{2cm}{}}{\end{adjustwidth}}
		\addcontentsline{ldf}{figure}{0}%
		\begin{slide_env}}
		{\end{slide_env}
	}

%%%%%%%%%%%%%%%%%%%%%%%%%%%%%%%%%%%%%

\begin{document}

\maketitle

\section*{Code couleur}

\begin{slide}
Les concepts essentiels à faire passer avec la slide : ce qu'il faut retenir !
\end{slide}

\begin{experience}
Les simulations pour agrémenter la présentation.
\end{experience}

\begin{transition}
Les transitions pour le lien entre les slides.
\end{transition}

\section{Mise en perspective didactique d’un dossier de recherche}

\begin{slide}
Présentation de l'exposé en trois parties :
\begin{itemize}
\item le parcours universitaire ;
\item les travaux de thèse ;
\item la séquence pédagogique.
\end{itemize}
\end{slide}

\begin{transition}
\end{transition}

\section{Parcours universitaire}

\begin{slide}
\end{slide}

\begin{transition}
\end{transition}

\section{Tester la mécanique quantique}

\begin{slide}
\end{slide}

\begin{transition}
\end{transition}

\section{Mouvement brownien -- État quantique fondamental d’un oscillateur mécanique}

\begin{slide}
\end{slide}

\begin{transition}
\end{transition}

\section{Le micro-pilier en quartz en quelques chiffres}

\begin{slide}
\end{slide}

\begin{transition}
\end{transition}

\section{Mouvement brownien et thermométrie}

\begin{slide}
\end{slide}

\begin{transition}
\end{transition}

\section{Mesures interférométriques de petits déplacements}

\begin{slide}
\end{slide}

\begin{transition}
\end{transition}

\section{La cavité optomécanique}

\begin{slide}
\end{slide}

\begin{transition}
\end{transition}

\section{Couplage optomécanique et refroidissement}

\begin{slide}
\end{slide}

\begin{transition}
\end{transition}

\section{Refroidissement par rétroaction}

\begin{slide}
\end{slide}

\begin{transition}
\end{transition}

\section{Résultats}

\begin{slide}
\end{slide}

\begin{transition}
\end{transition}

\section{Vers une amélioration de la sensibilité des mesures de déplacement}

\begin{slide}
\end{slide}

\begin{transition}
\end{transition}

\section{Photo}

\begin{slide}
\end{slide}

\begin{transition}
\end{transition}

\section{Valorisation -- Oscillateurs et micro-pilier }

\begin{slide}
\end{slide}

\begin{transition}
\end{transition}

\section{Valorisation -- Cours et TD}

\begin{slide}
\end{slide}

\begin{experience}
\end{experience}

\begin{transition}
\end{transition}

\section{Valorisation -- TP : caractérisation mécanique}

\begin{slide}
\end{slide}

\begin{transition}
\end{transition}

\section{Valorisation -- TP : fonction de transfert}

\begin{slide}
\end{slide}

\begin{transition}
\end{transition}

\section{Vidéo}

\begin{slide}
\end{slide}

\begin{transition}
\end{transition}

\section{Valorisation -- TP : étude spectrale}

\begin{slide}
\end{slide}

\begin{transition}
\end{transition}

\section{Valorisation -- Évaluation}

\begin{slide}
\end{slide}

\begin{transition}
\end{transition}

\section{Merci pour votre attention !}

\end{document}