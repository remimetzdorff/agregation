%\iffalse
%<*dtx>
\ProvidesFile
%========================================================================
                        {MAKEBST.DTX}
%========================================================================
%</dtx>
%% Copyright 1993-2003 Patrick W Daly
%% Max-Planck-Institut f\"ur Sonnensystemforschung
%% Max-Planck-Str. 2
%% D-37191 Katlenburg-Lindau
%% Germany
%% E-mail: daly@mps.mpg.de
%%
%% With additions by Arthur Ogawa
%% E-mai: ogawa@teleport.com
%%
%% This program can be redistributed and/or modified under the terms
%% of the LaTeX Project Public License Distributed from CTAN
%% archives in directory macros/latex/base/lppl.txt; either
%% version 1 of the License, or any later version.
%%
%% This is a contributed file to the LaTeX2e system.
%% -------------------------------------------------
%  This is a (La)TeX program that generates docstrip batch jobs
%    that may be used to produce customized bibliographic style files.
%     Docstrip options available:
%        program - to generate the program file, makebst.tex
%        optlist - (with program) to list unused options in .dbj file
%        optverbose - (with program and optlist) to add verbose comments to .dbj file
%        driver  - to produce a LaTeX2e driver file to print the documentation
%  Installation:
%    LaTeX this file: ==> docstrip installation file makebst.ins
%                         AND the (LaTeX2e) documentation
%    (La)TeX makebst.ins: ==> `program' file makebst.tex
%    (makebst.ins may be edited as needed)
%-------------------------------------------------------------------
%<program>\def\ProvidesFile#1 [#2 #3 #4]{\def\filename{#1}%
%<program>  \def\fileversion{#3}\def\filedate{#2}}
%<*program>
%\fi
 \ProvidesFile{makebst}
    [2003/09/08 4.1 (PWD, AO)]
%\iffalse
%  Here follows abbreviated usage description for the stripped version
 % This file is to be run under TeX (or even LaTeX)
 % It interactively asks questions about the bibliographic style file
 % that you want to produce. When it is finished, it writes a docstrip
 % driver file that produces that .bst file from the generic .mbs that
 % you specified; optionally, it will call the docstrip run immediately.
 % For details, read the documentation in the source file makebst.dtx.
 %--------------------------------------------------------------------
%</program>
%\fi
% \changes{1.0}{1993 Aug 17}{Initial version}
% \changes{1.1}{1994 May 25}{Change extensions to avoid \LaTeXe\ conflicts}
% \changes{2.0}{1994 Jul 01}{Update documentation to \LaTeXe}
% \changes{2.1}{1994 Dec 29}{Add the \texttt{optlist} option}
% \changes{2.1}{1995 Jan 2}{Read \texttt{.opt} file only exceptionally}
% \changes{3.0}{1995 Mar 15}{Allow \texttt{.mbs} files more freedom, to
%        include other files, for example}
% \changes{3.1}{1997 Apr 29}{Conform (optionally) to \texttt{docstrip} 2.4g}
% \changes{3.2}{1999 Feb 24}{Update copyright notice}
% \changes{4.0}{1999 Jul 20}{AO: show all choices in output file}
% \changes{4.1}{2003 Sep 8}{PWD: Conform to \texttt{docstrip} 2.4}
%
% \CheckSum{698}
% \CharacterTable
%  {Upper-case    \A\B\C\D\E\F\G\H\I\J\K\L\M\N\O\P\Q\R\S\T\U\V\W\X\Y\Z
%   Lower-case    \a\b\c\d\e\f\g\h\i\j\k\l\m\n\o\p\q\r\s\t\u\v\w\x\y\z
%   Digits        \0\1\2\3\4\5\6\7\8\9
%   Exclamation   \!     Double quote  \"     Hash (number) \#
%   Dollar        \$     Percent       \%     Ampersand     \&
%   Acute accent  \'     Left paren    \(     Right paren   \)
%   Asterisk      \*     Plus          \+     Comma         \,
%   Minus         \-     Point         \.     Solidus       \/
%   Colon         \:     Semicolon     \;     Less than     \<
%   Equals        \=     Greater than  \>     Question mark \?
%   Commercial at \@     Left bracket  \[     Backslash     \\
%   Right bracket \]     Circumflex    \^     Underscore    \_
%   Grave accent  \`     Left brace    \{     Vertical bar  \|
%   Right brace   \}     Tilde         \~}
%
%\iffalse
%<*install>
%^^A =============================================
%^^A    Here is the docstrip installation file
%^^A    It is written on first LaTeX run
%^^A =============================================
\begin{filecontents}{makebst.ins}
% Simply TeX or LaTeX this file to extract various files from
%  the source file `makebst.dtx'

% This installation file works with docstrip 2.4 or later.

\input docstrip

\preamble
=============================================
IMPORTANT NOTICE:
This is a generated file.

It is subject to the same copyright conditions (see below)
as in the original file: \inFileName.
It may not be distributed without \inFileName.

Full documentation can be obtained by LaTeXing that original file.
Only a few abbreviated comments remain here to describe the usage.
=============================================
\endpreamble
\postamble
<<<<< End of generated file <<<<<<
\endpostamble
\keepsilent
\askforoverwritefalse

\generate{\file{makebst.tex}{\from{makebst.dtx}{program,%
% Three variants on makebst.tex; select following options.
% 1. the .dbj file contains by default only the list of selected options
%    (Select none of the following)
% 2. the .dbj file contains by default a list of all offered options,
%    with the non-selected ones commented out, add next option.
  optlist,%
% 3. as above but with full comments added for each option, add next option.
%  optverbose%
}}}

\preamble
============================================
This is the driver file to produce the LaTeX documentation
from the original source file \inFileName.

Make changes to it as needed. (Never change the file \inFileName!)
============================================
\endpreamble
\postamble

End of documentation driver file.
\endpostamble

\generate{\file{makebst.drv}{\from{makebst.dtx}{driver}}}
\endbatchfile

\obeyspaces
\Msg{******************************************}%
\Msg{* For documentation, process makebst.dtx *}%
\Msg{*    or the driver file      makebst.drv *}%
\Msg{******************************************}
\end{filecontents}
%</install>
%<*driver>
\NeedsTeXFormat{LaTeX2e}
\documentclass{ltxdoc}
%<driver>%\EnableCrossrefs %Comment out when .ind file ready
  \DisableCrossrefs %May stay; zapped by \EnableCrossrefs
%<driver>%\RecordChanges %Comment out when .gls file ready
%<driver>%\CodelineIndex %Comment out when .ind file ready
  \CodelineNumbered %May stay
%<driver>%\OnlyDescription
\begin{document}
   \DocInput{makebst.dtx}
\end{document}
%</driver>
%\fi
%
% \DoNotIndex{\begin,\CodelineIndex,\CodelineNumbered,\def,\DisableCrossrefs}
% \DoNotIndex{\DocInput,\documentclass,\EnableCrossrefs,\end,\GetFileInfo}
% \DoNotIndex{\NeedsTeXFormat,\OnlyDescription,\RecordChanges,\usepackage}
% \DoNotIndex{\ProvidesClass,\ProvidesPackage,\ProvidesFile,\RequirePackage}
% \DoNotIndex{\LoadClass,\PassOptionsToClass,\PassOptionsToPackage}
% \DoNotIndex{\DeclareOption,\CurrentOption,\ProcessOptions,\ExecuteOptions}
% \DoNotIndex{\AtEndOfClass,\AtEndOfPackage,\AtBeginDocument,\AtEndDocument}
% \DoNotIndex{\InputIfFileExists,\IfFileExists,\ClassError,\PackageError}
% \DoNotIndex{\ClassWarning,\PackageWarning,\ClassWarningNoLine}
% \DoNotIndex{\PackageWarningNoLine,\ClassInfo,\PackageInfo,\MessageBreak}
% \DoNotIndex{\space,\protect,\DeclareRobustCommand,\CheckCommand}
% \DoNotIndex{\newcommand,\renewcommand,\providecommand,\newenvironment}
% \DoNotIndex{\renewenvironment,\newif,\newlength,\newcounter,\setlength}
% \DoNotIndex{\setcounter,\if,\ifx,\ifcase,\ifnum,\ifdim,\else,\fi}
% \DoNotIndex{\texttt,\textbf,\textrm,\textsl,\textsc}
% \DoNotIndex{\textup,\textit,\textmd,\textsf,\emph}
% \DoNotIndex{\ttfamily,\rmfamily,\sffamily,\mdseries,\bfseries,\upshape}
% \DoNotIndex{\slshape,\scshape,\itshape,\em,\LaTeX,\LaTeXe}
% \DoNotIndex{\filename,\fileversion,\filedate,\let,\empty}
% \DoNotIndex{\%,\(,\),\{,\},\@,\@@end,\^,\batchfile,\begingroup,\catcode}
% \DoNotIndex{\closein,\closeout,\csname,\day,\divide,\edef,\endcsname}
% \DoNotIndex{\endgroup,\endinput,\endpostamble,\endpreamble,\expandafter}
% \DoNotIndex{\from,\gdef,\generateFile,\generate,\global,\hours,\ifeof,\immediate}
% \DoNotIndex{\input,\keepsilent,\loop,\month,\multiply,\newcount}
% \DoNotIndex{\newlinechar,\newread,\newwrite,\number,\openin,\openout}
% \DoNotIndex{\or,\par,\postamble,\preamble,\read,\relax,\repeat,\string}
% \DoNotIndex{\temp,\time,\undefined,\write,\year}
% \DoNotIndex{\advance,\today,\minutes,\aftergroup,\askforoverwrite}
% \DoNotIndex{\endbatchfile}
%
% \setcounter{IndexColumns}{2}
% \setlength{\IndexMin}{10cm}
% \setcounter{StandardModuleDepth}{1}
%
% \GetFileInfo{makebst}
% \newcommand\theprog{\texttt{\filename}}
%
% \title{\bfseries Customizing Bibliographic Style Files}
%
% \author{Patrick W. Daly}
%
% \date{This paper describes program \theprog\\
%       version \fileversion{} from \filedate
%   \thanks{Work on \texttt{custom-bib}~4.00 was supported by the
%          American Physical Society}\\[1ex]
%       (including additions by Arthur Ogawa, \texttt{ogawa@teleport.com})
%  }
%
% \maketitle
%
%^^A In order to keep all marginal notes on the one (left) side:
%^^A (otherwise they switch sides disastrously with twoside option)
% \makeatletter \@mparswitchfalse \makeatother
%
% \pagestyle{myheadings}
% \markboth{P. W. DALY}{CUSTOMIZING BIBLIOGRAPHIES}
%
% \newcommand{\btx}{\textsc{Bib}\TeX}
% \newcommand{\dtx}{\textsf{docstrip}}
%
% \parskip=1ex \parindent=0pt
%
% \section{Introduction}
% This \TeX{} program is meant to be used together with generic
% bibliographic style files to produce customized \texttt{.bst} files for
% running with \btx. The generic, or master file, can be processed by
% \dtx{} with selected options to achieve the desired bibliographic
% style. To this end, a \dtx{} batch job should be made up.
% However, because of the large number of options available, an
% interactive, dialogue system would be more convenient.
%
% This program, \theprog, accomplishes this goal.
% It defines macros to establish such a \dtx{} batch job file, and
% to organize a menu of options. The menu information is contained,
% however, in the master file itself, since the two are intimately related.
% Thus different master files with totally different option structures may
% be accommodated.
%
% The batch job could in fact be made up with an editor without calling
% \theprog, but this program does simplify the task.
%
% Incidentally, the \dtx{} run can only be carried out by means of
% a batch job. Running \dtx{} interactively inserts default pre- and
% postambles in the text, the latter including an |\endinput| command that
% \btx{} will not understand.
%
% \section{The Master File}
% The master file is a \btx{} bibliographic style file containing
% alternative coding depending on \dtx{} options. The options are
% selected when \dtx{} is run, either interactively or through a
% batch job.
%
% Suppose that one of the options is called \texttt{xyz}. Then the following
% alternatives are possible:
% \begin{quote}
% |%<xyz> | \em one line of coding
% \end{quote}
% \emph{includes} the single line of coding;
% \begin{quote}
% |%<!xyz> | \em one line of coding
% \end{quote}
% \emph{excludes} the single line;
% \begin{quote}
% |%<*xyz> | \\
% \emph{several lines of coding}\\
% |%</xyz> |
% \end{quote}
% \emph{includes} all the bracketed lines;
% \begin{quote}
% |%<*!xyz> | \\
% \emph{several lines of coding}\\
% |%</!xyz> |
% \end{quote}
% \emph{excludes} all the bracketed lines.
%
% Options may be logically combined: the symbol \verb!|! is a logical
% \textsf{or}, |&| a logical \textsf{and}, |!| a logical \textsf{not};
% parentheses \texttt{(} and \texttt{)} may be used to group options.
%
% \subsection{Using with \dtx}
% (The \dtx\ command syntax shown here is that for version~2.4 and later,
% released December, 1996.)
%
% In order to generate a true \btx{} style file with selected options from
% the master file, it is necessary to run a \dtx{} batch job.
% Suppose that the master file is named \texttt{master.mbs}, the
% resulting \btx{} style file is to be \texttt{silly.bst}, and the batch job
% file itself is called \texttt{silly.dbj}. To produce this with options, say,
% \texttt{xyz} and \texttt{abc}, the batch job would look something like:
% \begin{quote}\begin{verbatim}
% \input docstrip
%
% \preamble
% This is for Journal of Silly Results
% \endpreamble
%
% \postamble
% End of customized bst file
% \endpostamble
%
% \keepsilent
%
% \askforoverwritefalse
%
% \generate{\file{silly.bst}{\from{master.mbs}{xyz,abc}}}
% \endbatchfile
% \end{verbatim}
% \end{quote}
% A preamble is not necessary, although it is advisable to include some
% statement about the application of the bibliographic style. A postamble
% \emph{is} vital, otherwise the default will add |\endinput| at the end
% of the file, something that \btx{} will not understand. The |\keepsilent|
% is optional and just suppresses \dtx{} output during processing.
% Similarly the |\askforoverwritefalse| suppresses the warning that a file of
% the same name is to be overwritten.
%
% \subsection{The Menu File}
% This program, \theprog, simplifies the creation of the batch job file. To
% do that, it needs information on the available options. This information
% must be stored in a special format, in the master file itself.
% Alternatively, that information may be extracted and stored in a file
% with the same root name but extension \texttt{.opt}. \textsl{This feature
% is not recommended since it can lead to inconsistencies!}
% The format of the menu information is illustrated below in
% Section~\ref{sec:menu}.
%
% In the master file, this information must be enclosed within \dtx{}
% options |%<*options>| \dots\ |%</options>| and \emph{must} be ended by an
% |\endoptions| command. It may also include any number of comments.
% The rest of the file must be enclosed within |%<*!options>| \dots\
% |%</!options>| to exclude it when the menu information is extracted
% with \dtx.
%
% A sample menu in the master file to select one or none of options
% \texttt{xyz} \emph{or} \texttt{zyx} would look thus:
% \begin{quote}\begin{verbatim}
% %<*options>
% \mes{Select one of these}
% \optdef{f}{xyz}{Option forword}{to do forward stuff}
% \optdef{r}{zyx}{Option reverse}{to do reverse stuff}
% \optdef{*}{}{None of the above}{}
% \getans
% \endoptions
% %</options>
% %<*!options>
% . . . . . .
% %</!options>
% \end{verbatim}
% \end{quote}
% An explanation of these commands is to be found in Section~\ref{sec:menu}.
%
% The menu information may be extracted from the master file by means of
% \dtx{} and stored in a file with extension \texttt{.opt}. If this
% file is present, \theprog{} offers to read it instead of the master file,
% although this is \emph{not} recommended, as explained above.
%
%
% \section{The Program \protect\theprog}
%
% \subsection{Installing \protect\theprog}
%
% The \theprog\ program comes as a documented source file named
% \theprog\texttt{.dtx}, which needs to be processed by \dtx\ to extract the
% actual `program' file \theprog\texttt{.tex}. The easiest way to do this is
% simply to process the installation batch file \theprog\texttt{.ins} with
% \TeX\ or \LaTeX, as
% \begin{quote}
% \texttt{tex} \theprog\texttt{.ins}\\
% \hspace*{2em} or\\
% \texttt{latex} \theprog\texttt{.ins}
% \end{quote}
%
% There are in fact three variants of \theprog\ that may be extracted: the
% basic one lists by default only those options that have been selected; the
% more refined one (and the default) lists all options offered with the
% rejected ones commented out; the third version also adds more detailed
% comments. Even in the first two versions, the user will be asked
% interactively if s/he wants the additional features of the others.
%
% One can select the variant by editing \theprog\texttt{.ins}.
%
% Another choice that can be made is whether the \texttt{.dbj} files are to
% conform to \dtx\ version~2.4 syntax or not. By default,
% \theprog\texttt{.ins} tests the current version and automatically
% configures \theprog\texttt{.tex} to write the correct syntax. This too may
% be overridden by editing \theprog\texttt{.ins}. (Note that the older syntax
% is still understood by the newer version of \dtx.)
%
% Reminder: the older syntax requires |\def\batchfile|\marg{filename} as
% the first line in a \dtx\ batch job, where \emph{filename} is the name of
% the batch file itself. The newer syntax does not need this line, but
% requires |\endbatchfile| at the very end instead. The advantage of the new
% syntax is that one can edit and rename such a file without having to change
% its name in the first line. The old syntax leads to great frustration if
% one forgets to change \emph{filename}.
%
% Another difference in the syntaxes (actually introduced in version~2.3 in
% June, 1996) is the use of the command |\generate| instead of
% |\generateFile|. Its advantage is that it permits multiple files to be
% extracted in one pass, something that is not exploited at all by \theprog.
%
% \subsection{Running \protect\theprog}
% This is actually a \TeX{} program, although it will also run under \LaTeX.
% In that sense, it is like \dtx{} itself.
% Thus run the program with (something like)
% \begin{quote}
% \texttt{tex} \theprog\\
% \hspace*{2em} or\\
% \texttt{latex} \theprog
% \end{quote}
%
% The program first asks for the name of the master file. This is
% the file containing all possible bibliographic style commands, with
% \dtx{} options for selective output. A default name is offered, as well
% as a default extension (\texttt{.mbs}).
%
% Next, the program asks for the name of the output file, the \texttt{.bst}
% file. The extension here is optional, defaulting to \texttt{.bst}. This name
% also determines the name of the batch job file, which will have the same
% root name with the extension \texttt{.dbj}, for the \emph{\dtx{} batch job}.
%
% The actual interrogation then begins. All the information for the menus
% is contained in the master bibliographic style file.  The reason for this
% is that the menu information must conform to the available options in the
% master file, so it makes sense that one file should contain both. The
% master file is only read up to the |\endoptions| command.
%
% Finally, the batch job file is closed, and the user is asked if it should
% be run.  If he does not take up this
% offer, or if he later edits the batch job, then it may be run manually
% with (something like)
% \begin{quote}\texttt{tex} \emph{bstname}\texttt{.dbj}\end{quote}
%
%
% \section{The Menu Information\label{sec:menu}}
% The set of questions in the interrogation must fit the available
% options in the master file. For this reason, the menu information is
% contained in the master file itself. The program \theprog{} supplies the
% macros that are used in the menu file to simplify writing and processing
% menu information.
%
% \DescribeMacro{\mes}
% To print a message to the terminal, use |\mes|\marg{text}. A new line
% may be forced within \emph{text} by means of |^^J|.
%
% \DescribeMacro{\ask}
% To interrogate the user for a response, use |\ask{\|\emph{com}|}|\marg{text},
% which writes \emph{text} to the terminal, and puts the
% response in the command |\|\emph{com}.
%
% \DescribeMacro{\optdef}
% Almost all interrogations will consist of a list of mutually exclusive
% options, one of which is the default. For each item in the list, one must
% specify the keyboard response that is to select it, the actual name of
% the \dtx{} option that realizes it, and two pieces of explanatory
% text. For example,
% \begin{quote}
% |\optdef{a}{abr}{Abbreviations}{of such words}|
% \end{quote}
% means that \texttt{abr} is the true \dtx{} option name that is
% selected by typing \texttt{a}. The two explanatory texts are written to the
% terminal immediately as part of the menu, but only the first text is
% echoed when the selection is made (for confirmation) and is also written
% to the batch job file (as comment).
%
% The default option must have the response |*|.
%
% A menu is written to the terminal, first with a |\mes| command to state
% the subject matter, and then with a sequence of |\optdef| statements,
% each of which also writes the texts to the terminal.
% \DescribeMacro{\getans}
% The response is then read in and processed with |\getans|, which writes
% the reply to the command |\ans| and writes the appropriate \dtx{}
% option to the batch job file. If the response does not correspond to any
% of those in the menu list, it is set to {\tt*}; if there is no {\tt*}
% in the list, then |\ans| is set to the last entry. The command |\ans|
% is still available afterwards for any extra testing that might be needed.
%
% An example menu appears as follows:
% \begin{quote}\begin{verbatim}
% \mes{^^JJOURNAL VOLUME NUMBER:}
% \optdef{*}{}{Volume plain}{as vol(num)}
% \optdef{i}{vol-it}{Volume italic}%
%        {as {\string\em\space vol}(num)}
% \optdef{b}{vol-bf}{Volume bold}%
%       {as {\string\bf\space vol}(num)}
% \optdef{d}{vol-2bf}{Volume and number bold}%
%       {as {\string\bf\space vol(num)}}
% \getans
% \end{verbatim}
% \end{quote}
%
% \DescribeMacro{\beginoptiongroup}
% Further structure for the interrogation is provided by the
% |\beginoptiongroup| \dots |\endoptiongroup| sequence,
% which should act as a container for the |\mes| \dots
% |\optdef| \dots |\getans| commands described above.
% For example:
% \begin{quote}
% \begin{verbatim}
% \beginoptiongroup{JOURNAL VOLUME NUMBER:}{}
% \optdef{*}{}{Volume plain}{as vol(num)}
% \optdef{i}{vol-it}{Volume italic}%
%        {as {\string\em\space vol}(num)}
% \optdef{b}{vol-bf}{Volume bold}%
%       {as {\string\bf\space vol}(num)}
% \optdef{d}{vol-2bf}{Volume and number bold}%
%       {as {\string\bf\space vol(num)}}
% \getans
% \endoptiongroup
% \end{verbatim}
% \end{quote}
% presents the same effect as the previous example.
% The virtue of the option group is in providing a single markup for all
% interrogations and having a consistent appearance in the generated file.
%
% This feature has been added with version~4.0 of \theprog.
%
%
% \section{More Complex Batch Jobs}
% Version 3.0 of \theprog{} allows the master file to define more
% sophisticated batch jobs, such as additional master files with their own
% options. This is made possible because the options are not written
% directly in the |\generate| command, as in earlier versions, but to a
% command |\MBopts|. The batch file then contains something like:
% \begin{quote}
% |\def\MBopts{\from|\marg{source.ext}|{%|\\
% \hspace*{1em}\emph{lines from menu session}\\
% | }}|\\
% |\generate{\file|\marg{output.ext}|{\MBopts}}|
% \end{quote}
%
% Normally the \emph{lines from menu session} contain just the \dtx{}
% options. However, the master file could add other things to the
% definition of |\MBopts|, even closing it and starting a new definition.
% It just has to make sure that the braces are balanced.
%
% \DescribeMacro{\MBaskfile}
% A number a macros are provided, which are used by \theprog{} itself, to
% simplify making complex menus. To ask for the name of a file
% interactively,
% \begin{quote}
% |\MBaskfile|\marg{Prompting text}\parg{root.ext}\marg{io}|\|\emph{fname}
% \end{quote}
% may be given, where \emph{root.ext} is the default name of the file,
% \emph{io} is \texttt{i} (for input) if the file must already exist, and
% |\|\emph{fname} is the command that receives the file name. The root name
% will be in |\froot|, the extension in |\fext|.
%
% \DescribeMacro{\wr}
% Text is written to the batch job file with
% \begin{quote}
% |\wr|\marg{text}
% \end{quote}
% Any commands in \emph{text} that are to be written literally must be
% preceded by |\string|.
%
% \DescribeMacro{\MBswitch}
% Since any braces in \emph{text} must be balanced, something special must
% be done to permit them to be printed as normal characters. The command
% |\MBswitch| accomplishes this; the parentheses \texttt{( )} replace |{ }|
% as the delimiters. This should always be given within |\begingroup| \dots
% |\endgroup|.
%
% As an example, suppose the master file contains only half the coding for
% the \texttt{.bst} file, the other half being in one of several other
% master files. We must prompt for this second file, include it for its
% options, and make sure that |\MBopts| knows about it. The following code
% in the master file will do this.
% \begin{quote}
% |\MBaskfile{Name of second master file}(aa.mbs)i\xfile|\\
% |\begingroup\MBswitch|\\
% |\wr(\string\MBopta})|\\
% |\wr(\string\from{\xfile}{\string\MBoptb}})|\\
% |\wr(\string\def\string\MBopta{\pc)|\\
% |\endgroup|\\
% \emph{regular menu information for first file}\\
% |\begingroup\MBswitch|\\
% |\wr(}\string\def\string\MBoptb{\pc)|\\
% |\endgroup|\\
% |\input\xfile\relax|\\
% |\begingroup\MBswitch|\\
% |\wr({\pc)|\\
% |\endgroup|\\
% |\endoptions|
% \end{quote}
%
% The resulting \texttt{.dbj} file contains
% \begin{quote}
% |\def\MBopts{\from{first.mbs}{%|\\
% |\MBopta}|\\
% |\from{second.mbs}{\MBoptb}}|\\
% |\def\MBopta{%|\\
% \emph{first set of options}\\
% |}\def\MBoptb{%|\\
% \emph{second set of options}\\
% |{%|\\
% |  }}|\\
% |\generate{\file{sample.bst}{\MBopts}}|
% \end{quote}
%
% \StopEventually{\PrintIndex\PrintChanges}
%
% \section{Coding}
% This section presents and explains the actual coding of the macros.
% It is nested between |%<*program>| and |%</program>|, which
% are indicators to \dtx{} that this coding belongs to the program
% file.
%
% \subsection{Preliminaries}
% The first thing is to open up i/o devices for communicating with the
% terminal and files. (Some of this has been borrowed from \dtx{}.)
% The terminal input and output are |\ttyin| and |\ttyout|
% respectively, while the output file if |\outfile|.
%    \begin{macrocode}
%<*program>
\newwrite\outfile
\newread\ttyin
\newread\infile
\newwrite\ttyout
%    \end{macrocode}
% \begin{macro}{\mes}\begin{macro}{\wr}\begin{macro}{\umes}
% \changes{4.0}{1999 July 20}{AO: define \cs{umes}}
% The commands for outputting text are defined: |\mes| writes to the
% terminal, |\wr| writes it argument directly to the output file,
% while |\umes| writes to the terminal and adds its argument as a comment
% to the output file.
%    \begin{macrocode}
\def\mes{\immediate\write\ttyout}
\def\wr#1{\immediate\write\outfile{#1}}
\def\umes#1{\mes{^^J#1}\wr{\pc#1}}%
%    \end{macrocode}
% \end{macro}\end{macro}\end{macro}
%
% To assist inserting new lines in the middle of text, define a newline
% symbol.
%    \begin{macrocode}
\newlinechar=`\^^J
%    \end{macrocode}
%
% \begin{macro}{\MBswitch}
% \changes{3.0}{1995 Feb 5}{Add macro}
% There are times when we need to write a line of code to the output file
% with unbalanced braces in that line. (They are balanced in another line.)
% Such lines are written with |\wr{...}|. If the braces in the argument are
% not balanced, then there will be trouble. To get around this, change the
% category codes of the braces to `other' and let parentheses take their
% place.
%    \begin{macrocode}
\def\MBswitch{\catcode`\{=12 \catcode`\}=12
              \catcode`\(=1 \catcode`\)=2\relax}
%    \end{macrocode}
% The way to employ this is as
% \begin{quote}\begin{verbatim}
% \begingroup\MBswitch
% \wr(..{..)
% \endgroup
% \end{verbatim}
% \end{quote}
% \end{macro}
%
% \begin{macro}{\ask}
% To get a response from the terminal, use |\ask|. However, there are some
% complications here. If only carriage-return is pressed, then the response
% command is equal to |\par|; for anything else, a typed-in text includes a
% trailing blank. We must test for |\par| and remove the blank if it is
% there.
%    \begin{macrocode}
\def\defpar{\par}
\def\remblk#1 @@{#1}
\def\ask#1#2{\mes{#2}\read\ttyin to #1\ifx#1\defpar\def#1{}\else
   \edef#1{\expandafter\remblk#1@@}\fi}
%    \end{macrocode}
% \end{macro}
%
% \begin{macro}{\getroot}
% \begin{macro}{\getext}
% To parse the name of a file into root and extension, use commands
% |\getroot| and |\getext|.
%    \begin{macrocode}
\def\groot#1.#2@@{#1}
\def\getroot#1{\expandafter\groot#1.@@}
\def\gext#1.#2.#3@@{#2}
\def\getext#1{\expandafter\gext#1..@@}
%    \end{macrocode}
% \end{macro}\end{macro}
%
% \begin{macro}{\MBaskfile}
% \changes{3.0}{1995 Feb 5}{Add macro to get a file name}
% Several times it is necessary to ask for a file name interactively, and
% maybe test if it exists. This might even be done in the \texttt{.mbs}
% file, so provide a macro to simplify this task. The syntax is
% \begin{quote}
% |\MBaskfile|\marg{Prompting text}\parg{root.ext}\marg{io}|\|\emph{fname}
% \end{quote}
% where \emph{root.ext} is the default name for the file sought, and
% |\|\emph{fname} is the command that contains the final file name. The
% commands |\froot| and |\fext| will contain the root and extensions of the
% file name, if they are needed for further parsing. If
% \emph{io}=\texttt{i} (for input), then the resulting file must already
% exist, else the macro loops again. If \emph{root} is blank, then only the
% extension is given as default, but a file root name must be entered.
%    \begin{macrocode}
\def\MBaskfile#1(#2.#3)#4#5{%
\loop
  \def\ans{#2.#3}
\if!#2!
 \if!#3!\ask{#5}{#1}\fi
  \ask{#5}{#1 (default extension=#3)}\else
  \ask{#5}{#1 (default=\ans)}
\fi
  \ifx#5\empty \edef#5{\ans}\fi
  \edef\froot{\getroot#5}
  \edef\fext{\getext#5}
  \ifx\fext\empty \def\fext{#3}\fi
  \edef#5{\froot.\fext}
\if#4i
  \def\temp{Cannot find file `#5'}
  \openin\infile#5\relax
  \ifeof\infile \def\ans{}\fi \closein\infile
\else
 \def\temp{There is no default}
 \ifx\froot\empty \def\ans{}\fi
\fi
  \ifx\ans\empty \mes{*** \temp}
\repeat}
%    \end{macrocode}
% \end{macro}
%
% \begin{macro}{\pc}\begin{macro}{\pcpc}\begin{macro}{\spsp}
% Now for some special commands to simplify outputting \% signs and double
% spaces to the output file.
%    \begin{macrocode}
{\catcode`\%=12
 \gdef\pc{%}
 \gdef\pcpc{%% }
}
\def\spsp{\space\space}
%    \end{macrocode}
% \end{macro}\end{macro}\end{macro}
%
% \begin{macro}{\Now}
% \changes{2.0}{1994 Jul 1}{Make \cmd{\today} output YYYY/MM/DD}
% In order to date-and-time-stamp the resulting batch job file, we need
% macros to produce the current date and time. (In \TeX{} there is no
% |\today| command.)
%    \begin{macrocode}
\newcount\hours
\newcount\minutes
\def\SetTime{\hours=\time
        \global\divide\hours by 60
        \minutes=\hours
        \multiply\minutes by 60
        \advance\minutes by-\time
        \global\multiply\minutes by-1 }
\SetTime
\def\now{\number\hours:\ifnum\minutes<10 0\fi\number\minutes}
\def\today{\number\year/\ifnum\month<10 0\fi\number\month
   /\ifnum\day<10 0\fi\number\day}
\def\Now{\today\space at \now}
%    \end{macrocode}
% \end{macro}
%
%
% \subsection{Menu Macros}
% \begin{macro}{\optdef}
% \changes{4.0}{1999 July 20}{AO: save all of \cs{optdef}'s arguments}
% For each menu, a general text is written with |\mes|, followed by a list
% of available options. The information that will be needed is
% \begin{enumerate}
% \item the response letter to select the option,
% \item the actual \dtx{} option name, as defined in the master
%   bibliographic style file,
% \item a piece of text that is printed in the menu list, to be echoed
%   in confirmation of the choice, and also to be written to batch job file
%   as a comment,
% \item a second piece of text that is only written to the menu, to enhance
%   the explanation.
% \end{enumerate}
% The true option name and the two pieces of text are stored as commands
% prefixed by |\opt@|, |\txt@|, and |\cmt@| respectively, followed by the response
% letter. Each option response letter is also stored in a list |\optlist|
% which is initialized to empty. The commands
% |\nxtopt| and |\rstopt| are used to extract the next and remaining
% options from the list.
%    \begin{macrocode}
\def\optdef#1#2#3#4{%
  \expandafter\def\csname opt@#1\endcsname{#2}%
  \expandafter\def\csname txt@#1\endcsname{#3}%
  \expandafter\def\csname cmt@#1\endcsname{#4}%
  \edef\optlist{\optlist#1,}%
  \def\OptAns{#1}%
  \mes{(#1) #3\space #4}%
}
\def\optlist{}
\def\nxtopt#1,#2@@{#1} \def\rstopt#1,#2@@{#2}
%    \end{macrocode}
% \end{macro}
%
% \begin{macro}{\getans}
% \changes{2.1}{1994 Dec 29}{Allow for \texttt{optlist} option}
% \changes{4.0}{1999 July 20}{AO: review \cs{optlist} forwards (FIFO)}
% The user selection is read in with |\getans|, into the command |\ans|.
% It then processes the response by first checking if there is an option
% corresponding to it; if not, the response |\ans| is set to the default |*|.
% If no star response exists, then it takes the last one entered as the default
% response. It then calls |\wropt| to write the necessary \dtx{} option
% and explanatory comment to the batch job file. Finally, it uses the
% option list |\optlist| to clear all the |\opt@| commands. This last step
% is necessary to avoid conflicts with previous menus: without it, a
% response that is not in the current list might however exist from an
% earlier menu.
%
% Note that prior to version~4.0 of this code, the |\optlist| was built
% via head accretion and traversed from the head back, that is, in LIFO order.
% As of version~4.0 it is processed in FIFO order.
% This way, the comments in the |.dbj| file are in the same order as
% the |\optdef| statements in the master file.
% The flag character (to terminate parsing the |\optlist|) is now
% a |%|$_{12}$, which cannot be entered as a response by the user, and is
% appended to the list at the beginning of |\getans| processing.
%    \begin{macrocode}
\newif\ifsw
\def\getans{%
 \edef\optlist{\optlist\pc,}%
 \ask{\ans}{\spsp Select:}%
  \expandafter\ifx\csname opt@\ans\endcsname\relax
  \def\ans{*}%
 \fi
 \expandafter\ifx\csname opt@\ans\endcsname\relax
  \let\ans\OptAns
 \fi
  \edef\ansx{\csname opt@\ans\endcsname}
  \swtrue \loop
    \edef\temp{\expandafter\nxtopt\optlist@@}%
    \edef\optlist{\expandafter\rstopt\optlist@@}%
  \if\temp\pc\swfalse\else
   \if\temp\ans
    \wropt\ans
   \else
    \ifoptlist\wrxopt\temp\fi
      \fi
      \expandafter\let\csname opt@\temp\endcsname\relax
    \fi
  \ifsw \repeat
 \def\optlist{}%
 \ifoptverbose
  \wr{\pc------\string\ans=\ans (==\ansx)-------}%
 \else
   \ifoptlist
     \wr{\pc--------------------}%
   \fi
 \fi
}
%    \end{macrocode}
% \end{macro}
%
% A special request from Frank Mittelbach asks if a list of unused options
% cannot be added to the batch job file, to assist editing it by hand. In
% this way, one knows what the \dtx{} options are immediately without having to
% search for them in the \texttt{.mbs} documentation.
%
% This feature was added in version~2.1, but by means of a \dtx\ option, so it
% could be turned off if necessary. Thus the extracted \theprog\texttt{.tex}
% file produced either the full list or only the selected options. Here `full
% list' means only those options that were offered. Any options that were
% conditionally offered, depending on previous selections, could be missing.
%
% For version~4.0, Arthur Ogawa changed this so that the full option list is
% switched on with the |\ifoptlist| flag, and not by an option at \dtx\
% extraction time. He also added an |\ifoptverbose| flag to include even more
% comments in the \texttt{.dtx} file. The user is asked at run time if s/he
% wants to activate these features.
%
% Furthermore, one can use the |\beginoptiongroup| \dots |\endoptiongroup|
% idiom to handle cases where options should be offered only contingent upon
% some condition. By doing so, the unused options are still presented
% as comments in the batch job file, along with a comment showing the
% dependency and a matching comment showing the scope.
%
% Finally, there are still \dtx\ options \texttt{optlist} and
% \texttt{optverbose} which produce \theprog\texttt{.tex} with the
% corresponding flags already set, in which case the features are always
% activated and the user interrogation is suppressed.
%
% \begin{macro}{\wropt}
% \changes{2.1}{1994 Dec 29}{Change \cs{temp} to \cs{tempx}}
% \changes{4.0}{1999 July 20}{AO: output the \cs{cmt@} part also;
%                 build \cs{optlist} forwards (FIFO).}
% The actual outputting of the option command to the batch job file is done
% by |\wropt|. It tests if the option name is blank (a default in the
% master bibliographic style, which need not be the menu default), writes
% out the option name, if present, and adds the explanatory comment.
%    \begin{macrocode}
\def\wropt#1{%
 \edef\tempx{\csname opt@#1\endcsname}%
 \if!\tempx!
  \wr{\spsp\spsp\pc: (def)
   \csname txt@#1\endcsname
   \ifoptverbose\space\csname cmt@#1\endcsname\fi
  }%
 \else
  \wr{\spsp\tempx,\pc:
   \csname txt@#1\endcsname
   \ifoptverbose\space\csname cmt@#1\endcsname\fi
  }%
 \fi
 \mes{\spsp You have selected: \csname txt@#1\endcsname}%
}
%    \end{macrocode}
% \end{macro}
%
% \begin{macro}{\wrxopt}
% \changes{2.1}{1994 Dec 29}{Add macro}
% \changes{4.0}{1999 July 20}{AO: output the \cs{cmt@} part also}
% \changes{4.0}{1999 July 20}{AO: build \cs{optlist} forwards (FIFO)}
% Writing the unused options to the batch job file is done with the |\wrxopt|
% command, which is heavily controlled by the flags |\ifoptlist| and
% |\ifoptverbose|.
%
%    \begin{macrocode}
\def\wrxopt#1{%
 \edef\tempx{\csname opt@#1\endcsname}%
 \if!\tempx!
  \wr{\pc\space\spsp\pc: (def)
   \csname txt@#1\endcsname
   \ifoptverbose\space\csname cmt@#1\endcsname\fi
  }%
 \else
  \wr{\pc\space\tempx,\pc:
   \csname txt@#1\endcsname
   \ifoptverbose\space\csname cmt@#1\endcsname\fi
  }%
 \fi
}
%    \end{macrocode}
% \end{macro}
%
% \begin{macro}{\beginoptiongroup}
% \begin{macro}{\endoptiongroup}
% One can structure the master file using the commands
% |\beginoptiongroup| \dots |\endoptiongroup|.
% The |\beginoptiongroup| command takes two arguments,
% the prompt text and a control expression.
% \begin{quote}
% |\beginoptiongroup|\\
% |  {CITATION NUMBERS (if numerical references)}|\\
% |  {\ifnumerical*\fi}|\\
% |\optdef{*}{}{arabic numbers}|\\
% |  {references are numbered 1, 2, 3, etc.}|\\
% |\optdef{d}{d'nai}{d'nai numerals}|\\
% |  {references are numbered base-25}|\\
% |\getans|\\
% \emph{more commands and option groups}\\
% |\endoptiongroup|\\
% \end{quote}
% In the above example, the master file has defined a |\newif| switch called
% |\ifnumerical|, and now uses this flag to enable the processing encapsulated
% within the option group: the control expression executes the active |*| command
% if |\ifnumerical| is true. More complex expressions are possible; use plain
% \TeX\ constructions to expand the star.
%
% The prompt text is output to the console and is also
% recorded as a comment in the generated |.dbj| file.
%
% If the control expression executes the active |*|, then
% the statements within the option group are executed as usual.
% If false, then the |.dbj| file will simply contain a record of the options
% that the user would have been able to chose from.
% In effect, the interrogation never comes: all the options are unused and
% are recorded (via |\wrxopt|) as comments.
%
% By this means, one can structure the |.dbj| file so that all options
% are made visible, even if some of them would not be accessible because of
% internal dependencies. The |.dbj| file will show as much detail about the menus
% of the master file as is possible.
%
% To enable a common idiom, we have caused the value of |\ans| to persist
% past the end of the option group. This means that one may safely test
% the value of |\ans| after the |\endoptiongroup|.
% If processing was turned off within
% the option group, then the value of |\ans| is the untypeable |$|$_{12}$.
%
% If the second argument is either nil or is a star, then the option group
% will be executed normally. Therefore you can employ this structure
% for all the processing involving the commonly used idiom.
% If the master file has statements like:
% \begin{quote}
% \begin{verbatim}
%\mes{PROMPT:}
%\optdef{*}{}{default}{extended comment}%
%\optdef{a}{opt-a}{option a}{another extended comment}%
%\optdef{b}{opt-b}{option b}{more extended comments}%
%\getans
%\if\ans a\whatever\fi
% \end{verbatim}
% \end{quote}
% they should be converted to:
% \begin{quote}
% \begin{verbatim}
%\beginoptiongroup{PROMPT:}{}
%\optdef{*}{}{default}{extended comment}%
%\optdef{a}{opt-a}{option a}{another extended comment}%
%\optdef{b}{opt-b}{option b}{more extended comments}%
%\getans
%\endoptiongroup
%\if\ans a\whatever\fi
% \end{verbatim}
% \end{quote}
% The benefit of this syntax is a single markup for all interrogations
% and a consistent appearance in the generated file.
%    \begin{macrocode}
\def\beginoptiongroup#1{\begingroup\activatestar\OGcontinue{#1}}%
\def\OGcontinue#1#2{%
 \endgroup
 \begingroup
  \let\OGswitch\secondoftwo\def\tempa{#2}%
  \ifx\tempa\empty\expandafter\firstoftwo\else\expandafter\secondoftwo\fi
  {%
   \activestar
  }{%
   \tempa
  }%
  \OGswitch{}{%
    \let\wropt\wrxopt
    \let\ask\nilans
    \def\mes##1{}%
  }%
  \def\OGmessage{#1}%
  \umes{\ifoptverbose<<\fi\OGmessage}%
}
\def\endoptiongroup{%
  \ifoptverbose\umes{>>\OGmessage}\fi
  \aftergroup\let\aftergroup\ans\expandafter
 \endgroup
 \ans
}
\def\activestar{\let\OGswitch\firstoftwo}
\def\activatestar{\catcode`\*13\relax}
{\activatestar\gdef*{\activestar}}
\def\firstoftwo#1#2{#1}
\def\secondoftwo#1#2{#2}
{\catcode`\$=12\gdef\nilans#1#2{\def\ans{$}}}
%    \end{macrocode}
%
% For more examples of using option groups, see the file \texttt{merlin.mbs}.
% \end{macro}\end{macro}
%
%
% \subsection{Initial Messages}
% The program can now start working. It first introduces itself and asks if
% the user wants an explanation of how the menus work.
%    \begin{macrocode}
\mes{***********************************^^J%
     * This is Make Bibliography Style *^^J%
     ***********************************^^J%
     It makes up a docstrip batch job to produce^^J%
     a customized .bst file for running with BibTeX}

\ask{\yn}{Do you want a description of the usage? (NO)}

\if!\yn!\else\if\yn n\else\if\yn N\else
\mes{In the interactive dialogue that follows,^^J%
     you will be presented with a series of menus.^^J%
     In each case, one answer is the default, marked as (*),^^J%
     and a mere carriage-return is sufficient to select it.^^J%
     (If there is no * choice, then the default is the last choice.)^^J%
     For the other choices, a letter is indicated^^J%
     in brackets for selecting that option. If you select^^J%
     a letter not in the list, default is taken.^^J^^J%
     The final output is a file containing a batch job^^J%
     which may be (La)TeXed to produce the desired BibTeX^^J%
     bibliography style file. The batch job may be edited^^J%
     to make minor changes, rather than running this program^^J%
     once again.}
\fi\fi\fi
%    \end{macrocode}
%
% Ask for the name of the master bibliographic style file,
% suggesting a default name. Test if the file exist (argument \texttt{i}).
% The name of the master file is split up into root and extension.
% \changes{3.0}{1995 Mar 15}{Default master file now \texttt{merlin.mbs}}
%    \begin{macrocode}
\MBaskfile{^^JEnter the name of the MASTER file}(merlin.mbs)i\mfile
\let\mroot=\froot
\let\mext=\fext
%    \end{macrocode}
%
% Originally, I intended the menu information to be in an \texttt{.opt}
% file, but this is dangerous: that file may not be consistent with the
% master. So now, issue a warning if an \texttt{.opt} file exists,
% substituting it only if explicitly requested. (This is useful for me
% when testing changes to \theprog{} and I only want a short menu.)
%    \begin{macrocode}
\edef\temp{\mroot.opt}
\openin\infile\temp\relax
\let\mnext=\mext
\ifeof\infile\else
  \ask{\yn}{** Warning: a file `\temp' also exists^^J
       \spsp Shall I read it for the menu information? (NO)^^J
       \spsp (Answer `yes' only if you really know what you are doing)}
\if\yn y\def\mnext{opt}\else\if\yn Y\def\mnext{opt}\fi\fi
\mes{Menu information read from `\mroot.\mnext'}
\fi
\closein\infile
%    \end{macrocode}
%
% Next, the name of the output \texttt{.bst} file is asked for. Here there is
% to be no default for the root part, although the extension defaults to
% \texttt{.bst}.
%    \begin{macrocode}
\MBaskfile{^^JName of the final OUTPUT .bst file?}(.bst)o\ofile
\let\oroot=\froot
\let\oext=\fext
%    \end{macrocode}
%
% A comment line of text is asked for. This will go into the preamble of
% the final \texttt{.bst} file and should describe the nature of the
% bibliographic style, i.e., for which journal(s) it is meant to apply.
%    \begin{macrocode}
\ask{\ans}{^^JGive a comment line to include in the style file.^^J%
           Something like for which journals it is applicable.}
%    \end{macrocode}
%
% \changes{1.1}{1994 May 25}{The \dtx{} driver has extension \texttt{.dbj}
%    instead of \texttt{.drv}}
% The output batch job file is to have the same root name as the output
% file, but with the extension \texttt{.dbj}, for \emph{\dtx{} batch job}.
% This file is opened and the initial contents are written.
%    \begin{macrocode}
\immediate\openout\outfile\oroot.dbj
\wr{\pcpc Driver file to produce \oroot.\oext\space from \mroot.\mext}
\wr{\pcpc Generated with \filename, version \fileversion\space (\filedate)}
\wr{\pcpc Produced on \Now}
\wr{\pcpc}
\wr{\string\input\space docstrip}
\wr{}
\wr{\string\preamble}
\wr{----------------------------------------}
\wr{*** \ans\space ***}
\wr{}
\wr{\string\endpreamble}
\wr{}
\wr{\string\postamble}
\wr{End of customized bst file}
\wr{\string\endpostamble}
\wr{}
\wr{\string\keepsilent}
\wr{}
\wr{\string\askforoverwritefalse}
%    \end{macrocode}
% \changes{3.0}{1995 Feb 5}{Options from master file written to control
%    sequence \cs{MBopts} instead of directly into \cs{generateFile}}
% The options will be written to the output file during the interrogation
% when the master file is input. These options are stored in |\MBopts|.
%
% Note: it is necessary to change the catagory codes of |{| and |}|
% temporarily, and to find substitutes, so that mismatched curly braces
% could be included in the output text. The same thing is done again at the
% end to close the braces finally. This is done with |\MBswitch|.
%    \begin{macrocode}
\begingroup\MBswitch
\wr(\string\def\string\MBopts{\string\from{\mroot.\mext}{\pc)
\endgroup
%    \end{macrocode}
%
% Now each selected option is written on a single line.
%
% \subsection{The Interrogation}
% \changes{2.1}{1995 Jan 2}{Get menu info from master file, not \texttt{.opt}
%   file.}
% The menu information is read in from the master file, or from a file
% with extension \texttt{.opt}, but only if one has explicitly requested
% this. (This is expert stuff; the \texttt{.opt} files should be avoided
% since they might not be up-to-date. Previously they were the default,
% but this has been changed in version~2.1 to avoid confusion.)
%    \begin{macrocode}
\newif\ifoptlist
%<optlist|optverbose>\optlisttrue
\ifoptlist\else
 \ask\yn{Do you want the unused option lines^^J%
         \spsp to appear as comments in the output? (NO)}
 \if!\yn!\else\if\yn n\else\if\yn N\else\optlisttrue\fi\fi\fi
\fi
\newif\ifoptverbose
%<optverbose>\optverbosetrue
\ifoptlist
 \ifoptverbose\else
  \ask\yn{Do you want verbose comments? (NO)}
  \if!\yn!\else\if\yn n\else\if\yn N\else\optverbosetrue\fi\fi\fi
 \fi
\fi
\edef\temp{\mroot.\mnext}
\let\endoptions=\endinput
\input\temp
%    \end{macrocode}
% Note that it is necessary to equate |\endoptions| to |\endinput| in
% case the master file is read in. An |\endinput| command in the master
% file would interfere with the \dtx{} operation, but this indirect
% method gets around that problem.
%
%
% \section{Closing the Output File}
% The output file is closed by writing the final line that closes the
% braces that were opened at the beginning. To this end, the catagory codes
% of |{| and |}| must be temporarily altered, as before.
%    \begin{macrocode}
\begingroup\MBswitch
\wr(\spsp}})
\endgroup
%    \end{macrocode}
%
% Now write the line that processes the options stored in |\MBopts|.
% The batch job file is finished and may be closed.
%    \begin{macrocode}
\wr{\string\generate{\string\file{\oroot.\oext}{\string\MBopts}}}
\wr{\string\endbatchfile}

\immediate\closeout\outfile
\mes{^^JFinished!!^^J%
     Batch job written to file `\oroot.dbj'}
%    \end{macrocode}
%
% The batch job may now be run. It is only necessary to input the file.
% However, the inputting should not occur with a group or within an |\if|
% \dots\ |\fi| clause. Furthermore, under \LaTeX{}, the |\end| command causes
% problems, because it has been redefined; the command |\@@end| contains the
% original |\end|.
%    \begin{macrocode}
\def\ofile{\oroot.dbj}
\ask{\yn}{Shall I now run this batch job? (NO)}
\def\temp{\relax}
\if!\yn!\else\if\yn n\else\if\yn N\else
\def\temp{\input\ofile}\fi\fi\fi
{\catcode`\@=11 \ifx\@@end\undefined\else
  \global\let\end=\@@end\fi}
\temp
\end
%</program>
%    \end{macrocode}
%
% \Finale
